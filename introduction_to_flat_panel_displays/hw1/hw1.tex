\documentclass[12pt, a4paper]{article}

\usepackage[hmargin=2.5cm, vmargin=2cm]{geometry}
\usepackage{amsthm, amssymb, mathtools, yhmath, graphicx}
\usepackage{fontspec, type1cm, titlesec, titling, fancyhdr, tabularx}
\usepackage{color, unicode-math, float, hhline}
\usepackage{parskip}

\usepackage[abbreviations, per-mode=symbol]{siunitx}
\usepackage[CheckSingle, CJKmath]{xeCJK}
\usepackage{CJKulem}
\usepackage{enumitem}
\usepackage{tikz}
\usepackage{circuitikz}
\usepackage{fancyvrb, hyperref}
\usepackage{etoolbox}
\AtBeginEnvironment{Verbatim}{\vspace{-12pt}}
\AtEndEnvironment{Verbatim}{\vspace{2pt}}
%\setCJKmainfont[BoldFont=cwTex Q Hei]{cwTex Q Ming}
%\setCJKsansfont[BoldFont=cwTex Q Hei]{cwTex Q Ming}
%\setCJKmonofont[BoldFont=cwTex Q Hei]{cwTex Q Ming}
\setCJKmainfont[BoldFont=cwTeX Q Hei]{cwTeX Q Ming}

\def\normalsize{\fontsize{12}{18}\selectfont}
\def\large{\fontsize{14}{21}\selectfont}
\def\Large{\fontsize{16}{24}\selectfont}
\def\LARGE{\fontsize{18}{27}\selectfont}
\def\huge{\fontsize{20}{30}\selectfont}

%\titleformat{\section}{\bf\Large}{\arabic{section}}{24pt}{}
%\titleformat{\subsection}{\large}{\arabic{subsection}.}{12pt}{}
%\titlespacing*{\subsection}{0pt}{0pt}{1.5ex}

\parindent=0pt
\parskip=16pt

\DeclarePairedDelimiter{\abs}{\lvert}{\rvert}
\DeclarePairedDelimiter{\norm}{\lVert}{\rVert}
\DeclarePairedDelimiter{\inpd}{\langle}{\rangle}
\DeclarePairedDelimiter{\ceil}{\lceil}{\rceil}
\DeclarePairedDelimiter{\floor}{\lfloor}{\rfloor}

\newcommand{\img}{\mathsf{i}}
\newcommand{\ex}{\mathsf{e}}
\newcommand{\dD}{\mathrm{d}}
\newcommand{\dI}{\,\mathrm{d}}

\title{平面顯示技術導論 HW \#1}
\author{B02901178 江誠敏}

\begin{document}

\input{ipython.tex}

\maketitle

\begin{enumerate}[label={\bf (\alph*)}, leftmargin=0pt]
  \item {\bf Calculate the (x, y) location of the light source D65 (spectrum is listed with the attached excel file) 
      by finding and normalizing the tristimulus values for a color with a spectral power distribution.} \\

    利用題目給的資料計算,以下用 python 做為主要的計算工具。

    首先將 \texttt{.xls} 檔讀入,並把 \texttt{Sheet1} 中的資料讀出。
    \begin{Verbatim}[commandchars=\\\{\}]
{\color{incolor}In [{\color{incolor}1}]:} \PY{k+kn}{from} \PY{n+nn}{pyexcel\PYZus{}xls} \PY{k}{import} \PY{n}{get\PYZus{}data}
        \PY{n}{data} \PY{o}{=} \PY{n}{get\PYZus{}data}\PY{p}{(}\PY{l+s+s1}{\PYZsq{}}\PY{l+s+s1}{data.xls}\PY{l+s+s1}{\PYZsq{}}\PY{p}{)}\PY{p}{[}\PY{l+s+s1}{\PYZsq{}}\PY{l+s+s1}{Sheet1}\PY{l+s+s1}{\PYZsq{}}\PY{p}{]}
\end{Verbatim}

    觀察一下各行的資訊。
    \begin{Verbatim}[commandchars=\\\{\}]
{\color{incolor}In [{\color{incolor}2}]:} \PY{n+nb}{print}\PY{p}{(}\PY{n}{data}\PY{p}{[}\PY{l+m+mi}{4}\PY{p}{]}\PY{p}{)}
\end{Verbatim}

            \begin{Verbatim}[commandchars=\\\{\}]
{\color{outcolor}Out[{\color{outcolor}2}]:} ['Wavelength (nm)', 'xbar', 'ybar', 'zbar', '', 'wavelength(nm)', 'intensity']
\end{Verbatim}

    之後把數值資料取出。
    \begin{Verbatim}[commandchars=\\\{\}]
{\color{incolor}In [{\color{incolor}3}]:} \PY{n}{rdata} \PY{o}{=} \PY{n}{data}\PY{p}{[}\PY{l+m+mi}{5}\PY{p}{:}\PY{p}{]}
\end{Verbatim}

    接著就把 XYZ tristimulus 的值計算出來,利用
    \[ X_\text{tot} = \sum_\omega \bar{x}(\omega) \cdot I(\omega) \]
    其中 $\bar{x}$ 是各個波長下的 color matching function,$I$ 則是 D65 各個波長的光強度。 

    \begin{Verbatim}[commandchars=\\\{\}]
{\color{incolor}In [{\color{incolor}4}]:} \PY{n}{\PYZus{}}\PY{p}{,} \PY{n}{X}\PY{p}{,} \PY{n}{Y}\PY{p}{,} \PY{n}{Z}\PY{p}{,} \PY{n}{\PYZus{}}\PY{p}{,} \PY{n}{\PYZus{}}\PY{p}{,} \PY{n}{I} \PY{o}{=} \PY{n+nb}{zip}\PY{p}{(}\PY{o}{*}\PY{n}{rdata}\PY{p}{)}
        \PY{n}{N} \PY{o}{=} \PY{n+nb}{len}\PY{p}{(}\PY{n}{X}\PY{p}{)}
\end{Verbatim}

    \begin{Verbatim}[commandchars=\\\{\}]
{\color{incolor}In [{\color{incolor}5}]:} \PY{n}{xtot}\PY{p}{,} \PY{n}{ytot}\PY{p}{,} \PY{n}{ztot} \PY{o}{=} \PY{l+m+mi}{0}\PY{p}{,} \PY{l+m+mi}{0}\PY{p}{,} \PY{l+m+mi}{0}
\end{Verbatim}

    \begin{Verbatim}[commandchars=\\\{\}]
{\color{incolor}In [{\color{incolor}6}]:} \PY{k}{for} \PY{n}{i} \PY{o+ow}{in} \PY{n+nb}{range}\PY{p}{(}\PY{n}{N}\PY{p}{)}\PY{p}{:}
            \PY{n}{xtot} \PY{o}{+}\PY{o}{=} \PY{n}{X}\PY{p}{[}\PY{n}{i}\PY{p}{]} \PY{o}{*} \PY{n}{I}\PY{p}{[}\PY{n}{i}\PY{p}{]}
            \PY{n}{ytot} \PY{o}{+}\PY{o}{=} \PY{n}{Y}\PY{p}{[}\PY{n}{i}\PY{p}{]} \PY{o}{*} \PY{n}{I}\PY{p}{[}\PY{n}{i}\PY{p}{]}
            \PY{n}{ztot} \PY{o}{+}\PY{o}{=} \PY{n}{Z}\PY{p}{[}\PY{n}{i}\PY{p}{]} \PY{o}{*} \PY{n}{I}\PY{p}{[}\PY{n}{i}\PY{p}{]}
\end{Verbatim}

\clearpage
    最後得出 $X_\text{tot}, Y_\text{tot}, Z_\text{tot}$
    \begin{Verbatim}[commandchars=\\\{\}]
{\color{incolor}In [{\color{incolor}7}]:} \PY{l+s+s1}{\PYZsq{}}\PY{l+s+s1}{xtot, ytot, ztot = }\PY{l+s+si}{\PYZpc{}.6f}\PY{l+s+s1}{, }\PY{l+s+si}{\PYZpc{}.6f}\PY{l+s+s1}{, }\PY{l+s+si}{\PYZpc{}.6f}\PY{l+s+s1}{\PYZsq{}} \PY{o}{\PYZpc{}} \PY{p}{(}\PY{n}{xtot}\PY{p}{,} \PY{n}{ytot}\PY{p}{,} \PY{n}{ztot}\PY{p}{)}
\end{Verbatim}

            \begin{Verbatim}[commandchars=\\\{\}]
{\color{outcolor}Out[{\color{outcolor}7}]:} 'xtot, ytot, ztot = 2008.773018, 2113.459438, 2301.492113'
\end{Verbatim}


    接著 $x, y$ 可由 
    \[ x = \frac{X_\text{tot}}{X_\text{tot} + Y_\text{tot} + Z_\text{tot}}, \quad
       y = \frac{Y_\text{tot}}{X_\text{tot} + Y_\text{tot} + Z_\text{tot}} \]
     得到
    \begin{Verbatim}[commandchars=\\\{\}]
{\color{incolor}In [{\color{incolor}8}]:} \PY{n}{sm} \PY{o}{=} \PY{n}{xtot} \PY{o}{+} \PY{n}{ytot} \PY{o}{+} \PY{n}{ztot}
\end{Verbatim}

    \begin{Verbatim}[commandchars=\\\{\}]
{\color{incolor}In [{\color{incolor}9}]:} \PY{n}{x}\PY{p}{,} \PY{n}{y} \PY{o}{=} \PY{n}{xtot}\PY{o}{/}\PY{n}{sm}\PY{p}{,} \PY{n}{ytot}\PY{o}{/}\PY{n}{sm}
\end{Verbatim}

    \begin{Verbatim}[commandchars=\\\{\}]
{\color{incolor}In [{\color{incolor}10}]:} \PY{l+s+s1}{\PYZsq{}}\PY{l+s+s1}{x = }\PY{l+s+si}{\PYZpc{}.6f}\PY{l+s+s1}{, y = }\PY{l+s+si}{\PYZpc{}.6f}\PY{l+s+s1}{\PYZsq{}} \PY{o}{\PYZpc{}} \PY{p}{(}\PY{n}{x}\PY{p}{,} \PY{n}{y}\PY{p}{)}
\end{Verbatim}

            \begin{Verbatim}[commandchars=\\\{\}]
{\color{outcolor}Out[{\color{outcolor}10}]:} 'x = 0.312712, y = 0.329008'
\end{Verbatim}

    因此 \[ x \approx 0.312712, \quad y \approx 0.329008. \]
    和 Wiki \footnote{\texttt{https://en.wikipedia.org/wiki/Illuminant\_D65\#Definition}} 上查到的似乎差不多。

  \item {\bf We would like to superpose two monochromatic (meaning single wavelength) sources in order 
      to obtain ideal white as a result. One of the source’s beam corresponds to a wavelength 480 nm. 
      Using the excel sheet as attached, what is the wavelength of the other source?} \\

    一樣先讀入資料,頡取有用的部分。
    \begin{Verbatim}[commandchars=\\\{\}]
{\color{incolor}In [{\color{incolor}1}]:} \PY{k+kn}{from} \PY{n+nn}{pyexcel\PYZus{}xls} \PY{k}{import} \PY{n}{get\PYZus{}data}
        \PY{k+kn}{from} \PY{n+nn}{math} \PY{k}{import} \PY{n}{sqrt}
        \PY{n}{data} \PY{o}{=} \PY{n}{get\PYZus{}data}\PY{p}{(}\PY{l+s+s1}{\PYZsq{}}\PY{l+s+s1}{data.xls}\PY{l+s+s1}{\PYZsq{}}\PY{p}{)}\PY{p}{[}\PY{l+s+s1}{\PYZsq{}}\PY{l+s+s1}{Sheet1}\PY{l+s+s1}{\PYZsq{}}\PY{p}{]}
        \PY{n}{rdata} \PY{o}{=} \PY{n}{data}\PY{p}{[}\PY{l+m+mi}{5}\PY{p}{:}\PY{p}{]}
        \PY{n}{wlmap} \PY{o}{=} \PY{p}{\PYZob{}}\PY{n+nb}{int}\PY{p}{(}\PY{n}{r}\PY{p}{[}\PY{l+m+mi}{0}\PY{p}{]}\PY{p}{)}\PY{p}{:} \PY{p}{(}\PY{o}{*}\PY{n}{r}\PY{p}{[}\PY{l+m+mi}{1}\PY{p}{:}\PY{l+m+mi}{4}\PY{p}{]}\PY{p}{,}\PY{p}{)} \PY{k}{for} \PY{n}{r} \PY{o+ow}{in} \PY{n}{rdata}\PY{p}{\PYZcb{}}
\end{Verbatim}

    之後把所有波長的 $x, y$ value 對照表建出。
    \begin{Verbatim}[commandchars=\\\{\}]
{\color{incolor}In [{\color{incolor}2}]:} \PY{n}{xymap} \PY{o}{=} \PY{p}{\PYZob{}}\PY{n}{x}\PY{p}{:} \PY{p}{(}\PY{n}{y}\PY{p}{[}\PY{l+m+mi}{0}\PY{p}{]}\PY{o}{/}\PY{n+nb}{sum}\PY{p}{(}\PY{n}{y}\PY{p}{)}\PY{p}{,} \PY{n}{y}\PY{p}{[}\PY{l+m+mi}{1}\PY{p}{]}\PY{o}{/}\PY{n+nb}{sum}\PY{p}{(}\PY{n}{y}\PY{p}{)}\PY{p}{)} \PY{k}{for} \PY{n}{x}\PY{p}{,} \PY{n}{y} \PY{o+ow}{in} \PY{n}{wlmap}\PY{o}{.}\PY{n}{items}\PY{p}{(}\PY{p}{)}\PY{p}{\PYZcb{}}
\end{Verbatim}

\newcommand{\ww}{\mathbf{w}}
\newcommand{\vv}{\mathbf{v}}
  在 CIE-1931 中,Ideal white 對應到的 $x, y$ 作標為 $\ww = (1/3, 1/3)$。假設兩個光源在 $x, y$ 作標
  上分別在點 $\vv_{480}$ 和點 $\vv_{\lambda}$,如果他們可以混出白色的光,則 $\vv_{480}, \ww, \vv_\lambda$ 共線,
  而且 $\ww$ 點還要在中間,此條件等價於 $\theta = \measuredangle \vv_{480} \ww\, \vv_\lambda = 180^\circ$,也就是
  \[ \frac{(\vv_{480} - \ww) \cdot (\vv_\lambda - \ww)}{ \norm{\vv_{480} - \ww} \, \norm{\vv_\lambda - \ww} }= -1 \]

  當然實際上不可能如此精準,因此我們找使上式最靠近 $-1$ 的波長 $\lambda$。
    \begin{Verbatim}[commandchars=\\\{\}]
{\color{incolor}In [{\color{incolor}3}]:} \PY{n}{W} \PY{o}{=} \PY{p}{(}\PY{l+m+mi}{1}\PY{o}{/}\PY{l+m+mi}{3}\PY{p}{,} \PY{l+m+mi}{1}\PY{o}{/}\PY{l+m+mi}{3}\PY{p}{)}
        \PY{n}{v480} \PY{o}{=} \PY{n}{xymap}\PY{p}{[}\PY{l+m+mi}{480}\PY{p}{]}
\end{Verbatim}

    \begin{Verbatim}[commandchars=\\\{\}]
{\color{incolor}In [{\color{incolor}4}]:} \PY{k}{def} \PY{n+nf}{angle}\PY{p}{(}\PY{n}{wl}\PY{p}{)}\PY{p}{:}
            \PY{n}{v1} \PY{o}{=} \PY{n+nb}{tuple}\PY{p}{(}\PY{n}{a}\PY{o}{\PYZhy{}}\PY{n}{b} \PY{k}{for} \PY{n}{a}\PY{p}{,} \PY{n}{b} \PY{o+ow}{in} \PY{n+nb}{zip}\PY{p}{(}\PY{n}{v480}\PY{p}{,} \PY{n}{W}\PY{p}{)}\PY{p}{)}
            \PY{n}{s1} \PY{o}{=} \PY{n}{sqrt}\PY{p}{(}\PY{n}{v1}\PY{p}{[}\PY{l+m+mi}{0}\PY{p}{]}\PY{o}{*}\PY{o}{*}\PY{l+m+mi}{2} \PY{o}{+} \PY{n}{v1}\PY{p}{[}\PY{l+m+mi}{1}\PY{p}{]}\PY{o}{*}\PY{o}{*}\PY{l+m+mi}{2}\PY{p}{)}
            \PY{n}{v2} \PY{o}{=} \PY{n+nb}{tuple}\PY{p}{(}\PY{n}{a}\PY{o}{\PYZhy{}}\PY{n}{b} \PY{k}{for} \PY{n}{a}\PY{p}{,} \PY{n}{b} \PY{o+ow}{in} \PY{n+nb}{zip}\PY{p}{(}\PY{n}{xymap}\PY{p}{[}\PY{n}{wl}\PY{p}{]}\PY{p}{,} \PY{n}{W}\PY{p}{)}\PY{p}{)}
            \PY{n}{s2} \PY{o}{=} \PY{n}{sqrt}\PY{p}{(}\PY{n}{v2}\PY{p}{[}\PY{l+m+mi}{0}\PY{p}{]}\PY{o}{*}\PY{o}{*}\PY{l+m+mi}{2} \PY{o}{+} \PY{n}{v2}\PY{p}{[}\PY{l+m+mi}{1}\PY{p}{]}\PY{o}{*}\PY{o}{*}\PY{l+m+mi}{2}\PY{p}{)}
            \PY{n}{dot} \PY{o}{=} \PY{n}{v1}\PY{p}{[}\PY{l+m+mi}{0}\PY{p}{]}\PY{o}{*}\PY{n}{v2}\PY{p}{[}\PY{l+m+mi}{0}\PY{p}{]} \PY{o}{+} \PY{n}{v1}\PY{p}{[}\PY{l+m+mi}{1}\PY{p}{]}\PY{o}{*}\PY{n}{v2}\PY{p}{[}\PY{l+m+mi}{1}\PY{p}{]}
            \PY{k}{return} \PY{n}{dot} \PY{o}{/} \PY{n}{s1} \PY{o}{/} \PY{n}{s2}
\end{Verbatim}

    \begin{Verbatim}[commandchars=\\\{\}]
{\color{incolor}In [{\color{incolor}5}]:} \PY{n}{bs}\PY{p}{,} \PY{n}{bwl} \PY{o}{=} \PY{l+m+mf}{1.0}\PY{p}{,} \PY{l+m+mi}{480}
\end{Verbatim}

    \begin{Verbatim}[commandchars=\\\{\}]
{\color{incolor}In [{\color{incolor}6}]:} \PY{k}{for} \PY{n}{f} \PY{o+ow}{in} \PY{n}{wlmap}\PY{p}{:}
            \PY{n}{t} \PY{o}{=} \PY{n}{angle}\PY{p}{(}\PY{n}{f}\PY{p}{)}
            \PY{k}{if} \PY{n}{t} \PY{o}{\PYZlt{}} \PY{n}{bs}\PY{p}{:}
                \PY{n}{bs}\PY{p}{,} \PY{n}{bwl} \PY{o}{=} \PY{n}{t}\PY{p}{,} \PY{n}{f}
\end{Verbatim}

    \begin{Verbatim}[commandchars=\\\{\}]
{\color{incolor}In [{\color{incolor}7}]:} \PY{l+s+s1}{\PYZsq{}}\PY{l+s+s1}{best wave length = }\PY{l+s+si}{\PYZpc{}d}\PY{l+s+s1}{, cos(theta) = }\PY{l+s+si}{\PYZpc{}.6f}\PY{l+s+s1}{\PYZsq{}} \PY{o}{\PYZpc{}} \PY{p}{(}\PY{n}{bwl}\PY{p}{,} \PY{n}{bs}\PY{p}{)}
\end{Verbatim}

            \begin{Verbatim}[commandchars=\\\{\}]
{\color{outcolor}Out[{\color{outcolor}7}]:} 'best wave length = 580, cos(theta) = -0.999879'
\end{Verbatim}

最後我們得到最佳的波長為 $\lambda = \SI{580}\nm$,此時 $\cos(\theta) \approx -0.999879$ ,算是非常接近的了!

\end{enumerate}

\end{document}

