\documentclass[12pt, a4paper]{article}

\usepackage[hmargin=2.5cm, vmargin=2cm]{geometry}
\usepackage{amsthm, amssymb, mathtools, yhmath, graphicx}
\usepackage{fontspec, type1cm, titlesec, titling, fancyhdr, tabularx}
\usepackage{color, float, hhline}

\usepackage[abbreviations, per-mode=symbol]{siunitx}
%\usepackage{luatexja-fontspec}
\usepackage[CheckSingle, CJKmath]{xeCJK}
\usepackage{CJKulem}
\usepackage{enumitem}
\usepackage{tikz}
\usepackage{csquotes}
\usepackage{algorithm2e}
\usepackage{circuitikz}
\usepackage{parskip}
\usepackage{bm}
%\setCJKmainfont[BoldFont=cwTex Q Hei]{cwTex Q Ming}
%\setCJKsansfont[BoldFont=cwTex Q Hei]{cwTex Q Ming}
%\setCJKmonofont[BoldFont=cwTex Q Hei]{cwTex Q Ming}
\setCJKmainfont[BoldFont=cwTeX Q Hei]{cwTeX Q Ming}

\def\normalsize{\fontsize{12}{18}\selectfont}
\def\large{\fontsize{14}{21}\selectfont}
\def\Large{\fontsize{16}{24}\selectfont}
\def\LARGE{\fontsize{18}{27}\selectfont}
\def\huge{\fontsize{20}{30}\selectfont}

%\titleformat{\section}{\bf\Large}{\arabic{section}}{24pt}{}
%\titleformat{\subsection}{\large}{\arabic{subsection}.}{12pt}{}
%\titlespacing*{\subsection}{0pt}{0pt}{1.5ex}

\parindent=0pt

\DeclarePairedDelimiter{\abs}{\lvert}{\rvert}
\DeclarePairedDelimiter{\norm}{\lVert}{\rVert}
\DeclarePairedDelimiter{\inpd}{\langle}{\rangle}
\DeclarePairedDelimiter{\ceil}{\lceil}{\rceil}
\DeclarePairedDelimiter{\floor}{\lfloor}{\rfloor}

\DeclareMathOperator*{\argmax}{arg\,max}

\newtheorem{lemma}{Lemma}
\newcommand{\img}{\mathsf{i}}
\newcommand{\ex}{\mathsf{e}}
\newcommand{\dD}{\mathrm{d}}
\newcommand{\dI}{\,\mathrm{d}}
\newcommand{\defeq}{\triangleq}
\DeclareMathOperator*{\ord}{\mathcal{O}}
\setlength{\droptitle}{-1.5cm} %title 與上緣的間距
\title{Algorithm HW\#2}
\author{B02901178 江誠敏}

\begin{document}
\maketitle
\section{Problem 1.}

First we define some notation. $\mathcal{N} = \{1, 2, \dots, n\}$ would be the elements
and $\mathcal{S} = \{S_1, \dots, S_m\}$ would be the sets in the set cover problem,
where each $S_i \subseteq \mathcal{N}$. Let $w_i = w(S_i)$ be the cost to choose $S_i$.
The corresponding relaxed LP is:

\begin{equation} \label{eq:set-cover-lp}
  \begin{array}{>{\displaystyle}l>{\displaystyle}r>{\displaystyle}l}
    \text{minimize} & \multicolumn{1}{l}{\displaystyle \sum_{S_i \in \mathcal{S}} w_i x_i} \\[10pt]
    \text{s.t.} & \sum_{j \in S_i} x_i \geq 1, & \ \forall j \in \mathcal{N} \\
  \end{array}
\end{equation}

The dual LP is

\begin{equation} \label{eq:set-cover-dual-lp}
  \begin{array}{>{\displaystyle}l>{\displaystyle}r>{\displaystyle}l}
    \text{maximize} & \multicolumn{1}{l}{\displaystyle \sum_{j \in \mathcal{N}} \alpha_j} \\[10pt]
    \text{s.t.} & \sum_{j \in S_i} \alpha_j \leq w_i, & \ \forall S_i \in \mathcal{S} \\
  \end{array}
\end{equation}

The greedy algorithm mentioned in class could be extend to give a dual feasible solution.

\RestyleAlgo{boxruled}
\LinesNumbered
\begin{algorithm}[H]
  \DontPrintSemicolon
  $C \gets \mathcal{N}$ \quad \tcp*[l]{Elements that has not been covered}
  \While{$C \neq \varnothing$}{
    Find a set $S$ with $S \cap C \neq \varnothing$ and minimize $t = w(S) / \abs{S \cap C}$. \;
    Set $\alpha_j \gets t / H_k$, for each $j \in S \cap C$. \tcc*[r]{\small $j$ is covered in this round}
    Choose $S$. \tcc*[r]{The cost is $w(S) = t \abs{S \cap C} = \sum_{j \in S \cap C} H_k \alpha_j$}
    $C \gets C \setminus S$. \;
  }
  \caption{The greedy algorithm taught in class}
\end{algorithm} \medskip

If this algorithm indeed gives a feasible dual solution $\bm{\alpha} = (\alpha_i)$,
then it is easy to see that the algorithm would give a solution of the set cover problem
with cost equal to $H_k \sum \alpha_i$, and
\[ H_k \sum \alpha_j \leq H_k \,\mathrm{OPT}_\mathrm{LP,dual} = H_k \,\mathrm{OPT}_\mathrm{LP,primal}
  \leq H_k \,\mathrm{OPT}_\mathrm{ILP,primal} \]
Thus it is an $H_k$-approximation solution.

Now, for any set $S = S_i$, we shall prove that $\sum_{j \in S} \alpha_j \leq w(S)$.
Assume that $S = \{e_1, e_2, \dots, e_h\}, \, h \leq k$, and let $t_i$ be the time (round)
which $e_i$ is covered. Reorder the element so we could assume that $t_1 \leq t_2 \leq \dots \leq t_h$.
Just before $e_i$ is choosen, if we choose $S$ to cover $e_i$, the cost is $w(S) / (h-i+1)$,
thus we must have $\alpha_{e_i} \leq w(S)/((h-i+1)H_k)$.
Hence
\[ \sum \alpha_{e_i} \leq w(S) H_k \left( \frac{1}{1} + \frac{1}{2} + \dots + \frac{1}{h} \right)
  \leq w(S) H_h / H_k \leq w(S) \]
and the prove is completed.

\section{Problem 2.}

Similar to problem 1., the corresponding relaxed LP is now:

\begin{equation} \label{eq:set-cover2-lp}
  \begin{array}{>{\displaystyle}l>{\displaystyle}r>{\displaystyle}l}
    \text{minimize} & \multicolumn{1}{l}{\displaystyle \sum_{S_i \in \mathcal{S}} w_i x_i} \\[10pt]
    \text{s.t.} & \sum_{j \in S_i} x_i \geq k_j, & \ \forall j \in \mathcal{N} \\
  \end{array}
\end{equation}

The dual LP is

\begin{equation} \label{eq:set-cover2-dual-lp}
  \begin{array}{>{\displaystyle}l>{\displaystyle}r>{\displaystyle}l}
    \text{maximize} & \multicolumn{1}{l}{\displaystyle \sum_{j \in \mathcal{N}} k_j \alpha_j} \\[10pt]
    \text{s.t.} & \sum_{j \in S_i} \alpha_j \leq w_i, & \ \forall S_i \in \mathcal{S} \\
  \end{array}
\end{equation}

We make a small modification to the algorithm.

\begin{algorithm}[H] \label{algo:modified-greedy}
  \DontPrintSemicolon
  $C \gets \mathcal{N}$ \quad \tcp*[l]{Elements that has not been covered}
  \lForEach(\quad \tcp*[h]{\small $y_j$ is the number of time $j$ is covered})
  {$j \in \mathcal{N}$}{$\alpha_j \gets 0$, $y_j \gets 0$}
  \While{$C \neq \varnothing$}{
    Find a set $S$ with $S \cap C \neq \varnothing$ and minimize $t = w(S) / \abs{S \cap C}$. \; \label{ln:define-t}
    Define $D \defeq S \cap C$. \;
    \ForEach(\tcc*[f]{\small We say $j$ is \textbf{covered once} in this round}){$j \in D$}{
      $\alpha_j \gets \alpha_j + t / (H_n k_j)$ \;
      $y_j \gets y_j + 1$ \;
      \If(\tcc*[f]{\small We say $j$ is \textbf{fully covered} in this round}){$y_j = k_j$} {
        $C \gets C \setminus \{ j \}$. \;
      }
    }
    Choose $S$.
  }
  \caption{The modified greedy algorithm}
\end{algorithm} \medskip

In each round The cost of choosing $S$ is
\[ w(S) = w(S) \abs{D} / \abs{D} = H_n \sum_{j \in D} k_j t / (H_n k_j) = H_n \sum_{j \in D} k_j \Delta \alpha_j, \]
where $\Delta \alpha_j$ is the amount of change of $\alpha_j$ in that round.
Thus the algorithm give a solution of the set cover problem with cost $H_n \sum k_j \alpha_j$.

So again, if we could prove that $\bm{\alpha} = (\alpha_i)$ is a feasible solution to the dual problem
\eqref{eq:set-cover2-dual-lp}, then
\[ H_n \sum \alpha_j \leq H_n \,\mathrm{OPT}_\mathrm{LP,dual} = H_n \,\mathrm{OPT}_\mathrm{LP,primal}
  \leq H_n \,\mathrm{OPT}_\mathrm{ILP,primal}, \]
and the solution would be an $H_n$-approximation solution.

Similarly, for any set $S = S_i$,
Assume that $S = \{e_1, e_2, \dots, e_h\}$, and let $t_i$ be the time (round)
which $e_i$ is {\bf fully covered}. Reorder the element so we could assume that $t_1 \leq t_2 \leq \dots \leq t_h$.
Notice that each element $e_i$ would be {\bf covered once} exactly $k_{e_i}$ times,
and these all happen before $e_i$ is fully covered.

Consider once in this $k_{e_i}$ times when $e_i$ is covered once.
Let $j \defeq e_i$. The change of $\alpha_{e_i}$ is
$\Delta \alpha_{e_i} = t / H_n k_{e_i}$ where $t$ is defined in line~\ref{ln:define-t} of algorithm~\ref{algo:modified-greedy}.
If we choose $S$, then $w(S) / \abs{S \cap C} \leq w(S) / (h - i + 1)$. Thus by the minimality
of $t$, we must have $t \leq w(S) / (h-i+1)$.
Hence
\[ \alpha_{e_i} = \sum \Delta \alpha_{e_i} \leq k_{e_i} \cdot \frac{w(S)}{(h-i+1) H_n k_{e_i}}
  = \frac{w(S)}{h-i+1}H_n^{-1} \]
So
\[ \sum \alpha_{e_i} \leq \sum_i \frac{w(S)}{h-i+1} H_n^{-1} = w(S) \frac{H_h}{H_n} \leq w(S) \]
And we conclude that $\bm{\alpha}$ is indeed a feasible solution.

\section{Problem 3.}
The relaxed LP correspond to the vertex cover is:
\begin{equation} \label{eq:vertex-cover-lp}
  \begin{array}{>{\displaystyle}l>{\displaystyle}r>{\displaystyle}l}
    \text{minimize} & \multicolumn{1}{l}{\displaystyle \sum_{v \in V} w_v x_v} \\[10pt]
    \text{s.t.} & x_u + x_v \geq 1, & \ \forall (u, v) \in E \\
  \end{array}
\end{equation}

Let $E(v)$ be all the edges adjacent to $v$, the dual LP is

\begin{equation} \label{eq:set-cover2-dual-lp}
  \begin{array}{>{\displaystyle}l>{\displaystyle}r>{\displaystyle}l}
    \text{maximize} & \multicolumn{1}{l}{\ \,\displaystyle \sum_{e \in E} \alpha_e} \\[10pt]
    \text{s.t.} & \sum_{e \in E(v)} \alpha_e \leq w_v, & \ \forall v \in V \\
  \end{array}
\end{equation}

Investigating the complementary slackness, we have
\[ \sum_e \alpha_e \leq \sum_{e} \footnote{$(u, v) = e$ here} \alpha_e (x_u + x_v)
  = \sum_v x_v \sum_{e \in E(v)} \alpha_e \leq \sum_v w_v x_v \]

Clearly we only need to consider the solution which every $x_v \leq 1$, so we can
assume $x_v + x_u \leq 2, \, \forall (u, v) \in E$. Then if either $x_v = 0$
or $\sum_{e \in E(v)} \alpha_e = w_v$ for each $v \in V$, we would have

\[ \sum_v w_v x_v = \sum_v x_v \sum_{e \in E(v)} \alpha_e = \sum_e \alpha_e (x_u + x_v)
  \leq 2 \sum_e \alpha_e \]

Thus if these an algorithm satisfy
\begin{enumerate}[itemsep=-4pt, topsep=-2pt]
  \item It gives a valid vertex cover $S$.
  \item It induces a feasible dual solution $\bm{\alpha} = (\alpha_e)$.
  \item The complementary slackness relation is satisfied. That is,
    for each vertex $v$, either $v$ is not picked into $S$ (i.e., $x_v = 0$),
    or $\sum_{e \in N(v)} \alpha_e = w_v$.
\end{enumerate}
Then this algorithm is a $2$-approximation algorithm. So we check these 3 conditions for
both algorithms.

\subsection{Algorithm 1}
\begin{enumerate}
  \item {\bf It gives a valid vertex cover}: Since eventually each edge is removed,
    and we only remove covered edge, all edges are covered when the algorithm ends.
  \item {\bf It induces a feasible dual solution}: Let $\alpha_e = 0$ at the beginning.
    Define the slackness $\tilde{r}_v$ of a vertex $v$ as $\tilde{r}_v \defeq w_v - \sum_{e \in N(v)} \alpha_e$,
    Then $\tilde{r}_v = w_v = r_v$ at the very start. Now, when reducing the residual
    weight $r_v$ by $c \cdot \deg(v)$, we also increase $\alpha_e$ by $c$ for all edges $e$. Then
    it is easy to see that $\tilde{r}_v$ also decrease by $c \cdot \deg(v)$. Thus $r_v = \tilde{r}_v$
    in any moment. Since $c \cdot \deg(v) = \min r_v / \deg(v) \cdot \deg(v) = \min r_v$ in each round,
    we never decrease $r_v$ below zero, thus $\tilde{r}_v \geq 0$ at the end. Hence
    $\sum_{e \in N(v)} \alpha_e \leq w_v$ for each $v$ when the algorithm ends, and $\bm{\alpha}$
    is then a feasible dual solution.
  \item {\bf The complementary slackness relation is satisfied}: It is clear from the statement
    since we add $v$ into $S$ only when $r_v = \tilde{r}_v = 0$ (i.e., $w_v = \sum_{e \in N(v)} \alpha_e)$.
\end{enumerate}

\subsection{Algorithm 2}
\begin{enumerate}
  \item {\bf It gives a valid vertex cover}: It is clear since the algorithm stops
    only when each edge is covered.
  \item {\bf It induces a feasible dual solution}: Similar as in algorithm 1, let $\alpha_e = 0$ at the beginning.
    When reducing the residual weight $r_u$ and $r_v$ by $t = \min(r_u, r_v)$,
    also increase $\alpha_e$ by $t$.
    Then it is easy to see that $\tilde{r}_u, \tilde{r}_v$ also decrease by $t$, thus $\tilde{r}_v$ matches $r_v$.
    Also $t \leq r_u$ and $t \leq r_v$, so we never decrease $r_v$ below zero. Hence
    $\sum_{e \in N(v)} \alpha_e \leq w_v$ and we know that $\bm\alpha$ is a feasible solution.
  \item {\bf The complementary slackness relation is satisfied}: It is clear from the statement
    since we add $v$ into $S$ only when it has zero residual weight $r_v$.

\end{enumerate}

{\bf Collaborators}: None.

\section{Problem 4.}
The relaxed LP correspond to the metric uncapacitated facility locating problem with penalty is:
\begin{equation} \label{eq:facil-lp}
  \begin{array}{>{\displaystyle}l>{\displaystyle}r>{\displaystyle}c*2{>{\displaystyle}l}}
    \text{minimize} & \multicolumn{4}{l}{\displaystyle \sum c_{i,j} x_{i,j} + \sum f_i y_i + \sum p_j z_j} \\[10pt]
    \text{s.t.} & z_j + \sum_i x_{i, j} & \geq & 1, & \forall j \\
    & y_i - x_{i, j} & \geq & 0, & \forall i, j \\[6pt]
    & y_i,\, x_{i, j},\, z_j & \geq & 0, & \forall i, j
  \end{array}
\end{equation}

The dual LP is:
\begin{equation} \label{eq:facil-dual-lp}
  \begin{array}{>{\displaystyle}l>{\displaystyle}r>{\displaystyle}c*2{>{\displaystyle}l}}
    \text{maximize} & \multicolumn{4}{l}{\displaystyle \sum \alpha_j} \\[10pt]
    \text{s.t.} & \alpha_j - \beta_{i, j} & \leq & c_{i, j}, & \forall i, j \\[8pt]
    & \sum_j \beta_{i,j} & \leq & f_i, & \forall i \\
    & \alpha_j & \leq & p_j, & \forall j \\[8pt]
    & \alpha_j, \beta_{i, j} & \geq & 0, & \forall i, j
  \end{array}
\end{equation}

Thus we modify the algorithm taught in class as following, where some definition is listed below:
\begin{itemize}
  \item If we choose to give up client $j$, we say $j$ is {\bf covered} but {\bf unconnected}.
  \item If a client $j$ is {\bf connected}, it is simultaneously {\bf covered}.
  \item If $\beta_{i, j} > 0$, then we say client $j$ {\bf pays} for facility $i$.
  \item If there is a client $j$ such that $j$ {\bf pays} for both $i, i'$, then we say
    $i, i'$ are {\bf conflicted} with each other.
  \item At the end of the algorithm, we choose the facilities that is {\bf truely opened},
    and connect each {\bf connected} client $j$ to a facility $i$ such that $(i, j)$ is {\bf truely connected}.
\end{itemize}

\begin{algorithm}[H] \label{algo:facil-algo}
  \DontPrintSemicolon
  \tcp{Phase 1}
  Initially set $\alpha_j \gets 0, \beta_{i, j} \gets 0$. \;
  \While{There is still a client not covered}{
    \Repeat{An event $e$ occurs}{ Increase all $a_j$ such that $j$ is not covered yet, until next event. \;  }
    \Switch{the event $e$}{
      \Case{$\alpha_j = c_{i, j}$ for some $i, j$}{
        Call the $(i, j)$ pair \underline{tight}. \;
        Whenever $\alpha_j$ changes in the future, set $\beta_{i, j} \gets \alpha_j - c_{i, j}$. \;
        \lIf{$i$ is already \underline{temporarily opened}}{\underline{Connect} $j$ to $i$}
      }
      \Case{$\alpha_j = p_j$ for some client $j$}{ \label{ln:alpha-equal-p}
        Choose not to connect $j$ with penalty $p_j$. \;
        Call $j$ \underline{covered} but \underline{unconnected}. \;
      }
      \Case{$\sum_j \beta_{i, j} = f_i$ for some facility $i$}{
        \underline{Temporarily open} facility $i$.\;
        \ForEach{\underline{uncovered} or \underline{unconnected} facility $i$ with $(i, j)$ being \underline{tight}}{
          \underline{Connect} $j$ to $i$ \label{ln:open-connect}
        }
      }
    }
  }
  \tcp{Phase 2}
  $I \gets$ any maximal set of \underline{temporarily opened} facilities without conflicted. \;
  \lForEach{facility $i \in I$}{\underline{Truely open} $i$.}
  \ForEach{connected client $j$}{
    \lIf{there exists $i \in I$ such that $(i, j)$ is \underline{tight}}
    {\underline{truely directly connect} $j$ to $i$. \label{ln:tight-connect}}
    \Else{
      Let $i$ be the facility such that $j$ is connected to, then $i$ is conflicted with some $i' \in I$
      since $I$ is maximal. \;
      \underline{Truely indirectly connect} $j$ to $i'$.
    }
  }
  \caption{The modified greedy algorithm}
\end{algorithm}

Now we analyze the cost. The cost could be decomposed into three parts:
\[ X \defeq \sum c_{i, j} x_{i, j}, \quad Y \defeq \sum f_i y_i, \quad Z \defeq \sum p_j z_j \]
where $x_{i, j} = 1$ if $(i, j)$ truely connected, $y_i = 1$ if facility $i$ is {\bf truely opened}
and $z_j = 1$ if client $j$ is uncovered (those we give up with penalties), and these value equals
$0$ otherwise.

Let $I$ be the opened facilities, $U$ be all the uncovered clients, $C_d$ be
the clients {connected directly}, $C_i$ be the clients {connected indirectly}.
Define $X_d = \sum\limits_{j \in C_d} c_{i, j} x_{i, j}, X_i = \sum\limits_{j \in C_i} c_{i, j} x_{i, j}$,
then $X = X_d + X_i$.

For $Z$, by line~\ref{ln:alpha-equal-p} in algorithm~\ref{algo:facil-algo}, $\alpha_j = p_j$
for these client $j$, thus $Z = \sum_{j \in U} z_j = \sum_{j \in U} \alpha_j$.

For $Y+X_d$,
\[ Y = \sum_{i \in I} f_i = \sum_{i \in I} \beta_{i, j} = \sum_{\substack{i \in I \\ j \text{ pays for } i}} \beta_{i, j} \]
Notice that if $j$ pays for $i$, then $(i, j)$ is tight. By line~\ref{ln:open-connect}, $j$ is connected
and thus $j \not\in U$, also by line~\ref{ln:tight-connect}, $j \not\in C_i$, so
$j \in C_d$. Since there is no conflict in $I$, each client {pays} for at most one
facility in $I$, hence
\[ Y = \sum_{i \in I} f_i \leq \sum_{\substack{i \in I, \\ j \in C_d}} \beta_{i, j} \]
And we have
\[ Y + X_d = \sum_{\substack{i \in I, \\ j \in C_d}} \beta_{i, j}
  + \sum_{\substack{j \in C_d \\ i, j \text{ connected}}} c_{i, j} \ \leq \sum_{j \in C_d} \alpha_j \]
Since if $(i, j)$ is connected, $(i, j)$ tight and thus $\alpha_j = \beta_{i, j} + c_{i, j}$.

For $X_i$, let $j$ be a client in $C_d$, $i$ be the client $j$ is originally connected to and
$i' \in I$ be the client $j$ is truely indirectly connected to at the very end.
Since $i, i'$ are in conflict, there is a client $j'$ such that $j'$ pays for both $i, i'$,
so $(i, j), (i, j'), (i', j')$ are all tight.
$j'$ would not be connected after $j$ connected to $i$, or else once $i$ is opened,
$j'$ would immediately be connected to $i$. Thus $\alpha_{j'} \leq \alpha_j$,
so $c_{i', j} \leq c_{i', j'} + c_{i, j'} + c_{i, j} \leq \alpha_{j'} + \alpha_{j'} + \alpha_j
\leq 3 \alpha_j$ since these edges are tight. Hence
\[ X_i = \sum_{\substack{j \in C_i \\ i', j \text{ connected}}} c_{i', j}
  \leq 3 \sum_{j \in C_i} \alpha_j \]

Combining all above, we have
\[ X + Y + Z = X_i + X_d + Y + Z \leq \sum_{j \in U} \alpha_j + \sum_{j \in C_d} \alpha_j
  + 3 \sum_{j \in C_i} \alpha_j \leq 3 \sum_{j} \alpha_j \]

\end{document}

