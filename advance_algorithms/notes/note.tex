\documentclass[12pt, a4paper]{article}

%%%%%%%%%%%%%%%紙張大小設定%%%%%%%%%%%%%%%
% \paperwidth=65cm
% \paperheight=160cm

%%%%%%%%%%%%%%%引入Package%%%%%%%%%%%%%%%
\usepackage[top=2cm, bottom=3cm, left=2cm, right=2cm]{geometry} % 上下左右距離邊緣2cm
\usepackage{amsmath}
\usepackage{mathtools,amsthm,amssymb,wasysym} % 引入 AMS 數學環境
\usepackage{yhmath}     % math symbol
\usepackage{graphicx}   % 圖形插入用
\usepackage{fontspec}   % 加這個就可以設定字體
\usepackage{type1cm}    % 設定fontsize用
\usepackage{titlesec}   % 設定section等的字體
\usepackage{titling}    % 加強 title 功能
\usepackage{fancyhdr}   % 頁首頁尾
\usepackage{tabularx}   % 加強版 table
\usepackage[square, comma, numbers, super, sort&compress]{natbib}
% cite加強版
\usepackage[unicode=true, pdfborder={0 0 0}, bookmarksdepth=-1]{hyperref}
% ref加強版
\usepackage[usenames, dvipsnames]{color}  % 可以使用顏色
\usepackage[shortlabels]{enumitem}  % 加強版enumerate
\usepackage{xpatch}
%\usepackage{haskell}

% \usepackage{soul}       % highlight
% \usepackage{ulem}       % 字加裝飾
% \usepackage{wrapfig}     % 文繞圖
\usepackage{floatflt}    % 浮動 figure
\usepackage{float}       % 浮動環境
\usepackage{caption}    % caption 增強
\usepackage{subcaption}    % subfigures
% \usepackage{setspace}    % 控制空行
% \usepackage{mdframed}   % 可以加文字方框
\usepackage{multicol}   % 多欄
\usepackage{siunitx}      % SI unit
\usepackage{indentfirst}
% \usepackage{dsfont}     % more mathbb

%%%%%%%%%%%%%%%%%%%TikZ%%%%%%%%%%%%%%%%%%%%%%
\usepackage{tikz}
\usetikzlibrary{calc, patterns}

%%%%%%%%%%%%%%中文 Environment%%%%%%%%%%%%%%%
\usepackage[CheckSingle, CJKmath]{xeCJK}  % xelatex 中文
\usepackage{CJKulem} % 中文字裝飾
\setCJKmainfont[BoldFont=cwTeX Q Hei]{cwTeX Q Ming}
\titleformat{\subsubsection}
{\normalfont\fontsize{13}{22}\bfseries}{}{0em}{\raisebox{1pt}{$\RHD$}\ }
%\setCJKsansfont[BoldFont=WenQuanYi Zen Hei]{AR PL UMing TW MBE}
%\setCJKmonofont[BoldFont=WenQuanYi Zen Hei]{AR PL UMing TW MBE}
% 設定中文為系統上的字型,而英文不去更動,使用原TeX字型

% \XeTeXlinebreaklocale "zh"             %這兩行一定要加,中文才能自動換行
% \XeTeXlinebreakskip = 0pt plus 1pt     %這兩行一定要加,中文才能自動換行

%%%%%%%%%%%%%%%字體大小設定%%%%%%%%%%%%%%%
%\def\normalsize{\fontsize{14}{21}\selectfont}
% \def\large{\fontsize{40}{60}\selectfont}
% \def\Large{\fontsize{50}{75}\selectfont}
% \def\LARGE{\fontsize{90}{20}\selectfont}
% \def\huge{\fontsize{34}{51}\selectfont}
% \def\Huge{\fontsize{38}{57}\selectfont}

%%%%%%%%%%%%%%%Theme Input%%%%%%%%%%%%%%%%
% \input{themes/chapter/neat}
% \input{themes/env/problist}

%%%%%%%%%%%titlesec settings%%%%%%%%%%%%%%
% \titleformat{\chapter}{\bf\Huge}
% {\arabic{section}}{0em}{}
% \titleformat{\section}{\centering\Large}
% {\arabic{section}}{0em}{}
% \titleformat{\subsection}{\large}
% {\arabic{subsection}}{0em}{}
% \titleformat{\subsubsection}{\bf\normalsize}
% {\arabic{subsubsection}}{0em}{}
% \titleformat{command}[shape]{format}{label}
% {編號與標題距離}{before}[after]

%%%%%%%%%%%%variable settings%%%%%%%%%%%%%%
% \numberwithin{equation}{section}
% \setcounter{secnumdepth}{4}  %章節標號深度
% \setcounter{tocdepth}{1}  %目錄深度
% \graphicspath{{images/}}  % 搜尋圖片目錄

%%%%%%%%%%%%%%%頁面設定%%%%%%%%%%%%%%%
\newcolumntype{C}[1]{>{\centering\arraybackslash}p{#1}}
\setlength{\headheight}{15pt}  %with titling
\setlength{\droptitle}{-1.5cm} %title 與上緣的間距
%\usepackage{parskip}
\parindent=0pt %設定縮排的距離
\parskip=1ex  %設定行距
% \pagestyle{empty}  % empty: 無頁碼
% \pagestyle{fancy}  % fancy: fancyhdr

% use with fancygdr
% \lhead{\leftmark}
% \chead{}
% \rhead{}
% \lfoot{}
% \cfoot{}
% \rfoot{\thepage}
% \renewcommand{\headrulewidth}{0.4pt}
% \renewcommand{\footrulewidth}{0.4pt}

% \fancypagestyle{firststyle}
% {
% \fancyhf{}
% \fancyfoot[C]{\footnotesize Page \thepage\ of \pageref{LastPage}}
% \renewcommand{\headrule}{\rule{\textwidth}{\headrulewidth}}
% }

%%%%%%%%%%%%%%%重定義一些command%%%%%%%%%%%%%%%
% \renewcommand{\contentsname}{目錄}  %設定目錄的標題名稱
% \renewcommand{\refname}{References}  %設定參考資料的標題名稱
\renewcommand{\abstractname}{\LARGE Abstract} %設定摘要的標題名稱
\newcommand{\infig}[3]{
  \begin{figure}[H]
    \centering
    \includegraphics[#3]{#1}
    \caption{\label{fig:mres}#2}
  \end{figure}%
}

%%%%%%%%%%%%%%%特殊功能函數符號設定%%%%%%%%%%%%%%%
% \newcommand{\citet}[1]{\textsuperscript{\cite{#1}}}
\DeclarePairedDelimiter{\abs}{\lvert}{\rvert}
\DeclarePairedDelimiter{\norm}{\lVert}{\rVert}
\DeclarePairedDelimiter{\inpd}{\langle}{\rangle} % inner product
\DeclarePairedDelimiter{\ceil}{\lceil}{\rceil}
\DeclarePairedDelimiter{\floor}{\lfloor}{\rfloor}
\DeclareMathOperator{\adj}{adj}
\DeclareMathOperator{\sech}{sech}
\DeclareMathOperator{\csch}{csch}
\DeclareMathOperator{\arcsec}{arcsec}
\DeclareMathOperator{\arccot}{arccot}
\DeclareMathOperator{\arccsc}{arccsc}
\DeclareMathOperator{\arccosh}{arccosh}
\DeclareMathOperator{\arcsinh}{arcsinh}
\DeclareMathOperator{\arctanh}{arctanh}
\DeclareMathOperator{\arcsech}{arcsech}
\DeclareMathOperator{\arccsch}{arccsch}
\DeclareMathOperator{\arccoth}{arccoth}
\newcommand{\np}[1]{\\[{#1}] \indent}
\newcommand{\transpose}[1]{{#1}^\mathrm{T}}
%%%% Geometry Symbol %%%%
\newcommand{\eu}{\mathrm{e}}
\newcommand{\degree}{^\circ}
\newcommand{\Arc}[1]{\wideparen{{#1}}}
\newcommand{\Line}[1]{\overleftrightarrow{{#1}}}
\newcommand{\Ray}[1]{\overrightarrow{{#1}}}
\newcommand{\Segment}[1]{\overline{{#1}}}
\newcommand{\Intd}[1]{\,\mathrm{d}#1}

\newcommand{\ord}{\operatorname{\mathcal{O}}}
\newcommand{\defeq}{\triangleq}
\DeclareMathOperator{\mmod}{\;mod\;}
\DeclareMathOperator{\Expect}{{\rm I\kern-.3em E}}

\providecommand\given{}
% can be useful to refer to this outside \Set
\newcommand*\SetSymbol[1][]{%
  \nonscript\:#1\vert
  \allowbreak
  \nonscript\:
\mathopen{}}
\DeclarePairedDelimiterX\Set[1]\{\}{%
  \renewcommand\given{\SetSymbol[\delimsize]}
  \,#1\,
}

\setlist{itemsep=0cm}
\setlist{topsep=.5em}


%%%%%%%%%%%%%%%證明、結論、定義等等的環境%%%%%%%%%%%%%%%
%\renewcommand{\proofname}{\bf proof:} %修改Proof 標頭
\newtheoremstyle{mystyle}% 自定義Style
{6pt}{15pt}%       上下間距
{}%               內文字體
{}%               縮排
{\bf}%            標頭字體
{.}%              標頭後標點
{1em}%            內文與標頭距離
{}%               Theorem head spec (can be left empty, meaning 'normal')

% 改用粗體,預設 remark style 是斜體
\newtheorem{theorem}{Theorem}
\newtheorem{formula}{公式}
\newtheorem{condition}{條件}
\newtheorem{supposition}{假設}
\newtheorem{conclusion}{結論}
\newtheorem{lemma}{Lemma}
\newtheorem{property}{性質}

\theoremstyle{mystyle} % 定理環境Style
\newtheorem{definition}{Definition}
\newtheorem{example}{Example}

\begin{document}
\setcounter{section}{5}

%\setcounter{subsection}{-1}
%\subsection{Recap}
%\begin{env}
  %$
%\end{env}
\subsection{Hash table}

\begin{definition}
  Given each data with a key in $U = \Set{0, 1, \dots, t-1 }$.
  A {\bf Hash table} is capable of performing the following three operations:
  \begin{enumerate}
    \item \texttt{Insert} $k$: Insert a data with key $k$.
    \item \texttt{Delete} $k$: Delete a data with key $k$.
    \item \texttt{Search} $k$: Search a data with key $k$.
  \end{enumerate}
  And each of them cost $\ord(1)$ time.
\end{definition}

The simplest way is to allocate an array with size $\abs{U}$, and directly
store the data with key $k$ in the $k$-th slot.

\begin{figure}[H]
  \centering
  \begin{tikzpicture}[scale=0.9, font=\footnotesize]
    \draw (0, 0) grid (3, 1);
    \draw[shift={(0.6, 0)}] (4.0, 0) grid (7.0, 1);
    \node at (3.8, 0.5) {$\dotsm$};
    \foreach \x in {0, 1, 2}
    \draw[thick, -latex] (\x, 2.5) ++ (0.5, 0) node[above]{$\x$} -- (\x.5, 1.2);

    \draw (3.8, 2.5) node[above]{$\dotsm$};

    \draw[thick, -latex] (5.1, 2.5) node[above]{$t\!-\!3$} -- (5.1, 1.2);
    \draw[thick, -latex] (6.1, 2.5) node[above]{$t\!-\!2$} -- (6.1, 1.2);
    \draw[thick, -latex] (7.1, 2.5) node[above]{$t\!-\!1$} -- (7.1, 1.2);
  \end{tikzpicture}
  \caption{The simplest hash table}
\end{figure}

But then, the space complexity is $\ord(U)$, which is usually intractable
(e.g., if $U$ is the set of ascii string with length no more than $10$, then
$U \approx \ord(128)^{10}$).

Thus we shall pick a {\bf hash function} $h :: U \to \Set{0, 1, \dots, m-1}$
with $m$ be a reasonable size of an array, and store the data with key $k$ in the $h(k)$-th slot.

\begin{figure}[H]
  \centering
  \begin{tikzpicture}[scale=0.9, font=\footnotesize]
    \draw[shift={(0, -.5)}] (1, 0) grid (3, 1);
    \draw[shift={(0.6, -.5)}] (4.0, 0) grid (6.0, 1);
    \node at (3.8, 0) {$\dotsm$};

    \fill[black!20] (-.5, 2.6) rectangle (8.1, 3.8);
    \node at (3.8, 3.2) {\normalsize $h$};

    \foreach \x in {0, 1, 2}
      \draw (\x, 4) ++ (0.5, 0) node[above]{$\x$};

    \draw (3.8, 4) node[above]{$\dotsm$};

    \draw (5.1, 4) node[above]{$t\!-\!3$};
    \draw (6.1, 4) node[above]{$t\!-\!2$};
    \draw (7.1, 4) node[above]{$t\!-\!1$};

    \draw (1.5, 1.8) node {$1$};
    \draw (2.5, 1.8) node {$2$};
    \draw (3.8, 1.8) node {$\dotsm$};
    \draw (5.1, 1.8) node {$m\!-\!2$};
    \draw (6.1, 1.8) node {$m\!-\!1$};

    \foreach \u/\d in {0.5/2.5, 1.5/5.1, 2.5/2.5, 5.1/3.6, 6.1/4, 7.1/6.1}
      \draw[thick, -latex] (\u, 4) -- (\u, 2.9) -- (\d, 2.05);

    \foreach \x in {2.4, 5.1, 2.6, 4, 3.6, 5.1, 6.1}
      \draw[thick, -latex] (\x, 1.5) -- (\x, 0.65);

  \end{tikzpicture}
  \caption{Hash table with a reasonable hash function}
\end{figure}

But since $m < t$, $h$ could not be injective, so there would be $k_1 \neq k_2$
such that $h(k_1) = h(k_2)$. \medskip

\begin{definition}
  If $h(k_1) = h(k_2)$ but $k_1 \neq k_2$, then a {\bf collision} occurs.
\end{definition}

\subsubsection{Collision resolution by chaining}

There are several ways to resolve the collision problem.
``Collision resolution by chaining'' is one of them.

\begin{figure}[H]
  \centering
  \begin{tikzpicture}[scale=1, font=\footnotesize]
    \draw (0, 0) grid (3, 1);
    \node at (3.8, 0.5) {$\dotsm$};
    \foreach \x in {0, 1, 2}
      \fill (\x.5, 0.5) node(v\x){} circle (0.1);
    \foreach \x/\y in {0/-.5, 0/-3, 1/-.5} {
      \draw (\x, \y) ++ (0.2, 0) {} rectangle ++(0.6, -2);
      \draw (\x, \y) ++ (0.2, -1.4) {} -- ++(0.6, 0);
      \fill (\x, \y) ++ (0.5, -1.7) {} circle (0.08);
    }

    \draw[very thick, -latex] (v0) -- ++ (0, -1.4);
    \draw[very thick, -latex] (.5, -2.2) -- ++ (0, -1.2);
    \draw[very thick, -latex] (v1) -- ++ (0, -1.4);
  \end{tikzpicture}
  \caption{An intuition of the chaining method}
\end{figure}

Instead of using an array, we use an array of link lists to store the data.
For each operations (\texttt{Insert}/\texttt{Delete}/\texttt{Search})
with key $k$, we first get the link list in the $k$-th slot of the array,
and do the correspond operation on the link list.

When analysing the performance below, we shall assume that the chaining
resolution is used.
\medskip

Now we define the load factor, which plays an important role in the performance of the hash table. \smallskip
\begin{definition}
  We define the {\bf load factor} to be $\alpha \defeq n / m$, where $n$ is the number of elements
  and $m$ is the number of table slots.
\end{definition}

\begin{definition}
  A {\bf uniform hashing function} is a hash function such that
  \begin{itemize}
    \item $\Pr(h(k) = x) = 1/m$ for each $0 \leq x < m$.
    \item $h(K_1), h(K_2), \dots, h(K_q)$ are independent if $K_1, \dots, K_q$ are independent.
  \end{itemize}
\end{definition}

A uniform hashing function is an ideal hash function, which has the following property.
\smallskip

\begin{theorem} \label{thm:uniform-hashing-expected-time}
  Assume uniform hashing, then the expected time for each operation is $\ord(\alpha)$.
\end{theorem}

\subsubsection{Expected single operation worst-case time complexity}

Let $n$ be the number of datas to store, and $m$ be the size of the hash table.
If $m = \ord(n)$, by theorem~\ref{thm:uniform-hashing-expected-time},
we know that the expected average time complexity of each operation is $\ord(n/m) = \ord(1)$.

Then we ask, what is the expected \emph{worst-case} time complexity?

Consider the following analogous problem. We put $n$ balls in $n$ bins
uniformly at random and independently, what is the expected number
of balls in the most loaded bin?

The answer turns out to be $\ord( \log n / \log \log n )$. Sadly, the proof
is quite complicated, so we shall prove a weaker form. \medskip

\begin{theorem}
  The event ``The number of balls in all bins are lower then $4 \log n / \log \log n$''
  has probability at least $1 - 1/n$. (i.e., the answer is $\ord( \log n / \log \log n )$
  in high probability as $n \to \infty$.)

  \begin{proof}
    Let $k = 4 \log n / \log \log n$ and $[S]^k \defeq \Set{ A \subseteq S \,:\, \abs{A} = k }$.
    We have
    \begin{align*}
      \Pr\,\Set{ \text{Bin } i \text{ has} \geq k \text{ balls} }
      &\stackrel{\text{(a)}}{\leq} \sum_{A \in [S]^k} \Pr\,\Set{ \text{Every ball in } A \text{ is in Bin } i } \\
      &\stackrel{\text{(b)}}{=} {n \choose k} \left( \frac{1}{n} \right)^k \\
      &= \frac{n!}{k! \, (n-k)!} \, \frac{1}{n^k}  \\
      &= \frac{1}{k!} \, \frac{n!}{(n-k)! \, n^k}
      \ \stackrel{\text{(c)}}{\leq} \frac{1}{k!}
      \ \stackrel{\text{(d)}}{\leq} \frac{1}{n^2} \qedhere
    \end{align*}
    Where (a) holds by the uniform bound, and (b) is because there are ${n \choose k}$ such
    subset $A$, and each corresponded event has probability $(1/n)^k$. (c) holds by the fact
    that $n (n-1) \dotsm (n-k+1) \leq n^k$.

    To prove (d), we need the help from Stirling's approximation formula.
    \medskip

    \begin{theorem}[Stirling's approximation formula]
      \[ k! = \sqrt{2 \pi k} \left( \frac{k}{\eu} \right)^k \left( 1 + \ord\big( \frac{1}{k} \big) \right) \]
      Also, for each $k$,
      \[ k! \geq \left( \frac{k}{\eu} \right)^k \]
    \end{theorem}

    Then we have
    \begin{align*}
      \frac{1}{k!} &\stackrel{\text{(a)}}{\leq}
        \left( \frac{\eu}{k} \right)^k \quad
        \stackrel{\text{(b)}}{\leq} \left( \frac{ \log \log n }{ \log n} \right)^k \\
      &= \exp\big( \big( \log (\log \log n / \log n ) \big) \cdot k \big) \\
      &= \exp\left( \big( \log \log \log n - \log \log n \big) \cdot \frac{4 \log n}{\log \log n} \right) \\
      &\stackrel{\text{(c)}}{\leq} \frac{1}{n^2}
    \end{align*}
  \end{proof}
\end{theorem}

Where the Stirling's approximation formula is used in (a).
Since $k = 4 \log n / \log \log n$ and $\eu \leq 4$, (b) holds.
Finally, (c) holds due to the fact that $\log \log \log n / \log \log n \leq \sup_x \log(x) / x
= 1 / \eu$ so $4 (\log \log \log n - \log \log n) / \log \log n \leq 4 (1/\eu - 1) \leq -2$.

\subsubsection{Intuition of the Stirling's approximation formula}

We could approximate the integral $\int_1^n \log x \, \Intd{x}$ by $1/2 \log 1 + \log 2 + \dots + 1/2 \log n$.
See the figure below.

\begin{figure}[H]
  \centering
  \begin{tikzpicture}[rect/.style={draw=red}]
    \fill[blue!15, scale=0.5,domain=1:9,smooth,variable=\x]
      (1, 0) -- plot ({\x},{2*ln(\x)}) -- (9, 0) -- cycle;
    \draw[thick, scale=0.5,domain=1:10,smooth,variable=\x,blue] plot ({\x},{2*ln(\x)});
    \foreach \x/\y in {2/0.693, 3/1.099, 4/1.386, 7/1.950, 8/2.080}
      \draw[thick, rect] ({\x/2-.25}, 0) rectangle ({\x/2+.25}, \y);
    \draw[thick, red] (4.25, 0) rectangle (4.5, 2.197);

    \node[red] at (2.75, 0.8) {$\dotsm$};

    \draw[thick, ->] (-0.1,0) -- (5.2,0) node[right] {$x$};
    \draw[thick, ->] (0,-0.1) -- (0,4.2) node[above] {$y$};

    \draw[thick, red] (0.5, 0) rectangle (0.75, 0);

    \foreach \x/\lb in {0.5/1, 1/2, 1.5/3, 4.5/$n$}
      \draw[thick] (\x, -.1) node[below]{\lb} -- (\x, .1);

  \end{tikzpicture}
\end{figure}

Then we have \footnote{Notice that we have no guarantee on how close the approximation is here.}
\[
  \log 2 + \dotsm + \log (n\!-\!1) + \frac{1}{2} \log n \approx
  \int_{1}^{n} \log x \, \Intd{x} = n \log n - n + 1
\]

Thus
\[
  \log(n!) = \log 2 + \dotsm + \log n \approx n \log n - n + \frac{1}{2} \log n + 1,
\]
and hence
\[
  n! \approx \eu \sqrt{n} \left(\frac{n}{e}\right)^n
\]
The error in the estimation of the integral cause the final
multiplier in the formula to be $\sqrt{2 \pi}$ instead of $\eu$.


\subsection{$2$-Universal Family of Hash Functions}

\subsubsection{Examples of hash function}
First we give some examples of hash function.

\begin{example} \hfill
\begin{enumerate}
  \item $h(k) = a k \pmod{m}$, where $a \in \mathbb{N}$ and $m$ is a prime.
  \item $h(k) = \lfloor m (k a \mmod 1) \rfloor$, where $a \in \mathbb{R}$, $m \in \mathbb{N}$,
    and $(x \mmod 1) \defeq x - \floor{x}$.
\end{enumerate}
These two hash functions map the keys to $\Set{0, 1, \dots, m-1}$.
\end{example}

\smallskip

\begin{definition}
  Let $H$ be a family of hash functions $\Set{ h_1, \dots, h_k }$,
  where each hash function has the signature $h_i :: U \to \Set{0, 1, \dots, m-1}$.

  $H$ is {\bf $2$-universal} if for any two elements $x, y \in U$ and $x \neq y$, $\Pr\Set{ h(x) = h(y) } \leq 1/m$,
  where $h$ is uniformly choosed in random.

  Another equivalent definition is
  the number of $h_i$ satisfying $h_i(x) = h_i(y)$ is no more than $k / m$ for each $x \neq y$.
\end{definition}

We now give an example of a $2$-universal family of hash functions.

\begin{example}
  Given $x, y$, take the binary representations
  $x = x_0 x_1 \dotsm x_r$ and $y = y_0 y_1 \dotsm y_r$.

  Let $a = \langle a_0, \dots, a_r \rangle$ with each $a_i \in \Set{0, 1, \dots, m-1}$.
  For each $a$, define
  \[ h_a(x) = (a_0 x_0 + a_1 x_1 + \dots + a_r x_r) \mmod m. \]

  We claim that this is a universal family of hash functions with $\abs{H} = m^{r+1}$.

  \begin{proof}
    Since $x \neq y$, without lost of generality, assume $x_0 = 0$ and $y_0 = 1$.
    Then it is easy to see that $f_k \triangleq h_{\langle k, a_1, \dots, a_r \rangle}$
    satisfies $f_i(x) = f_j(x)$ and $f_i(y) \neq f_j(y)$ for any $0 \leq i, j < m$ and $i \neq j$.
    So $f_k(y)$ iterate through $0, 1, \dots, m-1$. Thus there are exactly $1$ hash function $f$
    satisfies $f(x) = f(y)$ among these $m$ hash functions. Hence $\Pr\Set{ h(x) = h(y) } \leq 1/m$.
  \end{proof}

  Notice that if the key $k$ of each data satisfies $0 \leq k < n$, then taking
  $r = \ord(\log n)$ is sufficient.
\end{example}

\subsubsection{Another randomized max-cut algorithm}
Let $G = (V, E)$ be the problem instance, where $V = \Set{v_0, \dots, v_{n-1}}$
and $H = \Set{ h_i }$ be a $2$-universal family of hash functions
with each $h_i :: \Set{0, \dots, n} \to \Set{0, 1}$.
Pick a hash function $h$ randomly from $H$ uniformly.
We give a cut $C = (S, T)$, $S \sqcup T = V$ by $S = \Set{ v_i \mid h(i) = 0}$
and $T = \Set{ v_i \mid h(i) = 1}$.

For each edge $(u, v)$ in the cut of the optimal solution $C_{\text{opt}}$,
since $u \neq v$, $\Pr\Set{ h(u) \neq h(v)} \geq 1/2$,
so $\Pr\Set{ (u, v) \in C} \geq 1/2$, thus $\Expect \abs{C} \geq \Expect \abs{C_{\text{opt}}} / 2$,
which proves this algorithm is a $2$-approximation algorithm.

Notice that we only need $\ord(\log n)$ random bits to choose $h$,
and is less then the original algorithm, which required $\ord(n)$ random bits.

\subsection{Perfect hashing}

\begin{definition}
  A {\bf perfect hash function} is a hash function with no collisions.
\end{definition}

Obviously this is not achievable in general, so we shall assume that
we have prior knowledge on the keys which would be stored in the hash table,
and thus we could find a suitable hash function before all the operations come.

\begin{example}
  We give an example of creating a perfect hash function of $n$ keys.
  The procedure is listed in the following:
  \begin{enumerate}
    \item Choose a hash function $h$ from a $2$-universal family of hash functions,
      which maps the key to the set $\Set{0, \dots, n-1}$. Collisions are
      likely to happen here. Let $m_i$ be the number of keys that is mapped to $i$.
    \item Redo step 1. until $m_1^2 + m_2^2 + \dots + m_n^2 \leq 4n$.
    \item For each $0 \leq i < n$, let $K_i \defeq \Set{ x \mid h(x) = i }$, then $\abs{K_i} = m_i$.
      \begin{enumerate}[label=(\arabic*)]
        \item Choose a hash function $h_i$ from another $2$-universal family of hash functions
          which maps $K_i$ to the set $\Set{0, \dots, m_i^2-1}$.
        \item Similarly, redo (1) until no collision happen
      \end{enumerate}
  \end{enumerate}

  The final hashing value of a key $x$ is $(h(x), h_{h(x)}(x))$, which
  could be stored in an array of size $m_1^2 + \dots + m_n^2 = \ord(n)$.

  We shall then prove that the algorithm halts in reasonable time.
  \begin{proof}
    In step 1., let $C \defeq \# \Set{ h(x) = h(y) \mid x < y}$ be the number of collisions.
    Then we know that
    \[ \Expect C \stackrel{\text{(a)}}{\leq}
      \sum_{x < y} \Pr \Set{ h(x) = h(y) }
      \stackrel{\text{(b)}}{\leq} \frac{n (n - 1)}{2} \frac{1}{n} < \frac{n-1}{2} \]
    Where (a) is due to the union bound. Since $h$ is choosen from a $2$-univeral family
    of hash functions, $\Pr \Set{ h(x) = h(y) } = 1/n$, and there are $n (n-1)/2$ pairs $(x, y)$,
    (b) holds.

    Notice that
    \[ C = \sum_{i = 1}^n \frac{m_i (m_i - 1)}{2} = \frac{1}{2} \sum m_i^2 - \frac{n}{2} \]
    Hence
    \[ \Expect \sum m_i^2 \leq 2 \left( \frac{n-1}{2} + \frac{n}{2} \right) \leq 2 n \]
    Recall the Markov's inequality:
    \begin{theorem}
      If $X$ is a random variable with $\Pr\Set{ X \geq 0 } = 1$, then $\Pr \Set{X \geq \alpha \Expect X}
      \leq 1 / \alpha$.
    \end{theorem}

    Thus by Markov's inequality, we have $\Pr\Set{ \sum m_i^2 > 4n } \leq 1 / 2$.
    Hence in average, we only need to redo twice in step 1.

    In step 3-(1), let $C$ be the number of collisions in $j$-th table,
    which has $m_j$ elements and $m_j^2$ slots.
    Then 
    \[ \Expect C \leq  \frac{m_j (m_j - 1)}{2} \frac{1}{m_j^2} < \frac{1}{2} \]
    So again by Markov's inequality, $\Pr\Set{ \text{A collision occurs} } \leq 1 / 2$.
    Thus in average, we only need to redo twice for each table.
  \end{proof}
\end{example}
\end{document}
