\documentclass[12pt, a4paper]{article}

\usepackage[hmargin=2.5cm, vmargin=2cm]{geometry}
\usepackage{amsthm, amssymb, mathtools, yhmath, graphicx}
\usepackage{fontspec, type1cm, titlesec, titling, fancyhdr, tabularx}
\usepackage{color, unicode-math, float, hhline}

\usepackage[abbreviations, per-mode=symbol]{siunitx}
\usepackage[CheckSingle, CJKmath]{xeCJK}
\usepackage{CJKulem}
\usepackage{enumitem}
\usepackage{tikz}
\usepackage{csquotes}
\usepackage{algorithm2e}
\usepackage{circuitikz}
\usepackage{parskip}
%\setCJKmainfont[BoldFont=cwTex Q Hei]{cwTex Q Ming}
%\setCJKsansfont[BoldFont=cwTex Q Hei]{cwTex Q Ming}
%\setCJKmonofont[BoldFont=cwTex Q Hei]{cwTex Q Ming}
\setCJKmainfont[BoldFont=cwTeX Q Hei]{cwTeX Q Ming}

\def\normalsize{\fontsize{12}{18}\selectfont}
\def\large{\fontsize{14}{21}\selectfont}
\def\Large{\fontsize{16}{24}\selectfont}
\def\LARGE{\fontsize{18}{27}\selectfont}
\def\huge{\fontsize{20}{30}\selectfont}

%\titleformat{\section}{\bf\Large}{\arabic{section}}{24pt}{}
%\titleformat{\subsection}{\large}{\arabic{subsection}.}{12pt}{}
%\titlespacing*{\subsection}{0pt}{0pt}{1.5ex}

\parindent=0pt

\DeclarePairedDelimiter{\abs}{\lvert}{\rvert}
\DeclarePairedDelimiter{\norm}{\lVert}{\rVert}
\DeclarePairedDelimiter{\inpd}{\langle}{\rangle}
\DeclarePairedDelimiter{\ceil}{\lceil}{\rceil}
\DeclarePairedDelimiter{\floor}{\lfloor}{\rfloor}

\DeclareMathOperator*{\argmax}{arg\,max}

\newtheorem{lemma}{Lemma}
\newcommand{\img}{\mathsf{i}}
\newcommand{\ex}{\mathsf{e}}
\newcommand{\dD}{\mathrm{d}}
\newcommand{\dI}{\,\mathrm{d}}
\newcommand{\defeq}{\triangleq}
\DeclareMathOperator*{\ord}{\mathcal{O}}
\setlength{\droptitle}{-1.5cm} %title 與上緣的間距
\title{Algorithm HW\#1}
\author{B02901178 江誠敏}

\begin{document}
\maketitle
Collaborators: B02901027 茅耀文
\section{Problem 1.}

In the following, $v_i$ and $w_i$ will represent the value and weight of the $i$-th gift respectively,
$W$ be the weight limit, and let $V \defeq \sum v_i$. Also, let $V^*$ be the total value of optimal solution.
Assume that $w_i \leq W$ for all $i$ (or else the gift is useless), so $V^* \geq \max v_i$.

First we summarize some result we learned in lecture 1 and HW 0.5. \medskip

\begin{lemma} \hfill \label{lemma:recap}
  \begin{enumerate}
    \item There is an algorithm that solves the knapsack problem in $\ord(n V^*)$, where
      $V^*$ is the total value of the optimal solution.
    \item Fixing a problem instance, if we choose $b$ to be our rounding factor,
      then the time complexity is $\ord(n V^* / b)$,
      and we get an $(1 + 2nb / V^*)$-approximation algorithm. \label{lemma:ks-approx-algo}
    \item There is an $\ord(n \log n)$ greedy 2-approximation algorithm for the knapsack problem.
      Simply sort the gifts by $v_i / w_i$, so assume $v_i / w_i \geq v_{i+1} / w_{i+1}$, then choose \\
      $\max(v_1 + \dots + v_k, v_{k+1})$, where $k$ is the smallest indices
      letting $w_1 + \dots + w_{k+1} \geq W$.
  \end{enumerate}
\end{lemma}

So if we can choose $b = \epsilon V^* / (2n)$, then we would get an $(1+\epsilon)$-approximation algorithm
which runs in $\ord(n^2 \epsilon^{-1})$. Let $\tilde{V}$ be the solution given by the greedy algorithm
in 3., then $V^* / 2 \leq \tilde{V} \leq V^*$. So if we let $b = \epsilon
\tilde{V} / (4n)$, it would still be an $(1+\epsilon)$-approximation algorithm, and the running time
is $\ord(n^2 \epsilon^{-1})$.

Notice that the problem could also be solved without the greedy 2-approximation algorithm.
First we need a lemma. \medskip

\begin{lemma} \label{lemma:criteria}
  Let $\bar{V}$ be the optimal solution after rounding, and let $\hat{V}$ be the corresponding
  solution in the original problem. If $\bar{V} \leq (1 + \epsilon) \hat{V}$ then $\hat{V}$
  is an $(1+\epsilon)$ approximation algorithm.

  \begin{proof}
    This is because $V^* \leq \bar{V} \leq (1+\epsilon) \hat{V}$, so $\hat{V}$ is
    an $(1+\epsilon)$ approximation algorithm.
  \end{proof}
\end{lemma}

So we propose an algorithm as following:

\RestyleAlgo{boxruled}
\LinesNumbered
\begin{algorithm}[H]
  \DontPrintSemicolon
  $V \defeq \sum v_i$ \;
  $b \gets \epsilon V / (2n)$ \;
  \While{true}{
    $\hat{V} \gets$ an approximation solution using the algorithm in
    lemma~\ref{lemma:recap}-\ref{lemma:ks-approx-algo} with rounding factor $b$. \tcc*[r]{Cost $\ord(n V^*/ b)$}
    Check if $\hat{V}$ gives a solution good enough by lemma~\ref{lemma:criteria}. \tcc*[r]{Cost $\ord(n)$}
    \lIf{$\hat{V}$ is good enough}{\Return $\hat{V}$}
    \lElse{$b \gets b / 2$}
  }
  \caption{An $\ord(n^2 \epsilon^{-1})$ FPTAS for the knapsack problem}
\end{algorithm}

Notice that $V^* \leq V = \sum v_i$, so initially $b \geq \epsilon V^* / (2n)$. And once $b \leq \epsilon V^* / (2n)$,
$\hat{V}$ is then an $(1+\epsilon)$ approximation solution and the algorithm stops. At this moment,
$b \geq \epsilon V^* / (4n)$. Also, $V^* \geq \max v_i \geq V / n$, so the iteration runs at most $\log n$ times.
Hence the total running time is
\[ \ord(\log n) + 4n^2 \epsilon^{-1} + 2n^2 \epsilon^{-1} + n^2 \epsilon^{-1} + \dots = \ord(\log n) + \ord(n^2 \epsilon^{-1})
= \ord(n^2 \epsilon^{-1}) \]

\end{document}

