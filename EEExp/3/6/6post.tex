\documentclass[12pt, a4paper]{article}

\usepackage[hmargin=2.5cm, vmargin=2cm]{geometry}
\usepackage{amsthm, amssymb, mathtools, yhmath, graphicx}
\usepackage{fontspec, type1cm, titlesec, titling, fancyhdr, tabularx}
\usepackage{color}
\usepackage{unicode-math}
\usepackage{float}
\usepackage{subfig}
\usepackage{hhline}
\usepackage{comment}
\usepackage{siunitx}
\usepackage{csvsimple}
\usepackage{subcaption}

\usepackage[CheckSingle, CJKmath]{xeCJK}
\usepackage{CJKulem}
\usepackage{enumitem}
\usepackage{tikz}
\usepackage[siunitx]{circuitikz}
\usepackage{wrapfig}
%\setCJKmainfont[BoldFont=cwTex Q Hei]{cwTex Q Ming}
%\setCJKsansfont[BoldFont=cwTex Q Hei]{cwTex Q Ming}
%\setCJKmonofont[BoldFont=cwTex Q Hei]{cwTex Q Ming}
%\setCJKmainfont[BoldFont=cwTeX Q Hei]{cwTeX Q Ming}
\setmainfont{Linux Libertine O}
\setCJKmainfont[BoldFont=cwTeX Q Hei]{cwTeX Q Ming}

\def\normalsize{\fontsize{12}{18}\selectfont}
\def\large{\fontsize{14}{21}\selectfont}
\def\Large{\fontsize{16}{24}\selectfont}
\def\LARGE{\fontsize{18}{27}\selectfont}
\def\huge{\fontsize{20}{30}\selectfont}

%\titleformat{\section}{\bf\Large}{\arabic{section}}{24pt}{}
%\titleformat{\subsection}{\large}{\arabic{subsection}.}{12pt}{}
%\titlespacing*{\subsection}{0pt}{0pt}{1.5ex}

\parindent=24pt

\DeclarePairedDelimiter{\abs}{\lvert}{\rvert}
\DeclarePairedDelimiter{\norm}{\lVert}{\rVert}
\DeclarePairedDelimiter{\inpd}{\langle}{\rangle}
\DeclarePairedDelimiter{\ceil}{\lceil}{\rceil}
\DeclarePairedDelimiter{\floor}{\lfloor}{\rfloor}

\newcommand{\unit}[1]{\:(\text{#1})}
\newcommand{\df}[1]{\mathop{}\!\mathrm{d^#1}}
\newcommand{\img}{\mathrm{i}}
\newcommand{\dD}{\mathrm{d}}
\newcommand{\dI}{\,\mathrm{d}}

\title{ \bf {\Huge 電子電路實驗6: ADC/DAC}\\ 實驗結報}
\author{B02901178 江誠敏}

\begin{document}

\maketitle


\section{實驗結果}
\subsection{ADC}
\begin{center}
\begin{tabular}{p{3cm}p{3cm}}
	\hline
  Digital output & Voltage\\
	\hhline{==}
  1 & \SI{0.02}\V \\
	\hline
  2 & \SI{0.04}\V \\
	\hline
  4 & \SI{0.09}\V \\
	\hline
  8 & \SI{0.16}\V \\
	\hline
  16 & \SI{0.31}\V \\
	\hline
  32 & \SI{0.64}\V \\
	\hline
  64 & \SI{1.26}\V \\
	\hline
  128 & \SI{2.51}\V \\
	\hline
  255 & \SI{5.00}\V \\
	\hline
\end{tabular}
\end{center}

\subsection{DAC}
\begin{center}
\begin{tabular}{p{3cm}p{3cm}}
	\hline
  Digital output & Voltage\\
	\hhline{==}
  1 & \SI{0.0197}\V \\
	\hline
  2 & \SI{0.0389}\V \\
	\hline
  4 & \SI{0.0786}\V \\
	\hline
  8 & \SI{0.1561}\V \\
	\hline
  16 & \SI{0.3114}\V \\
	\hline
  32 & \SI{0.6241}\V \\
	\hline
  64 & \SI{1.2496}\V \\
	\hline
  128 & \SI{2.2498}\V \\
	\hline
  255 & \SI{4.973}\V \\
	\hline
\end{tabular}
\end{center}

\section{心得}
上次才說好像驗一個比一個簡單,沒想到這次馬上被
打臉,這次的實驗我看每個人都是做的一把鼻涕一把
眼淚,到了十二點我看都還有一半的人還沒有做完,
好險我早就知道實驗要做的快的技巧了–電路不 work
換個元件就對了,我 ADC 是整整換了 3 次才終於
拿到一個好的!
\end{document}

