\documentclass[12pt, a4paper]{article}

\usepackage[hmargin=2.5cm, vmargin=2cm]{geometry}
\usepackage{amsthm, amssymb, mathtools, yhmath, graphicx}
\usepackage{fontspec, type1cm, titlesec, titling, fancyhdr, tabularx}
\usepackage{caption}
\usepackage{color}
\usepackage{hhline}
\usepackage{unicode-math}
\usepackage{nicefrac}
\usepackage[abbreviations, per-mode=symbol]{siunitx}
\usepackage{comment}
\usepackage{float}
\usepackage{subcaption}

\usepackage[CheckSingle, CJKmath]{xeCJK}
\usepackage{CJKulem}
\usepackage{enumitem}
\usepackage[usenames, dvipsnames]{xcolor}
\usepackage{colortbl}
\usepackage{circuitikz}
%\setCJKmainfont[BoldFont=cwTex Q Hei]{cwTex Q Ming}
%\setCJKsansfont[BoldFont=cwTex Q Hei]{cwTex Q Ming}
%\setCJKmonofont[BoldFont=cwTex Q Hei]{cwTex Q Ming}
\setCJKmainfont[BoldFont=cwTeX Q Hei]{cwTeX Q Ming}

\def\normalsize{\fontsize{12}{18}\selectfont}
\def\large{\fontsize{14}{21}\selectfont}
\def\Large{\fontsize{16}{24}\selectfont}
\def\LARGE{\fontsize{18}{27}\selectfont}
\def\Huge{\fontsize{20}{30}\selectfont}

%\titleformat{\section}{\bf\Large}{\arabic{section}}{24pt}{}
%\titleformat{\subsection}{\large}{\\arabic{subsection}.}{12pt}{}
\titlespacing*{\subsection}{0pt}{0pt}{1.5ex}

\parindent=24pt

\DeclarePairedDelimiter{\abs}{\lvert}{\rvert}
\DeclarePairedDelimiter{\norm}{\lVert}{\rVert}
\DeclarePairedDelimiter{\inpd}{\langle}{\rangle}
\DeclarePairedDelimiter{\ceil}{\lceil}{\rceil}
\DeclarePairedDelimiter{\floor}{\lfloor}{\rfloor}

\newcommand{\unit}[1]{\:(\text{#1})}
\newcommand{\img}{\mathsf{i}}
\newcommand{\ex}{\mathsf{e}}
\newcommand{\dD}{\mathrm{d}}
\newcommand{\dI}{\,\mathrm{d}}
\DeclareSIUnit \uF {\micro \farad}
\DeclareSIUnit \mH {\milli \henry}

\newcommand{\tri}{$\rhd$}

\title{ \bf {\huge 電子電路實驗3,4: CMOS Operational Amplifier}\\ 實驗預報}
\author{B02901178 江誠敏}
%\date{2014/09/21}

\begin{document}

\maketitle

\section{Objectives}
\begin{enumerate}
  \item To make sure all the MOS of the circuit are able to enter saturation region
     so that the circuit can be applied to be an operational amplifier. 
  \item To comprehend the method of eliminating the Crossover Distortion of a class-B output stage. 
\end{enumerate}


\section{Procedures}
\subsection{DC Analysis}
\subsubsection{DC Analysis of CD4007}
\begin{enumerate}[itemsep=0pt]
  \item  The condition of entering saturation region of NMOS: 
    $V_{GD} \leq V_T (\SI{1.8}\V)$. 
  \item  The condition of entering saturation region of PMOS: 
    $V_{GD} \geq V_T (\SI{-1.8}\V)$. 
  \item Supply voltage $-V_{SS} = \SI{-8}\V \text{ and } V_{DD} = +\SI{8}\V$ to the circuit.
  \item  Provide voltage source $V_{CC} = +15V$, and $-V_{CC} = −15V$ to the circuit.
  \item Use digital multi-meter to confirm whether all of the MOS components 
    are able to enter saturation region as adjusting VR $\SI{1}\kohm$.
\end{enumerate}

\subsubsection{Detail procedure of DC Analysis 
of the Two-stage OP-Amp circuit }
\begin{enumerate}[itemsep=0pt]
  \item  adjust VR $R_2 = \SI{10}\kohm$ and check whether $(Q_5 , Q_7 , Q_8)$ are all 
    able to enter saturation region, that is, $V_{GD5,7,8} \geq V_T$ 
      (–1.8V). If one of them is not so, change the chip
       of CD4007 \#A and recheck again. 
  \item  Use multi-meter to measure $V_{GD5},V_{GD7},V_{GD8}$.
  \item  adjust VR $R_2 = \SI{10}\kohm$ and check whether $(Q_1, Q_2)$ are all 
    able to enter saturation region, that is, $V_{GD1,2} \geq V_T$ 
      (–1.8V). If one of them is not so, change the chip
      of CD4007 \#B and recheck again. 
  \item  Use multi-meter to measure $V_{GD1},V_{GD2}$.
  \item  In Fig.5, Use heck whether $(Q_3, Q_4)$ are all 
    able to enter saturation region, that is, $V_{GD3,4} \geq V_T$ 
      (–1.8V). If one of them is not so, change the chip
       of CD4007 \#C and recheck again. 
  \item  Use multi-meter to measure $V_{GD3},V_{GD4}$.
\end{enumerate}

\subsubsection{DC Analysis of the Two-stage OP-Amp circuit}

\begin{enumerate}[itemsep=0pt]
  \item In Fig. 6, adjust VR ($R_3$) $\SI{1}\kohm$ and use multi-meter to measure 
     $V_A$ and $V_E$ , and check whether $V_E$ is adjustable. If it not so
       , trouble shoot the circuit. 
       Check whether there is any wrong layout in your breadboard and 
       whether the VR ($R_3$) is functional by multimeter. 
  \item Record $V_A, V_E$.
\end{enumerate}

\subsubsection{Circuit implementation}

\begin{enumerate}[itemsep=0pt]
  \item In Fig. 2, adjust VR ($R_3$) $\SI{1}\kohm$ and use multi-meter to measure 
     $V_A , V_D , V_E, V_F$
\end{enumerate}

\subsection{Small-signal Analysis}
\subsubsection{Voltage Gain}
\begin{enumerate}[itemsep=0pt]
  \item Adjust VR ($R_3$) in Fig. 3 to have 
    $V_F \approx 0$.
  \item In Fig 4, apply the input small signal $V_i$ to the breadboard
    by using function generator to generate $v_i = v_{ac} \sin(2\pi f t),
    2v_{ac} = 20mV_{p-p}, f= \SI{1}\kHz$.
  \item Make sure that the $v_i$ is measured from the breadboard by using
    the probe from \texttt{CH1} in oscilloscope.
  \item Oscilloscope \tri \texttt{YT mode}. 
  \item Adjust VR ($R_3$) $\SI{1}\kohm$ in Fig. 2-2 to have maximum small-signal voltage 
     gain $V_F/V_i$ . 
  \item Keep the previous adjustment of $R_3$ constantly. 
  \item Record the voltage gain $A_M$.
  \item Record the $V_F, R_3$.
  \item Confirm whether the voltage gain is the same as the slope of teh
    curve at transition region measured in the step 6 at Exp 3.
  \item Increase/Decrease $V_i$ until the waveform of $V_F$ just distort.
  \item Record the peak-to-peak value of $V_i$.
\end{enumerate}

\subsubsection{Frequency Response}
\begin{enumerate}[itemsep=0pt]
  \item Set $v_i = v_{ac} \sin (2 \pi f t) , 2 v_{ac} = \SI{20}\mV_{p-p}, f = \SI{1}\kHz$.
    \item Function generator \tri Adjust Frequency and observe the voltage gain $A_v$ in 
      oscilloscope until $A_v = \sqrt{2} A_M$.
    \item Record the frequency $f_{3db}$.
    \item Change the frequency of input voltage source, and
      record the input and output voltage shown in oscilloscope.
\end{enumerate}

\subsubsection{Internal frequency compensation}
\begin{enumerate}[itemsep=0pt]
  \item Use $R_2 = \SI{100}\kohm, C_2 = \SI{10}\pF$ in to 
    implement the compensation circuit in Fig. 5.
  \item Record the voltage gain $A_{M,2}$ , dynamic range $V_i$
    , and frequency $f_{3dM,2}$. 
\end{enumerate}

\subsubsection{Feedback network compensation}
\begin{enumerate}[itemsep=0pt]
  \item Use $R_2 = \SI{100}\kohm, C_2 = \SI{10}\pF$ in to 
    implement the compensation circuit in Fig. 3.
  \item In Fig. 6, apply the input small signal $V_i$ to 
    the breadboard by using function generator to generate
    $v_i = v_{ac} \sin ( 2 \pi f_t) , 2 v_{ac} =
    \SI{100}\mV_{p_p} , f = \SI{1}\kHz$.
  \item Record the voltage gain $A_{M,2}$ , dynamic range $V_i$
    , and frequency $f_{3dM,2}$. 
\end{enumerate}
\end{document}


