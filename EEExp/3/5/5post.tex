\documentclass[12pt, a4paper]{article}

\usepackage[hmargin=2.5cm, vmargin=2cm]{geometry}
\usepackage{amsthm, amssymb, mathtools, yhmath, graphicx}
\usepackage{fontspec, type1cm, titlesec, titling, fancyhdr, tabularx}
\usepackage{color}
\usepackage{unicode-math}
\usepackage{float}
\usepackage{subfig}
\usepackage{hhline}
\usepackage{comment}
\usepackage{siunitx}
\usepackage{csvsimple}
\usepackage{subcaption}

\usepackage[CheckSingle, CJKmath]{xeCJK}
\usepackage{CJKulem}
\usepackage{enumitem}
\usepackage{tikz}
\usepackage[siunitx]{circuitikz}
\usepackage{wrapfig}
%\setCJKmainfont[BoldFont=cwTex Q Hei]{cwTex Q Ming}
%\setCJKsansfont[BoldFont=cwTex Q Hei]{cwTex Q Ming}
%\setCJKmonofont[BoldFont=cwTex Q Hei]{cwTex Q Ming}
%\setCJKmainfont[BoldFont=cwTeX Q Hei]{cwTeX Q Ming}
\setmainfont{Linux Libertine O}
\setCJKmainfont[BoldFont=cwTeX Q Hei]{cwTeX Q Ming}

\def\normalsize{\fontsize{12}{18}\selectfont}
\def\large{\fontsize{14}{21}\selectfont}
\def\Large{\fontsize{16}{24}\selectfont}
\def\LARGE{\fontsize{18}{27}\selectfont}
\def\huge{\fontsize{20}{30}\selectfont}

%\titleformat{\section}{\bf\Large}{\arabic{section}}{24pt}{}
%\titleformat{\subsection}{\large}{\arabic{subsection}.}{12pt}{}
%\titlespacing*{\subsection}{0pt}{0pt}{1.5ex}

\parindent=24pt

\DeclarePairedDelimiter{\abs}{\lvert}{\rvert}
\DeclarePairedDelimiter{\norm}{\lVert}{\rVert}
\DeclarePairedDelimiter{\inpd}{\langle}{\rangle}
\DeclarePairedDelimiter{\ceil}{\lceil}{\rceil}
\DeclarePairedDelimiter{\floor}{\lfloor}{\rfloor}

\newcommand{\unit}[1]{\:(\text{#1})}
\newcommand{\df}[1]{\mathop{}\!\mathrm{d^#1}}
\newcommand{\img}{\mathrm{i}}
\newcommand{\dD}{\mathrm{d}}
\newcommand{\dI}{\,\mathrm{d}}

\title{ \bf {\Huge 電子電路實驗5: Oscillators}\\ 實驗結報}
\author{B02901178 江誠敏}

\begin{document}

\maketitle


\section{實驗結果}
\subsection{Sinusoidal oscillators}
\subsubsection{Maximum amplitude}
\begin{center}
\begin{tabular}{p{3cm}p{3cm}}
	\hline
  Item & Value\\
	\hhline{==}
  $V_{o(p-p)}$ & $\SI{25.1}\V$ \\
  $f$ & $\SI{561}\Hz$ \\
  $R_{1}$ & $\SI{33}\kohm$ \\
	\hline
\end{tabular}
\end{center}

\subsubsection{Minimum amplitude}
\begin{center}
\begin{tabular}{p{3cm}p{3cm}}
	\hline
  Item & Value\\
	\hhline{==}
  $V_{o(p-p)}$ & $\SI{182}\mV$ \\
  $f$ & $\SI{556}\Hz$ \\
  $R_{1}$ & $\SI{675}\ohm$ \\
	\hline
\end{tabular}
\end{center}

\subsection{Triangular oscillators}
\begin{center}
\begin{tabular}{p{3cm}p{3cm}}
	\hline
  Item & Value\\
	\hhline{==}
  $V_{o(p-p)}$ & $\SI{15.8}\V$ \\
  $f$ & $\SI{5020}\Hz$ \\
	\hline
\end{tabular}
\end{center}

\section{心得}
不知道為什麼,總感覺電電實驗三的電路好像一個實驗比一個實驗簡單啊…
記得實驗一、二每個人都是接的一把鼻涕一把眼淚,現在做的快的好像
一個小時就做完了。希望期末考也考個簡單的吧…
\end{document}

