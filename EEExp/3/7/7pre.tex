\documentclass[12pt, a4paper]{article}

\usepackage[hmargin=2.5cm, vmargin=2cm]{geometry}
\usepackage{amsthm, amssymb, mathtools, yhmath, graphicx}
\usepackage{fontspec, type1cm, titlesec, titling, fancyhdr, tabularx}
\usepackage{caption}
\usepackage{color}
\usepackage{hhline}
\usepackage{unicode-math}
\usepackage{nicefrac}
\usepackage[abbreviations, per-mode=symbol]{siunitx}
\usepackage{comment}
\usepackage{float}
\usepackage{subcaption}

\usepackage[CheckSingle, CJKmath]{xeCJK}
\usepackage{CJKulem}
\usepackage{enumitem}
\usepackage[usenames, dvipsnames]{xcolor}
\usepackage{colortbl}
\usepackage{circuitikz}
%\setCJKmainfont[BoldFont=cwTex Q Hei]{cwTex Q Ming}
%\setCJKsansfont[BoldFont=cwTex Q Hei]{cwTex Q Ming}
%\setCJKmonofont[BoldFont=cwTex Q Hei]{cwTex Q Ming}
\setCJKmainfont[BoldFont=cwTeX Q Hei]{cwTeX Q Ming}

\def\normalsize{\fontsize{12}{18}\selectfont}
\def\large{\fontsize{14}{21}\selectfont}
\def\Large{\fontsize{16}{24}\selectfont}
\def\LARGE{\fontsize{18}{27}\selectfont}
\def\Huge{\fontsize{20}{30}\selectfont}

%\titleformat{\section}{\bf\Large}{\arabic{section}}{24pt}{}
%\titleformat{\subsection}{\large}{\\arabic{subsection}.}{12pt}{}
\titlespacing*{\subsection}{0pt}{0pt}{1.5ex}

\parindent=24pt

\DeclarePairedDelimiter{\abs}{\lvert}{\rvert}
\DeclarePairedDelimiter{\norm}{\lVert}{\rVert}
\DeclarePairedDelimiter{\inpd}{\langle}{\rangle}
\DeclarePairedDelimiter{\ceil}{\lceil}{\rceil}
\DeclarePairedDelimiter{\floor}{\lfloor}{\rfloor}

\newcommand{\unit}[1]{\:(\text{#1})}
\newcommand{\img}{\mathsf{i}}
\newcommand{\ex}{\mathsf{e}}
\newcommand{\dD}{\mathrm{d}}
\newcommand{\dI}{\,\mathrm{d}}
\DeclareSIUnit \uF {\micro \farad}
\DeclareSIUnit \mH {\milli \henry}

\newcommand{\tri}{$\rhd$}

\title{ \bf {\huge 電子電路實驗7: Vibrator}\\ 實驗預報}
\author{B02901178 江誠敏}
%\date{2014/09/21}

\begin{document}

\maketitle

\section{Objectives}
\begin{enumerate}
  \item To familiarize various kind of multivibra
    tors, including bistable, monostable and 
    unstable multivibrators. 
\end{enumerate}


\section{Procedures}
\subsection{Multivibrator using the Crystal oscillator}
\begin{enumerate}[itemsep=0pt]
  \item Supply voltage signal $V_{CC} = \SI{5}\V$ to pin 14 of 74LS00 IC and short
    pin 7 to the ground.
  \item Attach \texttt{CH1} probe of oscilloscope in the node of the output
    of second \texttt{NAND} gate.
  \item Observe whether the frequency of the measured wave is 
    sinusoidal with frequency almost same as that of 
    piezoelectric crystal (\SI{3.5}\MHz). Record the 
    frequency $f_0$.
\end{enumerate}

\subsection{Circuits of Sparkling lamp}
\begin{enumerate}[itemsep=0pt]
  \item Supply voltage $V_{CC} = \SI{12}\V$.
  \item Apply the values of $R = \SI{220}\ohm, C_1 = C_2 = \SI{47}\uF, 
    R_{B1} = R_{B2} = \SI{100}\kohm, R_{C1} = R_{C2} \approx \SI{0.5}\kohm$
  \item Oscilloscope \tri Press the \texttt{CH1} and \texttt{CH2} Menu
    \tri \texttt{Coupling} \tri \texttt{DC}.
  \item Adjust $R_{B1}$ and $R_{B2}$ to make the lamp be able
    to sparkle and make the oscillatory frequency of the lamp
    as ideally as you wish.
  \item Record the value of oscillatory frequency $f_0$.
  \item Use the \texttt{Cursors} menu button to measure
    and record the value of $T_H, T_L, \text{Duty cycle }$, 
    $R_{C1}, R_{C2}, R_{B1}, R_{B2}$.
\end{enumerate}

\subsection{An astable multivibrator using the LM555 IC}

\begin{enumerate}[itemsep=0pt]
  \item supply voltage $V_{CC} = \SI{5}\V$. Apply the values
    $R_A = R_B = \SI{10}\kohm \text{ and } C = \SI{1}\nF$.
  \item Adjust $R_A, R_B$ to fulfill the condition of $f_0 = \SI{100}\kHz$, 
    Duty cycle $= 90\%$ and $T_L = \SI{1}\us$.
  \item Attach \texttt{CH1} and \texttt{CH2} probes of oscilloscope to
    pin 3 and pin 6, respectively, and observe the measured waveform
    in \texttt{CH1} and \texttt{CH2}.
  \item Use the \texttt{Cursors} menu button to measure and record the
    value of $T_L, \texttt{ Duty cycle }$.
  \item Record $f_0, R_A, R_B$.
  \item Change the supply voltage $V_{CC} = \SI{10}\V, \SI{15}\V$ and
    repeat the process.
\end{enumerate}

\subsection{The series of 555 Circuits.}
\subsubsection{Circuit implementation}
\begin{enumerate}[itemsep=0pt]
\item Supply voltage $V_{CC} = \SI{5}\V, I_{SET} = \SI{0.5}\A$.
\item Do not connect \SI{8}{\ohm} speaker in the following steps.
\item Use $\SI{10}\kohm$ for $R_A, R_B, R_C, R_D$.
\item Attach \texttt{CH1} and \texttt{CH2} probes of 
  oscilloscope to pin 3 of $A_1$ and $A-2$ respecively.
\item Observe the measured wveform in \texttt{CH1} and \texttt{CH2}.
\end{enumerate}
\subsubsection{Frequency Adjustment}
\begin{enumerate}[itemsep=0pt]
\item Adjust $R_A$ to have $f_1 = \SI{300}\Hz \textasciitilde \SI{700}\Hz$ and 
  $R_B$ to have Duty cycle $= 50\%$.
\item Adjust $R_C$ to have $f_2 = \SI{0.1}\Hz \textasciitilde \SI{10}\Hz$ and 
  $R_D$ to have Duty cycle $= 50\%$.
\end{enumerate}
\subsubsection{Alarm Bell Adjustment}
\begin{enumerate}
\item Connect $\SI{8}\ohm$ speaker.
\item Adjust $R_A, R_C$ to have an appropriate sound.
\end{enumerate}

\subsubsection{Measurement}
\begin{enumerate}
\item Record $f_1, f_2, R_A, R_B, R_C, R_D$.
\end{enumerate}

\subsubsection{Speaker Replacement}
\begin{enumerate}
\item Replace the $\SI{8}\ohm$ speaker with LED, observe 
  whether the LED twinkles.
\end{enumerate}

\end{document}


