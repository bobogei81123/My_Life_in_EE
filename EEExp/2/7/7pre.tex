\documentclass[12pt, a4paper]{article}

\usepackage[hmargin=2.5cm, vmargin=2cm]{geometry}
\usepackage{amsthm, amssymb, mathtools, yhmath, graphicx}
\usepackage{fontspec, type1cm, titlesec, titling, fancyhdr, tabularx}
\usepackage{caption}
\usepackage{color}
\usepackage{hhline}
\usepackage{unicode-math}
\usepackage{nicefrac}
\usepackage[abbreviations, per-mode=symbol]{siunitx}
\usepackage{comment}
\usepackage{float}
\usepackage{subcaption}

\usepackage[CheckSingle, CJKmath]{xeCJK}
\usepackage{CJKulem}
\usepackage{enumitem}
\usepackage[usenames, dvipsnames]{xcolor}
\usepackage{colortbl}
\usepackage{circuitikz}
%\setCJKmainfont[BoldFont=cwTex Q Hei]{cwTex Q Ming}
%\setCJKsansfont[BoldFont=cwTex Q Hei]{cwTex Q Ming}
%\setCJKmonofont[BoldFont=cwTex Q Hei]{cwTex Q Ming}
\setCJKmainfont[BoldFont=cwTeX Q Hei]{cwTeX Q Ming}

\def\normalsize{\fontsize{12}{18}\selectfont}
\def\large{\fontsize{14}{21}\selectfont}
\def\Large{\fontsize{16}{24}\selectfont}
\def\LARGE{\fontsize{18}{27}\selectfont}
\def\Huge{\fontsize{20}{30}\selectfont}

%\titleformat{\section}{\bf\Large}{\arabic{section}}{24pt}{}
%\titleformat{\subsection}{\large}{\\arabic{subsection}.}{12pt}{}
\titlespacing*{\subsection}{0pt}{0pt}{1.5ex}

\parindent=24pt

\DeclarePairedDelimiter{\abs}{\lvert}{\rvert}
\DeclarePairedDelimiter{\norm}{\lVert}{\rVert}
\DeclarePairedDelimiter{\inpd}{\langle}{\rangle}
\DeclarePairedDelimiter{\ceil}{\lceil}{\rceil}
\DeclarePairedDelimiter{\floor}{\lfloor}{\rfloor}

\newcommand{\unit}[1]{\:(\text{#1})}
\newcommand{\img}{\mathsf{i}}
\newcommand{\ex}{\mathsf{e}}
\newcommand{\dD}{\mathrm{d}}
\newcommand{\dI}{\,\mathrm{d}}
\DeclareSIUnit \uF {\micro \farad}
\DeclareSIUnit \mH {\milli \henry}

\newcommand{\tri}{$\rhd$}

\title{ \bf {\huge 電子電路實驗6: Linear DC Regulator }\\ 實驗預報}
\author{B02901178 江誠敏}
%\date{2014/09/21}

\begin{document}

\maketitle

\section{Objectives}
\begin{enumerate}
  \item To familiarize with the construction and characteristics of Linear Regulator. 
  \item To visualize how Zener diode operate and its $I\text{--}V$ characteristics
\end{enumerate}


\section{Procedures}
\subsection{Differential Mode Small Signal Analysis}
\begin{enumerate}[itemsep=0pt]
\item Use $\SI{1}\kohm$ resistance for $R_2$ , $\SI{100}\kohm$ variable resistance for $R_L$,
  $\SI{10}\kohm$ variable resistor for $R$ , $R_1$ in Fig. 2. 
\item In order to measure the breakdown voltage of the Zener diode, slowly increase 
  $V_{DC}$ until the C.C. signal alerts in the power supply.   
\item Record the $V_{DC}$ value as breakdown voltage $V_r$ when the C.C. signal alerts. 
\item Record the breakdown voltage $V_r$.
\item Provide voltage source $+\SI{15}\V$, $\SI{-15}\V$, and $V_{DC}  = \SI{10}\V$ to the circuit. 
\item No need to use the Oscilloscope and function generator in the experiment 
\item Adjust the variable resistor $R_a$nd observe how the output voltage will change. 
\item Adjust the variable resistor $R_1$ and $R$ until the output voltage $V_o = \SI{5}{\V}$. 
\item Record the value of $R_1, R$. (Usually, $R \leq \SI{500}\ohm$) 
\item Adjust the variable resistor $R_L$ and observe how the output voltage will change.  
\item Record the output voltage $V_o$ and $R_L$.
\item Keep the previous adjustment of $R$ , $R_1$ constantly and adjust $R_L = 50 \text{--} 100 \si{\kohm}$ . 
\item Adjust the input voltage $V_{DC}$ and observe how the output voltage will change. 
\item Record the output voltage $V_o$ and $V_{DC}$.
\end{enumerate}

\end{document}


