\documentclass[12pt, a4paper]{article}

\usepackage[hmargin=2.5cm, vmargin=2cm]{geometry}
\usepackage{amsthm, amssymb, mathtools, yhmath, graphicx}
\usepackage{fontspec, type1cm, titlesec, titling, fancyhdr, tabularx}
\usepackage{caption}
\usepackage{color}
\usepackage{hhline}
\usepackage{unicode-math}
\usepackage{nicefrac}
\usepackage[abbreviations, per-mode=symbol]{siunitx}
\usepackage{comment}
\usepackage{float}
\usepackage{subcaption}

\usepackage[CheckSingle, CJKmath]{xeCJK}
\usepackage{CJKulem}
\usepackage{enumitem}
\usepackage[usenames, dvipsnames]{xcolor}
\usepackage{colortbl}
\usepackage{circuitikz}
%\setCJKmainfont[BoldFont=cwTex Q Hei]{cwTex Q Ming}
%\setCJKsansfont[BoldFont=cwTex Q Hei]{cwTex Q Ming}
%\setCJKmonofont[BoldFont=cwTex Q Hei]{cwTex Q Ming}
\setCJKmainfont[BoldFont=cwTeX Q Hei]{cwTeX Q Ming}

\def\normalsize{\fontsize{12}{18}\selectfont}
\def\large{\fontsize{14}{21}\selectfont}
\def\Large{\fontsize{16}{24}\selectfont}
\def\LARGE{\fontsize{18}{27}\selectfont}
\def\Huge{\fontsize{20}{30}\selectfont}

\titleformat{\section}{\bf\Large}{\arabic{section}}{24pt}{}
\titleformat{\subsection}{\large}{\arabic{subsection}.}{12pt}{}
\titlespacing*{\subsection}{0pt}{0pt}{1.5ex}

\parindent=24pt

\DeclarePairedDelimiter{\abs}{\lvert}{\rvert}
\DeclarePairedDelimiter{\norm}{\lVert}{\rVert}
\DeclarePairedDelimiter{\inpd}{\langle}{\rangle}
\DeclarePairedDelimiter{\ceil}{\lceil}{\rceil}
\DeclarePairedDelimiter{\floor}{\lfloor}{\rfloor}

\newcommand{\unit}[1]{\:(\text{#1})}
\newcommand{\img}{\mathsf{i}}
\newcommand{\ex}{\mathsf{e}}
\newcommand{\dD}{\mathrm{d}}
\newcommand{\dI}{\,\mathrm{d}}
\DeclareSIUnit \uF {\micro \farad}
\DeclareSIUnit \mH {\milli \henry}

\newcommand{\tri}{$\rhd$}

\title{ \bf {\huge 電子電路實驗3:VTC of CMOS Amplifier Circuits}\\ 實驗預報}
\author{B02901178 江誠敏}
%\date{2014/09/21}

\begin{document}

\maketitle

\section{Objectives}
\begin{enumerate}
  \item To familiarize with the measurement of VTC (Voltage Transfer Curve) for
    CMOS amplifier circuits.
  \item The effects of resistance between gate and source terminal for the VTC of
    CMOS inverter.
  \item Application of CMOS inverter and amplifier.
\end{enumerate}


\section{Procedures}
\begin{enumerate}[itemsep=0pt, label=\alph*.]
  \item {\large CMOS amplifier as an inverter} 
    \begin{enumerate}[label=(\arabic*)]
      \item In Fig. 1, supply voltage source $V_{DD} = +\SI{8}\V$ to the circuit.
      \item Use function generator to generate $v_i = V_{i(dc)} + v_{i(ac)} \sin(2 \pi ft), V_{i(dc)} = \SI{4}\V,
        v_{i(ac)} = \SI{4}\V, f = \SI{1}{\kHz}$. Provide the input small signal $V_i$ to the breadboard.
      \item Make sure that the $v_i$ is measured from the breadboard by using the
        Function generator  \tri Press the \texttt{FUNC} button \tri Set $\texttt{FREQ} = \SI{1}{\kHz}$, \texttt{SIN}
        \texttt{wave}  \tri  \texttt{ATTN} \texttt{0dB}  \tri  \texttt{SUB} \texttt{FUNC}  \tri  \texttt{OFFSET ON}  
        \tri \texttt{Adjust} \texttt{DC/OFFSET} and set dc offset value $V_{i(dc)} = \SI{4}\V$.
      \item \texttt{Oscilloscope}  \tri Press the \texttt{CH1} and \texttt{CH2} \texttt{MENU}  \tri  \texttt{Coupling} \tri DC .
      \item \texttt{Oscilloscope}  \tri Press the \texttt{DISPLAY} button  \tri  \texttt{Format}  \tri \texttt{XY} \texttt{mode} , the
        diagram will be the same as that shown in Fig. 2.
      \item Adjust the $\texttt{VOLTS/DIV}_{\texttt{in}}$ \texttt{CH1} and \texttt{CH2} so that the transition region of
        the diagram is obvious enough to determine the voltage gain.
      \item Record the voltage gain (Reference value $= −\SI{20}{\volt\per\volt}$) by
        observing the slope of the VTC in the transition region (the
        differentiation of input and output voltage value) shown in the curve at \texttt{XY}
        \texttt{mode}.
      \item Change the input voltage signal, and observe whether the shape of the
        diagram in \texttt{XY mode} is consistent.
      \item Record the value of input voltage source as it change.
    \end{enumerate}
  \item {\large CMOS analog circuit experiment} 
    \begin{enumerate}[label=(\arabic*)]
      \item In Fig. 3, use $R = \SI{20}\kohm, \SI{510}\kohm, \SI{1}\Mohm, \SI{3.9}{\kohm}, \text{ and } \SI{10}\Mohm$, respectively.
      \item Supply voltage source $V_{DD} = +\SI{8}\V$ to the circuit.
      \item Use function generator to generate $v_i = V_{i(dc)} + v_{i(ac)} \sin(2 \pi ft), V_{i(dc)} = \SI{4}\V,
        v_{i(ac)} = \SI{4}\V, f = \SI{1}{\kHz}$. Provide the input small signal $V_i$ to the breadboard.
      \item Make sure that the $v_i$ is measured from the breadboard by using the
        probe from CH1 in oscilloscope.
      \item Function generator  \tri Press the \texttt{FUNC} button  \tri $\texttt{FREQ} = \SI{1}{\kHz}$, \texttt{SIN}
        wave  \tri  \texttt{ATTN 0dB}  \tri  \texttt{SUB FUNC}  \tri  \texttt{OFFSET} \texttt{ON}  \tri Adjust
        \texttt{DC/OFFSET} and set dc offset value $V_{i(dc)} = \SI{4}\V$.
        Push the \texttt{DISPLAY} button  \tri  \texttt{Format}  \tri  \texttt{XY mode} , the diagram will be
        the same as that shown in Fig. 2.
      \item Adjust the \texttt{VOLTS/DIV} in \texttt{CH1} and \texttt{CH2} so that the transition region of
        the diagram is obvious enough to determine the voltage gain.
      \item Record the voltage gain $A_v$ in the follow table by observing the
        differentiation of input and output voltage value shown in the curve at \texttt{XY
        mode}.
        (Referent value = constantly $−\SI{20}{\volt \per \volt}$)
      \item Change the input voltage source $V_{DD}$ , and observe whether the shape of the
        diagram in XY mode is consistent.
      \item Record the value of input voltage source as it change:
    \end{enumerate}
  \item {\large CMOS analog amplifier circuit experiment}
    \begin{enumerate}[label=(\arabic*)]
      \item Supply voltage source $V_{DD} = +\SI{8}\V$ to the circuit.
      \item Use function generator to generate $v_i = V_{i(dc)} + v_{i(ac)} sin(2 \pi ft), v_{i(ac)} = \SI{4}\V, f
        = 1kHz$. Provide the input small signal $V_i$ to the breadboard.
      \item Function generator  \tri  \texttt{SUB FUNC}  \tri  \texttt{OFFSET OFF}.
      \item Make sure that the $v_i$ is measured from the breadboard
        by using the
        probe from CH1 in oscilloscope.
      \item Use voltage power supplier to supply DC voltage $V_{i(dc)} = \SI{4}\V$.
      \item Please beware whether the DC voltage and Small-Signal Voltage (SSV)
        supplement are properly connected in the circuit. (DC → SSV →
        ground. Do you know why?)
      \item Adjust the $v_{i(ac)}$ and $V_{i(dc)}$ so that $v_{o(ac)}$ can be achieved to the highest value
        and not be curtailed.
      \item Push the \texttt{DISPLAY} button  \tri  \texttt{Format}  \tri  \texttt{YT mode}  \tri  Press \texttt{MEASURE}
        button.
      \item Record the value in the table as follow.
      \item Make sure that the output amplitude achieves to the maximum and the $V_{i(dc)}$
        is well selected so that the output waveform is symmetry between positive
        half cycle and the negative half cycle.

    \end{enumerate}


\end{document}


