\documentclass[12pt, a4paper]{article}

\usepackage[hmargin=2.5cm, vmargin=2cm]{geometry}
\usepackage{amsthm, amssymb, mathtools, yhmath, graphicx}
\usepackage{fontspec, type1cm, titlesec, titling, fancyhdr, tabularx}
\usepackage{caption}
\usepackage{color}
\usepackage{hhline}
\usepackage{unicode-math}
\usepackage{nicefrac}
\usepackage[abbreviations, per-mode=symbol]{siunitx}
\usepackage{comment}
\usepackage{float}
\usepackage{subcaption}

\usepackage[CheckSingle, CJKmath]{xeCJK}
\usepackage{CJKulem}
\usepackage{enumitem}
\usepackage[usenames, dvipsnames]{xcolor}
\usepackage{colortbl}
\usepackage{circuitikz}
%\setCJKmainfont[BoldFont=cwTex Q Hei]{cwTex Q Ming}
%\setCJKsansfont[BoldFont=cwTex Q Hei]{cwTex Q Ming}
%\setCJKmonofont[BoldFont=cwTex Q Hei]{cwTex Q Ming}
\setCJKmainfont[BoldFont=cwTeX Q Hei]{cwTeX Q Ming}

\def\normalsize{\fontsize{12}{18}\selectfont}
\def\large{\fontsize{14}{21}\selectfont}
\def\Large{\fontsize{16}{24}\selectfont}
\def\LARGE{\fontsize{18}{27}\selectfont}
\def\Huge{\fontsize{20}{30}\selectfont}

%\titleformat{\section}{\bf\Large}{\arabic{section}}{24pt}{}
%\titleformat{\subsection}{\large}{\\arabic{subsection}.}{12pt}{}
\titlespacing*{\subsection}{0pt}{0pt}{1.5ex}

\parindent=24pt

\DeclarePairedDelimiter{\abs}{\lvert}{\rvert}
\DeclarePairedDelimiter{\norm}{\lVert}{\rVert}
\DeclarePairedDelimiter{\inpd}{\langle}{\rangle}
\DeclarePairedDelimiter{\ceil}{\lceil}{\rceil}
\DeclarePairedDelimiter{\floor}{\lfloor}{\rfloor}

\newcommand{\unit}[1]{\:(\text{#1})}
\newcommand{\img}{\mathsf{i}}
\newcommand{\ex}{\mathsf{e}}
\newcommand{\dD}{\mathrm{d}}
\newcommand{\dI}{\,\mathrm{d}}
\DeclareSIUnit \uF {\micro \farad}
\DeclareSIUnit \mH {\milli \henry}

\newcommand{\tri}{$\rhd$}

\title{ \bf {\huge 電子電路實驗4: Single-Stage BJT Amplifiers }\\ 實驗預報}
\author{B02901178 江誠敏}
%\date{2014/09/21}

\begin{document}

\maketitle

\section{Objectives}
\begin{enumerate}
  \item To analyze the dc bias from the device characteristic, and derive device
    parameters for BJT.
  \item Design the CE (Common Emitter, CE) amplifier by using the small-signal
        model.
  \item Measuring the fundamental characteristic, gain, bandwidth and resistance of
        BJT and confirm the measured data with those in 2. 
\end{enumerate}


\section{Procedures}
\subsection{Small-signal analysis}
\begin{enumerate}[itemsep=0pt]
  \item In Fig. 2, use $\SI{10}{\kohm}$ variable resistance for $R_1 , R_2 , R_C , R_E , R_S = 0, R_L = \infty$ . 
  \item Supply DC voltage source $V_{CC} = +\SI{9}\V$ to the circuit.
  \item  Provide the input small signal $V_i$ to the breadboard by using function generator to generate $V_i
    = v_{ac} \sin(2 \pi f t), 2 v_{ac} = \SI{20}\mV $, $f = 10 \~{} 20 \si{\kHz}$.
  \item  Make sure that the $v_i$ is measured from the breadboard by using the probe from CH1
  in oscilloscope.
\item  Function generator \tri  Press the \texttt{FUNC} button \tri  Set \texttt{FREQ} $ = 10 \~{} 20 \si{kHz}$,
  \texttt{SIN} wave \tri  \texttt{ATTN} $\SI{40}{\dB}$.  
\item  Oscilloscope \tri  Press the \texttt{CH1}  and \texttt{CH2} \texttt{MENU} \tri  \texttt{Coupling} \tri 
  \texttt{AC} .  
\item  Oscilloscope \tri  Press the \texttt{Measure} button \tri  Observe $V_{(p-p)}$ in \texttt{CH1} and \texttt{CH2} .
  \item  Adjust the variable resistance of $R_1 , R_2 , R_C \text{ and } R_E$ so that voltage gain could be reached to
    $A_M = \SI{100}{\volt\per\volt}$.
  \item  Record the voltage gain: $\si{\volt\per\volt}$ by observing the differentiation
    of input and output voltage peak-to-peak value shown in the curve at \texttt{YT} mode.
  \item 
    Function generator \tri  Press the \texttt{FUNC} button \tri  Adjust Frequency and observe the voltage gain $A_V$
  in oscilloscope until $A_V = 0.707  A_M$ .
  \item Record the frequency: $f_{L(a)}$
  \item  Using the principle of major pole, change the value of $C_2$ until $f_L = \SI{100}\Hz$. 
  \item Record the value of $C_2$.
  \item Change the frequency of input voltage signal, and record the input and output voltage shown in oscillos cope to the following table
Keep the previous adjustment consta nt, and DO NOT disconnect the signal input from functional generator.
  \item 
    Use multi-(RLC) meter to measure the following values: $R_{B1}, R_{B2}, R_C, R_E, V_E, V_C, V_B$, $I_E, I_C,I_B = I_E-I_C, \beta$. 
\end{enumerate}
\subsection{Measuring small-signal input and output resistance}
\begin{enumerate}[itemsep=0pt]
  \item In Fig. 3, use $\SI{10}\kohm$ variable resistance for $R_V$ .
  \item Adjust $R_V$ so that the measured peak-to-peak voltage value in “oscilloscope node” of Fig. 3 is half of $V_i$ .
  \item Record the value of $R_V = R_i$.
  \item In Fig. 4, use $\SI{10}\kohm$ variable resistance for $R_V$ .
  \item Adjust $R_V$ so that the measured peak-to-peak voltage value in “oscilloscope node” of Fig. 4 is half of $V_O$ .
  \item Record the value of $R_V = R_O$
\end{enumerate}
\end{document}


