\documentclass[12pt, a4paper]{article}

\usepackage[hmargin=2.5cm, vmargin=2cm]{geometry}
\usepackage{amsthm, amssymb, mathtools, yhmath, graphicx}
\usepackage{fontspec, type1cm, titlesec, titling, fancyhdr, tabularx}
\usepackage{color, unicode-math, float, hhline}

\usepackage[CheckSingle, CJKmath]{xeCJK}
\usepackage{CJKulem}
\usepackage{enumitem}
\usepackage{tikz}
\usepackage{circuitikz}
%\setCJKmainfont[BoldFont=cwTex Q Hei]{cwTex Q Ming}
%\setCJKsansfont[BoldFont=cwTex Q Hei]{cwTex Q Ming}
%\setCJKmonofont[BoldFont=cwTex Q Hei]{cwTex Q Ming}
\setCJKmainfont[BoldFont=cwTeX Q Hei]{cwTeX Q Ming}

\def\normalsize{\fontsize{12}{18}\selectfont}
\def\large{\fontsize{14}{21}\selectfont}
\def\Large{\fontsize{16}{24}\selectfont}
\def\LARGE{\fontsize{18}{27}\selectfont}
\def\huge{\fontsize{20}{30}\selectfont}

%\titleformat{\section}{\bf\Large}{\arabic{section}}{24pt}{}
%\titleformat{\subsection}{\large}{\arabic{subsection}.}{12pt}{}
%\titlespacing*{\subsection}{0pt}{0pt}{1.5ex}

\parindent=24pt

\DeclarePairedDelimiter{\abs}{\lvert}{\rvert}
\DeclarePairedDelimiter{\norm}{\lVert}{\rVert}
\DeclarePairedDelimiter{\inpd}{\langle}{\rangle}
\DeclarePairedDelimiter{\ceil}{\lceil}{\rceil}
\DeclarePairedDelimiter{\floor}{\lfloor}{\rfloor}

\newcommand{\img}{\mathsf{i}}
\newcommand{\ex}{\mathsf{e}}
\newcommand{\dD}{\mathrm{d}}
\newcommand{\dI}{\,\mathrm{d}}
\usetikzlibrary{intersections, calc}

\title{\vspace{-1.5cm} 經濟學 HW \#1 \vspace{-1.5cm}}

\begin{document}
\maketitle
毒品(如古柯鹼、鴉片、大麻)交易在各國都是頭痛的問題,各國政府有不同對應方法,結果也不同。

\begin{enumerate}[label=(\arabic*)]
  \item {\bf 請畫出在沒有任何管制下,毒品的供給和需求線(畫一組正常的供給和需求線即可)
    ,並顯示出均衡時的價格和數量。}

    \begin{figure}[H]
      \centering
      \begin{tikzpicture}[
          scale=2,
          D/.style={blue, thick},
          S/.style={red, thick},
          axis/.style={very thick, ->, >=stealth', line join=miter},
          dl/.style={dashed, thick, black!50!gray},
          every node/.style={color=black},
          dot/.style={circle,fill=black,minimum size=4pt,inner sep=0pt,
            outer sep=-1pt},
        ]
        \draw[axis,<->] (2.5,0) node(xline)[right] {$Q$} -|
        (0,2.5) node(yline)[above] {$P$};
        %\draw (0, 0) grid[step=.5] (2.5, 2.5);
        % IS-LM diagram
        \draw[S, name path=supply] 
        (0.4, 0.4) .. controls (1.25, 0.55) and (1.72, 1.4) .. (1.9, 1.9) node[above] {$S$};
        \draw[D, name path=demand] (0.4,1.8) .. controls (0.63, 1.1) and (1.3, 0.4) ..  (2,.3)
        node[right]{$D$}; 
        \path[name intersections={of=supply and demand,by=P}];
        \node[dot] at (P) {};

        \draw[dl] let \p{P}=(P) in (\x{P}, 0) node[below]{$Q_E$} -- (P) -- (0, \y{P}) node[left]{$P_E$};
      \end{tikzpicture}
    \end{figure}
    如上圖所示, $P_E$ 為均衡價格,$Q_E$ 為均衡數量。

  \item {\bf 如果法律規定販毒者(供給方)有罪,但吸毒者(需求方)無罪,均衡會發生什麼變化?}

    接續 (1) ,現在販毒者有罪,因此可預期供給下降,供給曲線將左移。

    \begin{figure}[H]
      \centering
      \begin{tikzpicture}[
          scale=2,
          D/.style={blue, thick},
          S/.style={red, thick},
          SS/.style={red!50!white, thick},
          axis/.style={very thick, ->, >=stealth', line join=miter},
          dl/.style={dashed, thick, black!50!gray},
          dl2/.style={dotted, thick, gray},
          every node/.style={color=black},
          dot/.style={circle,fill=black,minimum size=4pt,inner sep=0pt,
            outer sep=-1pt},
          al/.style={draw, thick, -latex},
        ]
        \draw[axis,<->] (2.5,0) node(xline)[right] {$Q$} -|
        (0,2.5) node(yline)[above] {$P$};
        %\draw (0, 0) grid[step=.5] (2.5, 2.5);
        % IS-LM diagram
        \draw[SS, name path=supply] 
        (0.4, 0.4) .. controls (1.25, 0.55) and (1.72, 1.4) .. (1.9, 1.9) node[above] {$S$};
        \draw[S, name path=supply2] 
        (0.1, 0.45) .. controls (0.95, 0.55) and (1.42, 1.4) .. (1.6, 1.9) node[above] {$S'$};
        \draw[D, name path=demand] (0.4,1.8) .. controls (0.63, 1.1) and (1.3, 0.4) ..  (2,.3)
        node[right]{$D$}; 
        \path[name intersections={of=supply and demand,by=P}];
        \path[name intersections={of=supply2 and demand,by=P2}];
        \node[dot] at (P) {};
        \node[dot] at (P2) {};
        \draw[al] (1.83, 1.85) -- (1.63, 1.85);
        \draw[dl2] let \p{P}=(P) in (\x{P}, 0) node[below]{} -- (P) -- (0, \y{P}) node[left]{};
        \draw[dl] let \p{P}=(P2) in (\x{P}, 0) node[below]{$Q'_E$} -- (P2) -- 
        (0, \y{P}) node[left]{$P'_E$};
      \end{tikzpicture}
    \end{figure}

    如上圖所示, $P'_E$ 為新的均衡價格,與原先的均衡價格相比上升了,
    $Q'_E$ 則為新的均衡數量,下降了。

  \item {\bf 如果法律規定吸毒者(需求方)有罪,但販毒者(供給方)無罪,均衡有何變化?
      和 (2) 的答案有何不同?}

    類似的我們可以預期需求將下降,需求曲線左移。
    \begin{figure}[H]
      \centering
      \begin{tikzpicture}[
          scale=2,
          D/.style={blue, thick},
          S/.style={red, thick},
          DD/.style={blue!50!white, thick},
          axis/.style={very thick, ->, >=stealth', line join=miter},
          dl/.style={dashed, thick, black!50!gray},
          dl2/.style={dotted, thick, gray},
          every node/.style={color=black},
          dot/.style={circle,fill=black,minimum size=4pt,inner sep=0pt,
            outer sep=-1pt},
          al/.style={draw, thick, -latex},
        ]
        \draw[axis,<->] (2.5,0) node(xline)[right] {$Q$} -|
        (0,2.5) node(yline)[above] {$P$};
        %\draw (0, 0) grid[step=.5] (2.5, 2.5);
        % IS-LM diagram
        \draw[S, name path=supply] 
        (0.4, 0.4) .. controls (1.25, 0.55) and (1.72, 1.4) .. (1.9, 1.9) node[above] {$S$};
        \draw[D, name path=demand] (0.4,1.8) .. controls (0.63, 1.1) and (1.3, 0.4) ..  (2,.3)
        node[right]{$D$}; 
        \draw[D, name path=demand2] (0.1,1.8) .. controls (0.33, 1.1) and (1.0, 0.4) ..  (1.7,.25)
        ++(0.15, -0.1) node[]{$D'$}; 
        \path[name intersections={of=supply and demand,by=P}];
        \path[name intersections={of=supply and demand2,by=P2}];
        \node[dot] at (P) {};
        \node[dot] at (P2) {};
        \draw[al] (0.41, 1.65) -- (0.21, 1.65);
        \draw[dl2] let \p{P}=(P) in (\x{P}, 0) node[below]{} -- (P) -- (0, \y{P}) node[left]{};
        \draw[dl] let \p{P}=(P2) in (\x{P}, 0) node[below]{$Q''_E$} -- (P2) -- 
        (0, \y{P}) node[left]{$P''_E$};
      \end{tikzpicture}
    \end{figure}

    如上圖所示, $P''_E$ 為新的均衡價格,與原先的均衡價格相比下降了,
    $Q''_E$ 則為新的均衡數量,同樣的也下降了。

    數量的變動趨勢與 (2) 相同,不同的地方在於價格的變動趨勢。 (2) 中
    的價格上升了,而 (3) 則下降了。

  \item {\bf 在 (2) 和 (3) 的均衡下,吸毒的人有何不同的特性?}

    在 (2) 中雖然需求方無罪,但是價格卻上升了,因此可以推斷在 (2) 的均衡下,
    吸毒者應該會是比較有錢的人。

    而 (2) 中需求方是有罪的,雖然價格下降了,但因人們害怕受到法律懲罰,
    因此需求量仍下降了,可以推斷依然購買的人應該是那些將「個人的快樂」
    放在「法律規範」前的人,或者說是「不怕被關」的人。
\end{enumerate}

\end{document}

