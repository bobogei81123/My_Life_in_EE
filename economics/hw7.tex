\documentclass[12pt, a4paper]{article}

\usepackage[hmargin=2.5cm, vmargin=2cm]{geometry}
\usepackage{amsthm, amssymb, mathtools, yhmath, graphicx}
\usepackage{fontspec, type1cm, titlesec, titling, fancyhdr, tabularx}
\usepackage{color, unicode-math, float, hhline}

\usepackage[abbreviations, per-mode=symbol]{siunitx}
\usepackage[CheckSingle, CJKmath]{xeCJK}
\usepackage{CJKulem}
\usepackage{enumitem}
\usepackage{tikz}
\usepackage{circuitikz}
%\setCJKmainfont[BoldFont=cwTex Q Hei]{cwTex Q Ming}
%\setCJKsansfont[BoldFont=cwTex Q Hei]{cwTex Q Ming}
%\setCJKmonofont[BoldFont=cwTex Q Hei]{cwTex Q Ming}
\setCJKmainfont[BoldFont=cwTeX Q Hei]{cwTeX Q Ming}

\def\normalsize{\fontsize{12}{18}\selectfont}
\def\large{\fontsize{14}{21}\selectfont}
\def\Large{\fontsize{16}{24}\selectfont}
\def\LARGE{\fontsize{18}{27}\selectfont}
\def\huge{\fontsize{20}{30}\selectfont}

%\titleformat{\section}{\bf\Large}{\arabic{section}}{24pt}{}
%\titleformat{\subsection}{\large}{\arabic{subsection}.}{12pt}{}
%\titlespacing*{\subsection}{0pt}{0pt}{1.5ex}

\parindent=0pt

\DeclarePairedDelimiter{\abs}{\lvert}{\rvert}
\DeclarePairedDelimiter{\norm}{\lVert}{\rVert}
\DeclarePairedDelimiter{\inpd}{\langle}{\rangle}
\DeclarePairedDelimiter{\ceil}{\lceil}{\rceil}
\DeclarePairedDelimiter{\floor}{\lfloor}{\rfloor}

\newcommand{\img}{\mathsf{i}}
\newcommand{\ex}{\mathsf{e}}
\newcommand{\dD}{\mathrm{d}}
\newcommand{\dI}{\;\mathrm{d}}

\title{\vspace{-1.5cm} 經濟學 HW \#7}
\author{b02902072 王鼎皓 \quad b03201001 楊松翰 \quad b02901178 江誠敏}
\begin{document}
\maketitle

\begin{enumerate}[label={\bf 8.\arabic*}]
  \setcounter{enumi}{16}
  \item {\bf 當 $a$ 為何時,$A$ 與 $A'$ 方為優勢策略?\\}
    以下用 $f(x, y)$ 表示甲選 $x$ 策略,乙選 $y$ 策略下甲的報酬。\\
    對甲來說,若乙選 $B'$,則因 $f(A, B') = 0 > -1 = f(B, B')$,$A$ 已是最佳策略。\\
    而如果乙選 $A'$,則 $A$ 是優勢策略的條件是 $f(A, A') = a > -10 = f(B, A')$,
    也就是 $a > -10$。\\
    對乙來說,因為此賽局是對稱的,同樣的我們會得到 $A'$ 是優勢策略的條件是 $a > -10$,
    因此結論即是 $a > -10$。

  \item {\bf $a$ 值為多少才是耐許均衡。\\}
    對甲來說,如果乙選 $A'$,則 $A$ 是最適策略的條件是 $f(A, A') = a > -10 = f(B, A')$,
    即 $a > -10$。\\
    而因賽局對稱,對乙來說也一樣,因此結論還是 $a > -10$。
    
  \item {\bf 當 $a = -11$ 時,$(B, B')$ 是否為一耐許均衡?\\}
    不是,因為對甲來說 $f(A, B') = 0 > -1 = f(B, B')$,因此如果乙選擇 $B'$,
    $A$ 才是甲的最適策略而非 $B$,因此 $(B, B')$ 並非耐許均衡。
  \item[{\bf 9.17}] 
    \begin{enumerate}[label={\bf (\arabic*)}]
      \item {\bf 請計算出在各勞動投入下,邊際產量是多少?}
      \item {\bf 如果該產出的單位價格是 10 元,請計算各勞動投入量下的邊際產值。} 
        \begin{table}[H]
          \centering
          \begin{tabular}{c|c|c|c|c}
            投入 & 產出量 & {\bf (1)}邊際產量 & {\bf (2)}邊際產值($P = 10$) & 邊際產值($P = 12.5$) \\
            \hline
            0 & 0 & - & - \\
            \hline
            1 & 5 & 5 & 50 & 62.5 \\
            \hline
            2 & 11 & 6 & 60 & 75.0 \\
            \hline
            3 & 18 & 7 & 70 & 87.5 \\
            \hline
            4 & 26 & 8 & 80 & 100.0 \\
            \hline
            5 & 33 & 7 & 70 & 87.5 \\
            \hline
            6 & 39 & 6 & 60 & 75.0 \\
            \hline
            7 & 44 & 5 & 50 & 62.5 \\
            \hline
            8 & 48 & 4 & 40 & 50.0 \\
            \hline
            9 & 51 & 3 & 30 & 37.5 \\
            \hline
            10 & 53 & 2 & 20 & 25.0 \\
            \hline
            11 & 54 & 1 & 10 & 12.5 \\
            \hline
            12 & 54 & 0 & 0 & 0.0 \\
            \hline
            13 & 53 & -1 & -10 & -12.5 \\
            \hline
          \end{tabular}
        \end{table}
      \item {\bf 如果每單位勞動投入的工資是 50 元,請問最適雇用量是多少?
          在該雇用量下,產出是多少?資本報酬是多少?。\\[10pt]} 
        在最適報酬下有
        \[ \text{邊際產值} = \text{工資} = 50 \]
        可知最適雇用量為 $7$ 。(事實上 $6$ 也是最佳解。)\\
        此時產出為 $44$,資本報酬為
        \[ 10 \cdot 44 - 50 \cdot 7 = 90 \;\text{(元)}\]
      \item {\bf 如果雇主所雇用的勞工數比第 (3) 項所算出的答案多 1 人,請重新計算資本
          報酬是多少?這個資本報酬是否比第 (3) 項所算出的為低?。\\[10pt]} 
        在雇用量為 $8$ 時,產出為 $48$
        資本報酬為
        \[ 10 \cdot 48 - 50 \cdot 8 = 80 \;\text{(元)}\]
        確實較 (3) 來的低。
      \item {\bf 如果產品價格上升為 12.5 元,請重新計算最適投入量。\\[10pt]} 
        與 (3) 同理,在最適報酬下有
        \[ \text{邊際產值} = \text{工資} = 50 \]
        可知新的最適雇用量為 $8$ 。\\
    \end{enumerate}
\end{enumerate}
\end{document}

