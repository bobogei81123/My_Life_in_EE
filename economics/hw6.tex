\documentclass[12pt, a4paper]{article}

\usepackage[hmargin=2.5cm, vmargin=2cm]{geometry}
\usepackage{amsthm, amssymb, mathtools, yhmath, graphicx}
\usepackage{fontspec, type1cm, titlesec, titling, fancyhdr, tabularx}
\usepackage{color, unicode-math, float, hhline}

\usepackage[abbreviations, per-mode=symbol]{siunitx}
\usepackage[CheckSingle, CJKmath]{xeCJK}
\usepackage{CJKulem}
\usepackage{enumitem}
\usepackage{tikz}
\usepackage{circuitikz}
%\setCJKmainfont[BoldFont=cwTex Q Hei]{cwTex Q Ming}
%\setCJKsansfont[BoldFont=cwTex Q Hei]{cwTex Q Ming}
%\setCJKmonofont[BoldFont=cwTex Q Hei]{cwTex Q Ming}
\setCJKmainfont[BoldFont=cwTeX Q Hei]{cwTeX Q Ming}

\def\normalsize{\fontsize{12}{18}\selectfont}
\def\large{\fontsize{14}{21}\selectfont}
\def\Large{\fontsize{16}{24}\selectfont}
\def\LARGE{\fontsize{18}{27}\selectfont}
\def\huge{\fontsize{20}{30}\selectfont}

%\titleformat{\section}{\bf\Large}{\arabic{section}}{24pt}{}
%\titleformat{\subsection}{\large}{\arabic{subsection}.}{12pt}{}
%\titlespacing*{\subsection}{0pt}{0pt}{1.5ex}

\parindent=0pt

\DeclarePairedDelimiter{\abs}{\lvert}{\rvert}
\DeclarePairedDelimiter{\norm}{\lVert}{\rVert}
\DeclarePairedDelimiter{\inpd}{\langle}{\rangle}
\DeclarePairedDelimiter{\ceil}{\lceil}{\rceil}
\DeclarePairedDelimiter{\floor}{\lfloor}{\rfloor}

\newcommand{\img}{\mathsf{i}}
\newcommand{\ex}{\mathsf{e}}
\newcommand{\dD}{\mathrm{d}}
\newcommand{\dI}{\;\mathrm{d}}

\title{\vspace{-1.5cm} 經濟學 HW \#6}
\author{b02902072 王鼎皓 \quad b03201001 楊松翰 \quad b02901178 江誠敏}
\begin{document}
\maketitle

若廠商生產稻米之生產函數為 $Q = 5 K \sqrt{L}$ , 短期之資本量為$K=1$,
工資率$w=5$, 資本租金為$r=5$。

\begin{enumerate}[label={\bf 6.\arabic*}]
    \setcounter{enumi}{2}
    \item {\bf 廠商之 AVC, SAC, SMC 函數各是多少?\\}
      因 $Q = 5 K \sqrt{L} \implies L = \left( Q/(5K) \right) ^2$,計算可得
      \begin{alignat*}{3}
        \text{AVC} & = \frac{w L}{Q} && = \frac{Q}{5} \\
        \text{SAC} & = \frac{w L + r K}{Q} && = \frac{Q}{5} + \frac{5}{Q} \\
        \text{SMC} & = \frac{\partial w L}{\partial Q} && = \frac{2Q}{5}\\
      \end{alignat*}
    \item {\bf 廠商之短期供給曲線為何?\\}
      短期供給曲線即為 SMC 高於 AVC 的部份。
      而因為 \[ \text{SMC} = 2Q / 5 \geq Q/5 = \text{AVC}, \; \forall Q \geq 0 \]
      因此供給曲線就是 SMC ,即 $ 2Q / 5 $。
    \item {\bf 若 $P = 20$,則廠商之最適產量為何?\\}
      \[ P = 2Q / 5 \implies Q = 2.5P = 50. \]
    \item {\bf 在短期,價格低到什麼程度,廠商才會退出市場?\\}
      當 SMC 低於 AVC 時廠商會退出市場,但因 SMC 恆高於 AVC,
      所以廠商在短期上不論價格都不會退出市場。
    \setcounter{enumi}{8}
    %\item {\bf 當 $P = 20$ 時,廠商的生產者剩餘是多少?\\}
      %\begin{align*}
        %\text{生產者剩餘} &= P^* Q^* - \int_{0}^{Q^*} P(Q) \dI Q \\
        %&= P^* Q^* - \int_{0}^{Q^*} \frac{2Q}{5} \dI Q \\
        %&= P^* Q^* - \left. \frac{Q^2}{5} \right|_{0}^{Q^*} = 500.
      %\end{align*}
    \item {\bf 若 $P = 20$,且老闆種米之機會成本為 $150$,則其短期經濟利潤是多少?\\}
      \[ PQ - (wL + rK + 150) = PQ - \left( w \frac{Q^2}{25K} + rK + 150 \right) = 345. \]

    \item {\bf 若整個經濟共有 $100$ 家相同的廠商,社會的總供給是多少?\\}
      每一家廠當的供給曲線都是 $P = 2Q / 5$ 也就是 $Q = 5P / 2$,而 $100$ 家廠商
      即數量為 $100$ 倍,因此總供給為
      \[ Q = 250P \]

    \item {\bf 若社會的總需求是 $Q = 252 - 2P$,則均衡之市場價格是多少?\\}
      解
      \[
        \left\{
          \begin{align*}
            Q^* &= 250P^* \\
            Q^* &= 252 - 2P^* \\
          \end{align*}
        \right.
      \]
      得到 $P^* = 1$。

    \item{\bf 若 $P = 1$,則廠商的最適供給量為何?}
      \[ P = 2Q / 5 = 2.5 \]

    \item{\bf 若 $P = 1$,則廠商的利潤值為何?\\}
      此時 $L = Q^2 / (25K) = 0.25$,則
      \[ \text{利潤} = PQ - (wL + rK) = -3.75 \]

    \item{長期而言,$P$ 會大於或小於 $1$ ?}
      因為利潤值為負,長期而言會有廠商退出,而使供給量下跌,最終使價格上升,
      因此長期而言 $P$ 因該會大於 $1$。
\end{enumerate}

\begin{enumerate}[label={\bf 7.\arabic*}]
    \setcounter{enumi}{1}
    \item {\bf 某書局敦南店對附近居民而言是獨占,其書藉需求公式如下:
        \[ Q = 20-2P, \]
      式中 $Q$ 為需求量、$P$ 為價格。假設每本書的成本都是 $5$ 元,且無固定成本。請問
      老闆可以賺多少錢?最適定價為多少?會賣出多少本?}

    先計算出 MR,因為 $P = 10 - Q/2$,
    \[ \text{MR} = \frac{\partial (PQ)}{\partial Q} = \frac{\partial P}{\partial Q}Q + P
      = 10 - Q \]
    而 SMC 恰好為 $5$,因此在最適定價時,
    \[ \text{MR} = \text{SMC} \implies 10 - Q = 5 \implies Q = 5 \]
    此時價格為 $10 - 5/2 = 7.5$,總共可賺 $(7.5 - 5) \cdot 5 = 12.5$。\\
    總結以上,老闆可以賺 $12.5$ 元,最適定價為 $7.5$ 元,會賣出 $5$ 本。
    
    \item {\bf 上題的消費者剩餘有多少?\\}
      若需求曲線為 $P(Q)$,用連續的觀點計算:
      \begin{align*}
        \text{消費者剩餘} &= \int_{0}^{Q^*} P(Q) \dI Q - P^* Q^*\\
        &= \int_{0}^{Q^*} \left(10 - \frac{Q}{2} \right) \dI Q - P^* Q^*\\
        &= \left. \left( 10Q - \frac{Q^2}{4} \right) \right|_{0}^{Q^*} - P^* Q^* = 6.25.
      \end{align*}
      如果以離散的方式計算,
      \[ \text{消費者剩餘} = \sum_{q=1}^{5} (P(q) - P^*) = 2.5+1.5+1+0.5+0 = 5 \]
 
  \end{enumerate}
\end{document}

