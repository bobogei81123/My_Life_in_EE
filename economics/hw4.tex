\documentclass[12pt, a4paper]{article}

\usepackage[hmargin=2.5cm, vmargin=2cm]{geometry}
\usepackage{amsthm, amssymb, mathtools, yhmath, graphicx}
\usepackage{fontspec, type1cm, titlesec, titling, fancyhdr, tabularx}
\usepackage{color, unicode-math, float, hhline, subcaption}

\usepackage[CheckSingle, CJKmath]{xeCJK}
\usepackage{CJKulem}
\usepackage{enumitem}
\usepackage{tikz}
\usepackage{circuitikz}
%\setCJKmainfont[BoldFont=cwTex Q Hei]{cwTex Q Ming}
%\setCJKsansfont[BoldFont=cwTex Q Hei]{cwTex Q Ming}
%\setCJKmonofont[BoldFont=cwTex Q Hei]{cwTex Q Ming}
\setCJKmainfont[BoldFont=cwTeX Q Hei]{cwTeX Q Ming}

\def\normalsize{\fontsize{12}{18}\selectfont}
\def\large{\fontsize{14}{21}\selectfont}
\def\Large{\fontsize{16}{24}\selectfont}
\def\LARGE{\fontsize{18}{27}\selectfont}
\def\huge{\fontsize{20}{30}\selectfont}

%\titleformat{\section}{\bf\Large}{\arabic{section}}{24pt}{}
%\titleformat{\subsection}{\large}{\arabic{subsection}.}{12pt}{}
%\titlespacing*{\subsection}{0pt}{0pt}{1.5ex}

\parindent=24pt

\DeclarePairedDelimiter{\abs}{\lvert}{\rvert}
\DeclarePairedDelimiter{\norm}{\lVert}{\rVert}
\DeclarePairedDelimiter{\inpd}{\langle}{\rangle}
\DeclarePairedDelimiter{\ceil}{\lceil}{\rceil}
\DeclarePairedDelimiter{\floor}{\lfloor}{\rfloor}

\newcommand{\img}{\mathsf{i}}
\newcommand{\ex}{\mathsf{e}}
\newcommand{\dD}{\mathrm{d}}
\newcommand{\dI}{\,\mathrm{d}}
\usetikzlibrary{intersections, calc}

\title{\vspace{-1.5cm} 經濟學 HW \#4}
\author{b02902072 王鼎皓 \quad b03201001 楊松翰 \quad b02901178 江誠敏}

\begin{document}
\maketitle
假設消費者只消費兩種商品:衣服和食物,衣服每單位價格 20 元,食物每單位價格 5 元,消費者的預算是 1,000 元。

\begin{enumerate}[label=(\arabic*)]
  \item {\bf 請繪出消費者的預算線。}

    以下都令 $(X_1, X_2)$ 代表食物和衣服分別買 $X_1$, $X_2$ 單位的消費組合。\\
    並令 $P_1 = 5, P_2 = 20$ 分別代表食物和衣服的單位價格。

    預算線為 
    \[ P_1 X_1 + P_2 X_2 = 1000, \]
    如下圖所示。

    \begin{figure}[H]
      \centering
      \begin{tikzpicture}[
          scale=2,
          bl/.style={blue, very thick},
          S/.style={red, thick},
          axis/.style={very thick, ->, >=stealth', line join=miter},
          dl/.style={dashed, thick, black!50!gray},
          every node/.style={color=black},
          dot/.style={circle,fill=black,minimum size=5pt,inner sep=0pt,
            outer sep=-1pt},
        ]
        \draw[axis,<->] (3,0) node(xline)[right] {$X_1$} -|
        (0,2) node(yline)[above] {$X_2$};

        \draw[bl] 
        (0, 1) node[dot]{} node[left]{$(0, 50)$} -- (2, 0) node[dot]{} node[below]{$(200, 0)$};
      \end{tikzpicture}
    \end{figure}

  \item {\bf 若消費者選擇的消費點是(100 單位食物, 25 單位衣服),請繪出無異曲線和預算線相切的樣子。}

    可知兩曲線需相切於 $(100, 25)$。

    \begin{figure}[H]
      \centering
      \begin{tikzpicture}[
          scale=2,
          bl/.style={blue, very thick},
          rl/.style={red, very thick},
          axis/.style={very thick, ->, >=stealth', line join=miter},
          dl/.style={dashed, thick, black!50!gray},
          every node/.style={color=black},
          dot/.style={circle,fill=black,minimum size=5pt,inner sep=0pt,
            outer sep=-1pt},
        ]
        \draw[axis,<->] (3,0) node(xline)[right] {$X_1$} -|
        (0,2) node(yline)[above] {$X_2$};

        \draw[bl] 
        (0, 1) node[dot]{} node[left]{$(0, 50)$} -- (2, 0) node[dot]{} node[below]{$(200, 0)$};
        \draw (1, 0.5) node[dot] node[below left]{$(100, 25)$};
        \draw[domain=0.26:2.5, samples=100, rl, latex-latex] plot[id=x] function{1 / (2*x)} coordinate(v1);
        \draw (v1) ++(0, 0.2) node{\footnotesize $U = U_0$};
      \end{tikzpicture}
    \end{figure}

  \item {\bf 政府擔心消費者吃不飽,因此送給每個消費者 50 單位食物券,憑券可免費購買 50 單位食物,但不能做他用。請繪出消費者新的預算線及消費均衡點。食物券的發放能否保證消費者一定會吃超過 100 單位的食物?}

    \begin{figure}[H]
      \centering
      \begin{tikzpicture}[
          scale=2,
          bl/.style={blue, line width=0.25mm},
          rl/.style={red, line width=0.25mm},
          axis/.style={very thick, ->, >=stealth', line join=miter},
          dl/.style={dashed, thick, black!50!gray},
          every node/.style={color=black},
          dot/.style={circle,fill=black,minimum size=5pt,inner sep=0pt,
            outer sep=-1pt},
        ]
        \draw[axis,<->] (3.3,0) node(xline)[right] {$X_1$} -|
        (0,2) node(yline)[above] {$X_2$};

        \draw[bl] 
        (0, 1) node[dot]{} node[left]{$(0, 50)$} -- (2, 0) node[dot]{};
        \draw[bl, very thick, dotted]
        (0.5, 1) -- (0, 1) ;
        \draw[bl, very thick]
        (0.5, 1) node[dot]{} node[left]{} -- (2.5, 0) node[dot]{} node[below]{$(250, 0)$};
        \draw[domain=0.26:2.5, samples=100, rl, latex-latex] plot[id=x] function{1 / (2*x)} coordinate(v1);
        \draw[domain=0.33:2.5, samples=100, rl, very thick, latex-latex] plot[id=x] function{25 / (32*x)}
        node[above]{\footnotesize $U = U_1$};;
        \draw (2.5, 0.15) node[right]{\footnotesize $U = U_0$};
        \draw[very thick, -latex] (0.43, 0.5) node[below]{$(50, 50)$} -- (0.5, 1);
        \draw (1, 0.5) node(c0)[dot]{} node[below]{$C_0$};
        \draw (1.25, 0.625) node(c1)[dot]{} node[above right]{$C_1$};
        \draw[-latex, thick, green!60!black] (c0) -- (c1);
      \end{tikzpicture}
      \caption{食物為正常財}
    \end{figure}

    \begin{figure}[H]
      \centering
      \begin{tikzpicture}[
          scale=2,
          bl/.style={blue, line width=0.25mm},
          rl/.style={red, line width=0.25mm},
          axis/.style={very thick, ->, >=stealth', line join=miter},
          dl/.style={dashed, thick, black!50!gray},
          every node/.style={color=black},
          dot/.style={circle,fill=black,minimum size=5pt,inner sep=0pt,
            outer sep=-1pt},
        ]
        \draw[axis,<->] (3.3,0) node(xline)[right] {$X_1$} -|
        (0,2) node(yline)[above] {$X_2$};

        \draw[bl] 
        (0, 1) node[dot]{} node[left]{$(0, 50)$} -- (2, 0) node[dot]{};
        \draw[bl, very thick, dotted]
        (0.5, 1) -- (0, 1) ;
        \draw[bl, very thick]
        (0.5, 1) node[dot]{} node[left]{} -- (2.5, 0) node[dot]{} node[below]{$(250, 0)$};
        \draw[domain=0.26:2.5, samples=100, rl, latex-latex] plot[id=x] function{1 / (2*x)} coordinate(v1);
        \draw[domain=0.33:2.5, samples=100, rl, very thick, latex-latex] plot[id=x] function{0.71616 / (x**0.3) + 3.5*exp(-4.5*x)}
        node[above]{\footnotesize $U = U_2$};;
        \draw (2.5, 0.15) node[right]{\footnotesize $U = U_0$};
        \draw[very thick, -latex] (0.43, 0.5) node[below]{$(50, 50)$} -- (0.5, 1);
        \draw (1, 0.5) node(c0)[dot]{} node[below]{$C_0$};
        \draw (0.91343, 0.79329) node(c1)[dot]{} node[above right]{$C_2$};
        \draw[-latex, thick, green!60!black] (c0) -- (c1);
      \end{tikzpicture}
      \caption{食物為劣等財}
    \end{figure}

    不一定,如上兩圖所示,新的最適選擇點可能為 $C_1, C_2$ ,且在 $C_2$ 的時候買食物的量
    反而比沒有發食物券前的 $100$ 單位還少。

    \clearpage

  \item {\bf 如果政府不發食物券,而以等值現金250元發給消費者,則預算線有何不同?消費者的滿足水準會變高還是變低?}

    \begin{figure}[H]
      \centering
      \begin{subfigure}[b]{0.49\textwidth}
        \centering
        \begin{tikzpicture}[
            scale=1.6,
            bl/.style={blue, very thick},
            S/.style={red, thick},
            axis/.style={very thick, ->, >=stealth', line join=miter},
            dl/.style={dashed, thick, black!50!gray},
            every node/.style={color=black},
            dot/.style={circle,fill=black,minimum size=5pt,inner sep=0pt,
              outer sep=-1pt},
          ]
          \draw[axis,<->] (3,0) node(xline)[right] {$X_1$} -|
          (0,2) node(yline)[above] {$X_2$};

          \draw[bl] 
          (0.5, 1) node[dot]{} node[above right]{$(50, 50)$} -- (2.5, 0) node[dot]{} node[below]{$(200, 0)$};
          \draw[bl, dotted] 
          (0.5, 1) -- (0, 1) node[dot]{} node[left]{$(0, 50)$};
        \end{tikzpicture}
        \caption{食物券}
      \end{subfigure}%
      \begin{subfigure}[b]{0.49\textwidth}
        \centering
        \begin{tikzpicture}[
            scale=1.6,
            bl/.style={blue, very thick},
            S/.style={red, thick},
            axis/.style={very thick, ->, >=stealth', line join=miter},
            dl/.style={dashed, thick, black!50!gray},
            every node/.style={color=black},
            dot/.style={circle,fill=black,minimum size=5pt,inner sep=0pt,
              outer sep=-1pt},
          ]
          \draw[axis,<->] (3,0) node(xline)[right] {$X_1$} -|
          (0,2) node(yline)[above] {$X_2$};

          \draw[bl] 
          (0, 1.25) node[dot]{} node[above right]{$(0, 62.5)$} -- (2.5, 0) node[dot]{} node[below]{$(200, 0)$};
        \end{tikzpicture}
        \caption{等值現金}
      \end{subfigure}
    \end{figure}

    如上圖所示,因為食物券不能做他用,所以即使食物買的再少,可以買衣服的數量不會變,還是 $1000/20 = 50$ 單位。
    而等值現金就沒有這個問題,因此食物券不會有 $(0, 62.5)$ 到 $(50, 50)$ ,也就是衣服數量超過 $50$ 的那一段。

    而假設在等值現金的狀況下,最適選擇點為 $(x_1, x_2)$。如果 $y_1 \leq 50$,那麼在食物券的
    情況下,因為此時預算線重合,最適選擇點是相同的,自然滿足水準也相同。

    但如果 $x_1 > 50$,那麼就有可能在食物券的情況下無法達到這個最適選擇點,滿足水準就會比等值現金低。

    \begin{figure}[H]
      \centering
      \begin{tikzpicture}[
          scale=2,
          bl/.style={blue, line width=0.25mm},
          rl/.style={red, line width=0.25mm},
          axis/.style={very thick, ->, >=stealth', line join=miter},
          dl/.style={dashed, thick, black!50!gray},
          every node/.style={color=black},
          dot/.style={circle,fill=black,minimum size=5pt,inner sep=0pt,
            outer sep=-1pt},
        ]
        \draw[axis,<->] (3.3,0) node(xline)[right] {$X_1$} -|
        (0,2) node(yline)[above] {$X_2$};

        \draw[bl, very thick, dotted]
        (0.8, 0.85) -- (0, 0.85) node[dot]{} node[left]{$(0, 50)$};
        \draw[bl, very thick]
        (0, 1.25) node[dot]{} node[left]{$(0, 62.5)$} -- (0.8, 0.85) node[dot]{} 
        node[left]{} -- (2.5, 0) node[dot]{} node[below]{$(250, 0)$};
        \draw[domain=0.05:2.5, samples=100, rl, very thick, -latex] plot[id=x] function{-(1.0/6)*log(x)+0.81281}
        node[below]{\footnotesize $U = U_1$};
        \draw[domain=0.05:2.5, samples=100, rl, very thick, -latex] plot[id=x] function{-(1.0/6)*log(x)+0.90023}
        node[above]{\footnotesize $U = U_2$};
        \draw[very thick, -latex] (0.43, 0.5) node[below]{$(50, 50)$} -- (0.8, 0.85);
        \draw (0.8, 0.85) node(c1)[dot]{} node[below]{$C_1$};
        \draw (0.33333, 1.08333) node(c2)[dot]{} node[above right]{$C_2$};
      \end{tikzpicture}
    \end{figure}

    如上圖所示,假如在等值現金下的最適選擇點為 $C_2 = (x_1, x_2)$ 且 $x_2 > 50$ ,那在
    食物券下的最適選擇點就會是一個角點解 $C_1 = (50, 50)$。此時食物卷下的滿足水準就會比
    等值現金低,也就是說換成等值現金後滿足水準會提高。這是合理的,因為兩者價值相同,且
    等值現金的自由度較高,消費者可以依自己的喜好做更多的調整。
    

\end{enumerate}

\end{document}

