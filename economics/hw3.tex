\documentclass[12pt, a4paper]{article}

\usepackage[hmargin=2.5cm, vmargin=2cm]{geometry}
\usepackage{amsthm, amssymb, mathtools, yhmath, graphicx}
\usepackage{fontspec, type1cm, titlesec, titling, fancyhdr, tabularx}
\usepackage{color, unicode-math, float, hhline}

\usepackage[CheckSingle, CJKmath]{xeCJK}
\usepackage{CJKulem}
\usepackage{enumitem}
\usepackage{tikz}
\usepackage{circuitikz}
\usepackage{systeme}
%\setCJKmainfont[BoldFont=cwTex Q Hei]{cwTex Q Ming}
%\setCJKsansfont[BoldFont=cwTex Q Hei]{cwTex Q Ming}
%\setCJKmonofont[BoldFont=cwTex Q Hei]{cwTex Q Ming}
\setCJKmainfont[BoldFont=cwTeX Q Hei]{cwTeX Q Ming}

\def\normalsize{\fontsize{12}{18}\selectfont}
\def\large{\fontsize{14}{21}\selectfont}
\def\Large{\fontsize{16}{24}\selectfont}
\def\LARGE{\fontsize{18}{27}\selectfont}
\def\huge{\fontsize{20}{30}\selectfont}

%\titleformat{\section}{\bf\Large}{\arabic{section}}{24pt}{}
%\titleformat{\subsection}{\large}{\arabic{subsection}.}{12pt}{}
%\titlespacing*{\subsection}{0pt}{0pt}{1.5ex}

\parindent=24pt

\DeclarePairedDelimiter{\abs}{\lvert}{\rvert}
\DeclarePairedDelimiter{\norm}{\lVert}{\rVert}
\DeclarePairedDelimiter{\inpd}{\langle}{\rangle}
\DeclarePairedDelimiter{\ceil}{\lceil}{\rceil}
\DeclarePairedDelimiter{\floor}{\lfloor}{\rfloor}

\newcommand{\img}{\mathsf{i}}
\newcommand{\ex}{\mathsf{e}}
\newcommand{\dD}{\mathrm{d}}
\newcommand{\dI}{\,\mathrm{d}}
\usetikzlibrary{intersections, calc}

\title{\vspace{-1.5cm} 經濟學 HW \#3}
\author{b02902072 王鼎皓 \quad b03201001 楊松翰 \quad b02901178 江誠敏}

\begin{document}
\maketitle
{
  \bf 假設需求函數和供給函數分別如下
  \begin{description}[labelindent*=0cm, itemsep=0cm]
    \item[需求:] $ Q = 100 - 0.5 P $
    \item[供給:] $ Q = 10 + P $
  \end{description}
}


\begin{enumerate}[label=(\arabic*)]
  \item {\bf 求出均衡價格 ($P$) 及均衡數量 ($Q$)。}

    解
    \[
      \left\{
    \begin{align*}
      Q_0 + 0.5 P_0 &= 100 \\
      Q_0 - P_0 &= 10
    \end{align*}
      \right.
    \]
    得到均衡價格 $P_0 = 60$,均衡數量 $Q_0 = 70$。

  \item {\bf 若政府對供給方課徵 $15$ 元的從量稅,請問供給函數變成怎樣?
      (用供給方的願受價格 ($P=Q-10$) 的概念表達從量稅的影響。)
      均衡的價格和數量各是多少?消費者(需求)和廠商(供給)各分擔多少稅?}

    課徵 $15$ 元的從量稅後,價格必須要多 $15$ 廠商才願意供給相同的量,因此供給函數
    變成 $P = Q - 10 + 15 = Q + 5$,也就是 $P - Q = 5$。

    解
    \[
      \left\{
    \begin{align*}
      0.5 P_1 + Q_1 &= 100 \\
      P_1 - Q_1 &= 5
    \end{align*}
      \right.
    \]
    得到新的均衡價格 $P_1 = 70$,均衡數量 $Q_1 = 65$。
    
    這 $15$ 元的稅收裡,消費者分擔了 $P_1 - P_0 = 10 \; (\text{元})$,
    廠商分擔了 $P_0 - (P_1 - 15) = 5 \; (\text{元})$。


\end{enumerate}

\end{document}

