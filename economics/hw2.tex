\documentclass[12pt, a4paper]{article}

\usepackage[hmargin=2.5cm, vmargin=2cm]{geometry}
\usepackage{amsthm, amssymb, mathtools, yhmath, graphicx}
\usepackage{fontspec, type1cm, titlesec, titling, fancyhdr, tabularx}
\usepackage{color, unicode-math, float, hhline}

\usepackage[CheckSingle, CJKmath]{xeCJK}
\usepackage{CJKulem}
\usepackage{enumitem}
\usepackage{tikz}
\usepackage{circuitikz}
%\setCJKmainfont[BoldFont=cwTex Q Hei]{cwTex Q Ming}
%\setCJKsansfont[BoldFont=cwTex Q Hei]{cwTex Q Ming}
%\setCJKmonofont[BoldFont=cwTex Q Hei]{cwTex Q Ming}
\setCJKmainfont[BoldFont=cwTeX Q Hei]{cwTeX Q Ming}

\def\normalsize{\fontsize{12}{18}\selectfont}
\def\large{\fontsize{14}{21}\selectfont}
\def\Large{\fontsize{16}{24}\selectfont}
\def\LARGE{\fontsize{18}{27}\selectfont}
\def\huge{\fontsize{20}{30}\selectfont}

%\titleformat{\section}{\bf\Large}{\arabic{section}}{24pt}{}
%\titleformat{\subsection}{\large}{\arabic{subsection}.}{12pt}{}
%\titlespacing*{\subsection}{0pt}{0pt}{1.5ex}

\parindent=24pt

\DeclarePairedDelimiter{\abs}{\lvert}{\rvert}
\DeclarePairedDelimiter{\norm}{\lVert}{\rVert}
\DeclarePairedDelimiter{\inpd}{\langle}{\rangle}
\DeclarePairedDelimiter{\ceil}{\lceil}{\rceil}
\DeclarePairedDelimiter{\floor}{\lfloor}{\rfloor}

\newcommand{\img}{\mathsf{i}}
\newcommand{\ex}{\mathsf{e}}
\newcommand{\dD}{\mathrm{d}}
\newcommand{\dI}{\,\mathrm{d}}
\usetikzlibrary{intersections, calc}

\title{\vspace{-1.5cm} 經濟學 HW \#2}
\author{b02902072 王鼎皓 \quad b03201001 楊松翰 \quad b02901178 江誠敏}

\begin{document}
\maketitle
{\bf 台灣每年用電量(需求)約2266億度(平均每人每年用1萬度電)
  ,其中約有408億度(約占18\%)為核能電廠所發的。}


\begin{enumerate}[label=(\arabic*)]
  \item {\bf 如果要廢止核電,又不開發新電源,就必須每年減少408億度電的需求,若民眾對需求的(自身)價格彈性為0.3,而目前電價為每度3元,請問電費要調升到多少元才能達到此目的?(請以點彈性計算)}

  假設需求數量對價格的函數為 $Q(p)$,且當 $p_0 = 3$ 時,$q_0 = Q(p_0) = 2266$。
  \[
    \frac{Q(p) - Q(p_0)}{p - p_0} \cdot \frac{p_0}{q_0} \approx -(\text{需求價格彈性})
  \]
  因此
  \[ p \approx p_0 + \frac{q_0 - Q(p)}{q_0} p_0 \cdot \frac{1}{\text{需求價格彈性}} \]
  若希望減少 $408$ 億度的電,也就是 $q_0 - Q(p) = 408$,計算後可以得到
  \[ p \approx 3 + \frac{408}{2266} \cdot 3 \cdot \frac{1}{0.3} \approx 4.8 \,(\text{元}) \]

  或者可以把「價格彈性變化」看作是「價格變化比例」和「數量變化比例」的比值,
  要減少 $18 \%$ 的數量,計算可知價格的增加比率為
  \[ 18 \% / 0.3 = 60 \% \]
  因此新的價格大約是
  \[ 3 + 3 \cdot 60 \% = 4.8 \, (元) \]

  \clearpage

  \item {\bf 如果不調整電價,而以鼓勵再生能源發電來彌補廢核的缺口,以目前再生能源發電量約 50 億度,每度收購電費約 5 元
      ,若供給的自身價格彈性為 1.2,政府將收購價格提高 $50\%$ 到 
      $7.5$ 元,請問再生能源發電的供給量會增加到多少?(請以點彈性計算)}

    由 \footnote{第一式左的最後一項應為 $(p_1+p_2) / (q_1+q_2)$,但因為這兩題都把供給/需求曲線
    近似為直線,所以沒有差別}
    \begin{align*}
      & (供給價格彈性) \approx \frac{q_2 - q_1}{p_2 - p_1} \cdot \frac{p_1}{q_1}  \quad \\
      \Rightarrow & \ \left(\frac{p_2}{p_1} - 1\right) \cdot (供給價格彈性) \approx \frac{q_2}{q_1} - 1 \\
      \Rightarrow & \ q_2 \approx q_1 + \left(\frac{p_2}{p_1} - 1\right) \cdot (供給價格彈性) \cdot q_1
    \end{align*}

    代入 $p_2 / p_1 - 1 = 0.5, q_1 = 50$ 計算後得到
    \[ q_2 \approx 50 + 0.5 \cdot 1.2 \cdot 50 \approx 80 \, (億度) \]


  \item {\bf 由 (1) 和 (2) 可知,節省用電比增加再生能源發電效果大。
      電力需求彈性愈高,提高電價愈可抑制用電需求。有什麼辦法可以提高用電的需求彈性?}

    電力需求彈性一般來說應該不高,因為電力幾乎可說是日常生活的必需品,難以被替代。
    且台灣電價便宜,對一般家庭而言電費應在所得比例上不占太多。\footnote{103 年每戶家庭電費支出佔消費支出比率約
      $1.36 \%$,來源:台電網站。}

		若要增加需求彈性,可能有以下方法:
    \begin{enumerate}[label=\arabic*.]
      \item 推擴、補助節能家電:
        這樣當電價上漲時,可能對消費者而言,購買更省電的電子產品來減低
        電費可能會比較划算,因此彈性會增加。
      \item 提早預告漲價:如果突然宣佈漲價,因為電力在生活中難以取代,
        民眾難以在一時之間因應,調整空間不大,彈性自然不高,因此如果
        提早預告漲價,大家才會有時間思考如何省電並執行,彈性才會高。
    \end{enumerate}


\end{enumerate}

\end{document}

