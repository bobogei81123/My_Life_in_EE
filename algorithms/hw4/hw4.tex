\documentclass[12pt, a4paper]{article}

\usepackage[hmargin=2.5cm, vmargin=2cm]{geometry}
\usepackage{amsthm, amssymb, mathtools, yhmath, graphicx}
\usepackage{fontspec, type1cm, titlesec, titling, fancyhdr, tabularx}
\usepackage{color, float, hhline}
\usepackage{scalerel}
%\setmonofont{Source Code Pro}

\usepackage[abbreviations, per-mode=symbol]{siunitx}
\usepackage[CheckSingle, CJKmath]{xeCJK}
\usepackage{CJKulem}
\usepackage{enumitem}
\setlist{itemsep=0pt, topsep=0pt}
\usepackage{tikz}
\usepackage{csquotes}
\usepackage{circuitikz}
\usepackage{footnote}
\usepackage{complexity}
\usepackage[options]{algorithm2e}
\PassOptionsToPackage{hyphens}{url}\usepackage{hyperref}
\makesavenoteenv{table}
\makesavenoteenv{tabular}
%\setCJKmainfont[BoldFont=cwTex Q Hei]{cwTex Q Ming}
%\setCJKsansfont[BoldFont=cwTex Q Hei]{cwTex Q Ming}
%\setCJKmonofont[BoldFont=cwTex Q Hei]{cwTex Q Ming}
\setCJKmainfont[BoldFont=cwTeX Q Hei]{cwTeX Q Ming}
\AtBeginDocument{\mathcode`\_=\string"715F }

\def\normalsize{\fontsize{12}{18}\selectfont}
\def\large{\fontsize{14}{21}\selectfont}
\def\Large{\fontsize{16}{24}\selectfont}
\def\LARGE{\fontsize{18}{27}\selectfont}
\def\huge{\fontsize{20}{30}\selectfont}

%\titleformat{\section}{\bf\Large}{\arabic{section}}{24pt}{}
%\titleformat{\subsection}{\large}{\arabic{subsection}.}{12pt}{}
%\titlespacing*{\subsection}{0pt}{0pt}{1.5ex}

\parindent=0pt
\parskip=0.5em

\DeclarePairedDelimiter{\abs}{\lvert}{\rvert}
\DeclarePairedDelimiter{\norm}{\lVert}{\rVert}
\DeclarePairedDelimiter{\inpd}{\langle}{\rangle} % inner product
\DeclarePairedDelimiter{\ceil}{\lceil}{\rceil}
\DeclarePairedDelimiter{\floor}{\lfloor}{\rfloor}
\DeclareMathOperator{\adj}{adj}
\DeclareMathOperator{\sech}{sech}
\DeclareMathOperator{\csch}{csch}
\DeclareMathOperator{\arcsec}{arcsec}
\DeclareMathOperator{\arccot}{arccot}
\DeclareMathOperator{\arccsc}{arccsc}
\DeclareMathOperator{\arccosh}{arccosh}
\DeclareMathOperator{\arcsinh}{arcsinh}
\DeclareMathOperator{\arctanh}{arctanh}
\DeclareMathOperator{\arcsech}{arcsech}
\DeclareMathOperator{\arccsch}{arccsch}
\DeclareMathOperator{\arccoth}{arccoth}
\underset{x}{\operatorname{argmin}} 
\DeclareMathOperator*{\argmin}{arg\,min}
\DeclareMathOperator*{\argmax}{arg\,max}

\newcommand{\np}[1]{\\[{#1}] \indent}
\newcommand{\transpose}[1]{{#1}^\mathrm{T}}
%%%% Geometry Symbol %%%%
\newcommand{\degree}{^\circ}
\newcommand{\Arc}[1]{\wideparen{{#1}}}
\newcommand{\Line}[1]{\overleftrightarrow{{#1}}}
\newcommand{\Ray}[1]{\overrightarrow{{#1}}}
\newcommand{\Segment}[1]{\overline{{#1}}}
\DeclareMathOperator*{\Bigcdot}{\scalerel*{\cdot}{\bigodot}}

\newcommand{\opord}{\operatorname{\mathcal{O}}}
\newcommand{\ord}[1]{\opord\left(#1\right)}

\newcommand{\img}{\mathsf{i}}
\newcommand{\ex}{\mathsf{e}}
\newcommand{\dD}{\mathrm{d}}
\newcommand{\dI}{\,\mathrm{d}}
\setlength{\droptitle}{-1.5cm} %title 與上緣的間距
\newtheorem{lemma}{Lemma}

\title{Algorithm HW\#3}
\author{B02901178 江誠敏}
%\DeclareMathOperator{\lowb}{lowerBound}

\begin{document}
\maketitle
Collaborators: None

\section{Problem 1.}
We shall define our vertices set as $V = \{ (x_1, x_2, \cdots, x_m) : 0 \leq x_i \leq c_i, \forall i \}$.
Which $(x_1, x_2, \cdots, x_m)$ represent the state such that the $i$-th cup has $x_i$ mls of water.
Now the three types of operations correspond to 3 types of edge.
\begin{enumerate}
  \item Completely fill one cup: For each $1 \leq i \leq m$ and a vertex $(x_1, x_2, \cdots, x_m)$.
    If $x_i < a_i$, there is an edge from $(x_1, \cdots, x_i, \cdots, x_m)$ to
    $(x_1, \cdots, a_i, \cdots, x_m)$ with cost $a_i - x_i$ (since we are minimizing the total water used).
		Let these edges be $E_1$.
  \item Empty one cup: For each $1 \leq i \leq m$ and a vertex $(x_1, x_2, \cdots, x_m)$.
    If $x_i > 0$, there is an edge from $(x_1, \cdots, x_i, \cdots, x_m)$ to
    $(x_1, \cdots, 0, \cdots, x_m)$ with cost $0$.
		Let these edges be $E_2$.
  \item Pour water from cup $c_i$ to cup $c_j$ until either $c_i$ is empty or $c_j$ is full:
 		For each $1 \leq i, j \leq m, i \neq j$ and a vertex $(x_1, x_2, \cdots, x_m)$.
    If $x_i > 0 \text{ and } x_j < a_j$, let $y = \min(x_i, a_j - x_j)$, then there is an edge from
		$(x_1, \cdots, x_i, \cdots, x_j, \cdots, x_m)$ to $(x_1, \cdots, x_i-y, \cdots, x_j+y, \cdots, x_m)$ with cost $0$.
		Let these edges be $E_3$.
\end{enumerate}
Then let $E = E_1 \cup E_2 \cup E_3$. Now the problem become ``Finding the shortest path
from $s = (0, 0, \cdots, 0)$ to $t = (c_1, 0, \cdots, 0)$''. The number of vertex is
$(a_1 + 1) (a_2 + 1) \dots (a_m + 1) = n$. And notice that if $a_i \geq 1, \forall i$
(If some $a_i = 0$, we could just ignore those cups.), Then $\prod (a_i + 1) \geq 2^m \geq m$. 

Now, for a vertex $v = (x_1, x_2, \cdots, x_m)$, there are at most $m$ edges in $E_1$,
at most $m$ edges in $E_2$, at most $m^2$ edges in $E_3$ adjacent to $v$,
so $\ord{E} \leq \ord{m^2 n} \leq \ord{n^3}$.

Hence by using shortest path algorithm, such as Dijkstra's algorithm, the problem could be solve in
$\ord{E + V \log V} = \ord{n^3 + n \log n}$ time, which is polynomial in $n$.

\section{Problem 2.}
First we observe that if $(e_1, e_2, \cdots, e_k)$ are a path, and let $p_i$ be the probability that $e_i$
is cancelled. Then the probability that no flight is cancelled in the path is $\prod (1 - p_i)$.
And since $-\log(\cdot)$ is decreasing in $[0, 1]$, so
\[ \text{Maximize } \prod (1 - p_i) \iff \text{Minimize } - \log \prod(1 - p_i) = \sum -\log(1 - p_i) \]

So if there is a flight from $A_i$ to $A_j$ with cancelling probability $p_{i,j}$, we could think it
as there is an edge from $A_i$ to $A_j$ with cost $c_{i,j} \triangleq - \log (1 - p_{i,j})$.
The problem becomes ``Finding the shortest path from $A_1$ to $A_n$''. And notice that
$1 - p_{i, j} \leq 1 \implies c_{i, j} = - \log(1 - p_{i, j}) \geq 0$. So there is no negative edges,
hence using Dijkstra's algorithm, the problem could be solve in $\ord{E + n \log n}$ where $E$ is the
number of flights.

\section{Problem 3}
Add an additional node $C_0$. We construct the (undirected) edges as follow.
\begin{enumerate}
  \item for all $1 \leq i \leq n$. There is an edge $(C_0, C_i)$ with cost equals $p_i$.
  \item for all $1 \leq i, j \leq n, i \neq j$. There is an edge $(C_i, C_j)$ with cost equals $w_{i, j}$.
\end{enumerate}
Any configuration would correspond to an edge set $E'$ by following: 
\begin{enumerate}
  \item If we construct a power station at city $C_i$, then the edge $(C_0, C_i)$ is in $E'$.
  \item If we construct an undirected power line between $C_i, C_j$, the edge $(C_i, C_j)$ is in $E'$.
\end{enumerate}

Now, if $C_i$ has a power station, then $(C_0, C_i) \in E'$ so $C_i$ is connected to $C_0$ in $E'$.
And if there is a power supply path (a series of power lines) that connect $C_i$ to another city $C_j$ with
a power station, then $C_i$ is connected to $C_j$ and $C_j$ is connected to $C_0$, hence $C_i$ is
connected to $C_0$ in $E'$. Thus the contrain is satisfied if and only if every node $C_i$ is connected
to $V_0$, which is equivalent that every node is connected to each other.

But the odd thing is that the statement says that some costs are negative, so finding a minimum
spanning tree might not be the best solution. (Imagine that all $p_i$ and $w_{i, j}$ are negative.
Then obviously build anything would be the best solution.) What we shall find is not a minimum
spanning tree, but a ``Minimum spanning graph'', which is a subgraph $G' = (V, E')$
in $G$ such that each $v \in V$ is in $G'$ and are connected in $G'$.

So we modify the Kruskal's algorithm as following. It is easy to see that the algorithm is correct.
Since $G'$ need not to be a tree anymore, there is no lost that we choose all
the negative edge to be in $E'$. After choosing these edge, vertices may already form some
connected components. Seeing these components as new ``vertices'',
since if all edge is positive the minimum spanning tree is always one of the
minimum spanning graph, we could applying the normal Kruskal's algorithm and find the answer.
\RestyleAlgo{boxruled}
\LinesNumbered
\begin{algorithm}[H]
  \DontPrintSemicolon
  \SetKwInOut{Input}{input}\SetKwInOut{Output}{output}
  \SetKwFunction{Union}{Union}
  \SetKwFunction{Find}{Find}
  \Input{A graph $G = (V, E)$}
  \Output{Its minimum spanning graph $G' = (V, E')$}
  \BlankLine
  ($E^-, E^+) \gets \big( \{e \in E: e \text{ has cost} < 0\}, \{e \in E: e \text{ has cost} \geq 0\} \big)$ \; 
  $E' \gets \varnothing$ \;
  \ForEach{$e = (u, v) \in E^-$ sorted by cost increasingly}{
    $E' = E' \cup \{(u, v)\}$ \;
    \Union{$u$, $v$} \;
  }
  \ForEach{$e = (u, v) \in E^+$ sorted by cost increasingly}{
    \If{\Find{$u$} $\neq$ \Find{$v$}} {
      $E' = E' \cup \{(u, v)\}$ \;
      \Union{$u$, $v$} \;
    }
  }
  \Return{$G' = (V, E')$\;}
  \caption{Modified Kruskal's algorithm}
\end{algorithm}

The running time is same as Kruskal's algorithm, $\ord{E \log E} = \ord{E \log V}$.

\section{Problem 4}
Let $w_{i}$ be the node represent each TA, so $V = \{C_i\} \cup \{w_i\} \cup \{s, t\} \triangleq C \cup W \cup \{s, t\}$.
Where $s, t$ is the source and sink node which will be used in constructing flow graph.

Now construct (directed) edges as following:
\begin{itemize}
  \item For all $w_i \in W$, $(s, w_i) \in E$ which has capacity $2$.
  \item For all $c_i \in C$, if $w_j \in S_i$, then $(w_i, C_i) \in E$ and has capacity $1$.
  \item For all $c_i \in C$, $(C_i, t) \in E$ and has capacity $n_i$.
\end{itemize}
Let $N = \sum n_i$. We shall proof that a flow with capacity $N$ correspond to
a valid solution and vise versa. But we need the following lemma.
\begin{lemma}
  If $f$ is an $s$-$t$ flow such that each edge has integer flow.
  Then $f$ could be decomposed to unit flow $f = f_1 + f_2 + \dots + f_k$.
  That is, $f_i$ send $1$ unit from $s$ to $t$.
\end{lemma}
\begin{proof}
  Start from $s$, find an edge $(s, u_1)$ which $f(s, u_1) > 0$. Since each vertex $v$ except
  $s, t$ has $f^+(v) = f^-(v)$, we could find $u_2$ so that $f(u_1, u_2) > 0$. Continue the procedure
  and eventually we would reach $t$. That is, we would find a path $s, u_1, u_2, \cdots, t$
  such that each edge in the path has positive flow. Since each edge has integer flow,
  these edge has flow greater than $1$. So let $f_1$ be the flow which has $1$ unit of
  flow on these edge and let $f' = f - f_1$, then $f'$ is still an integer flow.
  By induction we decompose $f$ to unit flows $f_1 + f_2 + \cdots + f_k$.
\end{proof}

By lemma. If $f$ is a flow with size $N$, decomposed $f$ to unit flow $f = f_1 + f_2 + \dots + f_N$.
We know that each flow has the form $s \to w_i \to C_j \to t$. We shall let this flow
represent that ``Letting $w_i$ to be one of the TA of course $C_j$''. Then we check that:

\begin{itemize}
  \item Each $w_i$ would only be TA of at most $2$ courses, since $s \to w_i$ has capacity $2$,
    so there are at most $2$ $f_i$ which start with $s \to w_i \cdots$.
  \item Each pair $(w_i, C_j)$ would only be in one $f_i$ since $w_i \to C_j$ has capacity $1$.
    So there would not be a strange scenario that a TA is counted at two of the TAs of a course $C_j$.
  \item Since $f$ has size $N$, and the sum of capacity of all edges into $t$ is exactly $N$.
    So the flow in these edge has to be equal to their capacity, $n_i$. Hence each course
    would have exactly $n_i$ TAs.
\end{itemize}
Again, since the sum of capacity of all edges into $t$ is exactly $N$, if $f$ is a flow of size $N$,
it is surely a maximum flow. Hence we turn the problem into ``Checking if the maximum flow of the
graph is $N$''. Using any flow algorithm (except Ford-Fulkerson), such as Edmonds-Karp
or Dinic's, the problem could be solved in polynomial time of $V, E$.
For example, if Dinic's algorithm is used, the algorithm runs in $\ord{V E \log V} \leq \ord{V^3 \log V}$.

\section{Problem 5}
We shall proof that ``Set covering'' could be reduced to this problem, and hence
this problem (``Steiner tree in graph'') is $\NP$-complete (Assume that we already know this
problem is in $\NP$.)

Recall that the set covering is state as following: Given $U = \{1, 2, \cdots, n\}$,
$S = \{S_1, \cdots, S_m\}$ with $S_i \subseteq U$, and $k$,
is there a subgroup $T$ of $S$ with size $k$ such that $\bigcup_{S_i \in T} S_i = U$?

Now construct a graph $G = (V, E)$. $V = X \cup Y \cup \{z\}$.
Where
\begin{itemize}
  \item $X \triangleq \{ x_1, x_2, \cdots, x_n\}$, correspond to each element in $U$.
  \item $Y \triangleq \{ y_1, y_2, \cdots, y_m\}$, correspond to each set in $S$.
\end{itemize}
And the (undirected) edges is construct as following:
\begin{itemize}
  \item For all $x_i \in X, y_j \in Y$, $(x_i, y_j) \in E \iff i \in S_j$. These edges has cost $0$.
    Denote these edges by a set $E_0$.
  \item For all $y_j \in Y$, $(y_j, z) \in E$ and has cost $1$. Denote this edge by $e_j$.
\end{itemize}
Let the steiner vertices $V'$ as in the statement to be $V' \triangleq X \cup \{z\}$.
Then, if $T = \{ S_{i_1}, S_{i_2}, \cdots, S_{i_h} \}$ is a set cover of size $h$, then
choose $E' = \{ e_{i_t} : 1 \leq t \leq h \} \cup E_0$, which has cost $h$. Every vertex in $V'$ is
connected to $z$, since for all $i$, $i \in S_\alpha$ for some $S_\alpha \in T$.
Then $x_i \xrightarrow{0} y_\alpha \xrightarrow{e_\alpha} z$ is a path by the fact that $e_\alpha \in E'$.
So every vertex in $V'$ is connected to $z$ and thus they are connected.

Conversely, if $E'$ is a solution to the problem with cost $h$, let $T = \{ S_i : e_i \in E' \}$.
Then $\abs{T} = h$ since $\abs{E' \cap \{ e_i \}} = h$. Then since $E'$ connects all $V'$, so
for each $i$, there is a path from $x_i$ to $z$ and we know that the path has the form
$x_i \to y_j \xrightarrow{e_j} z$, so a $e_j \in E'$, and then $i \in S_j$ and $S_j \in T$.
Thus $S_j$ is a set cover with size $h$.

Hence there is a set cover with size no larger than $h$ if and only if there is a solution
with cost no larger that $h$ of this problem about this certain graph. Hence this problem
is harder than set cover, which is an $\NP$-complete problem. Thus the problem is in $\NP$-complete.

\begin{figure}[H]
  \centering
  \begin{tikzpicture}[
    nd/.style={draw, thick, circle, minimum size=8mm},
    fd/.style={fill=black!20},
    ed/.style={thick},
    tk/.style={line width=0.8mm}
    ]
    \node[nd, fd] (x1) at (0, 0) {$x_1$};
    \node[nd, fd] (x2) at (2, 0) {$x_2$};
    \node[nd, fd] (x3) at (4, 0) {$x_3$};
    \node[nd, fd] (x4) at (6, 0) {$x_4$};
    \node[nd] (y1) at (1, 2) {$y_1$};
    \node[nd] (y2) at (3, 2) {$y_2$};
    \node[nd] (y3) at (5, 2) {$y_3$};
    \node[nd, fd] (z)  at (3, 4) {$z$};
    \draw[ed, tk] (y1) -- (x1);
    \draw[ed, tk] (y1) -- (x2);
    \draw[ed, tk] (y1) -- (x3);
    \draw[ed, tk] (y2) -- (x2);
    \draw[ed, tk] (y2) -- (x4);
    \draw[ed] (y3) -- (x1);
    \draw[ed] (y3) -- (x3);
    \draw[ed, tk] (y1) edge node[above left]{$1$} (z);
    \draw[ed, tk] (y2) edge node[left]{$1$} (z);
    \draw[ed] (y3) edge node[above right]{$1$} (z);
  \end{tikzpicture}
  \caption{The steiner graph corresponding to the set cover problem \\
    $S = \{\{1, 2, 3\}, \{2, 4\}, \{1, 3\}\}$.}
\end{figure}

\end{document}

