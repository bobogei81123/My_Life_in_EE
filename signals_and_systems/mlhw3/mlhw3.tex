\documentclass[12pt, a4paper]{article}

\usepackage[hmargin=2.5cm, vmargin=2cm]{geometry}
\usepackage{amsthm, amssymb, mathtools, yhmath, graphicx}
\usepackage{fontspec, type1cm, titlesec, titling, fancyhdr, tabularx}
\usepackage{color}
\usepackage{unicode-math}
\usepackage{float}
\usepackage{hhline}
\usepackage{comment}
\usepackage[abbreviations]{siunitx}
\usepackage{csvsimple}
\usepackage{subcaption}
\usepackage{cleveref}
\usepackage{listings}
\definecolor{mygreen}{rgb}{0,0.6,0}
\lstset{
  basicstyle=\footnotesize\ttfamily,
  breaklines=true,
  keywordstyle=\color{blue},
  numbers=left,
  numberstyle=\tiny\color{mygray},
  commentstyle=\color{mygreen}, 
}

\usepackage[CheckSingle, CJKmath]{xeCJK}
\usepackage{CJKulem}
\usepackage{enumitem}
\usepackage{tikz}
\usepackage[siunitx]{circuitikz}
\usepackage{wrapfig}
\usepackage{sourcecodepro}
%\setCJKmainfont[BoldFont=cwTex Q Hei]{cwTex Q Ming}
%\setCJKsansfont[BoldFont=cwTex Q Hei]{cwTex Q Ming}
%\setCJKmonofont[BoldFont=cwTex Q Hei]{cwTex Q Ming}
\setCJKmainfont[BoldFont=cwTeX Q Hei]{cwTeX Q Ming}
\setmonofont{Source Code Pro}

\def\normalsize{\fontsize{12}{18}\selectfont}
\def\large{\fontsize{14}{21}\selectfont}
\def\Large{\fontsize{16}{24}\selectfont}
\def\LARGE{\fontsize{18}{27}\selectfont}
\def\huge{\fontsize{20}{30}\selectfont}

\newtheorem{lemma}{Lemma}

%\titleformat{\section}{\bf\Large}{\arabic{section}}{24pt}{}
%\titleformat{\subsection}{\large}{\arabic{subsection}.}{12pt}{}
%\titlespacing*{\subsection}{0pt}{0pt}{1.5ex}

\parindent=24pt

\DeclarePairedDelimiter{\abs}{\lvert}{\rvert}
\DeclarePairedDelimiter{\norm}{\lVert}{\rVert}
\DeclarePairedDelimiter{\inpd}{\langle}{\rangle}
\DeclarePairedDelimiter{\ceil}{\lceil}{\rceil}
\DeclarePairedDelimiter{\floor}{\lfloor}{\rfloor}

\newcommand{\unit}[1]{\:(\text{#1})}
\newcommand{\df}[1]{\mathop{}\!\mathrm{d^#1}}
\newcommand{\img}{\mathrm{i}}
\newcommand{\dD}{\mathrm{d}}
\newcommand{\dI}{\,\mathrm{d}}

\title{ \bf {\Huge Signal and System}\\ MATLAB Homework \#3}
\author{B02901178 江誠敏}

\begin{document}

\maketitle

\section{Problem 1}
\begin{enumerate}[label=(\alph*)]
    \addtocounter{enumi}{1}
  \item {\bf Find the frequency response $H(e^{\img\omega})$ in (a). 
      Plot the magnitude response and the phase response (in degrees) of the filter.
    } \\[12pt]

    \begin{figure}[H]
      \centering
      \begin{subfigure}{0.49\textwidth}
        \includegraphics[width=\textwidth]{fig1-1.eps}
        \caption{Magnitude response}
      \end{subfigure}%
      \begin{subfigure}{0.49\textwidth}
        \includegraphics[width=\textwidth]{fig1-2.eps}
        \caption{Phase response}
      \end{subfigure}
      \caption{Plots of the frequency response}
    \end{figure}

    \clearpage
  \item {\bf Find and plot the filtered signal of $x[n]$.} \\[12pt]

    \begin{figure}[H]
      \centering
      \includegraphics[width=0.6\textwidth]{fig1-3.eps}
      \caption{Plot of filtered $x[n]$.}
    \end{figure}


  \item {\bf Please repeat parts (a)--(c) with $L = 7, f_c = 0.1 \text{ and } f_s = 10$.} \\[12pt]
    \begin{figure}[H]
      \centering
      \begin{subfigure}{0.49\textwidth}
        \includegraphics[width=\textwidth]{fig2-1.eps}
        \caption{Magnitude response}
      \end{subfigure}%
      \begin{subfigure}{0.49\textwidth}
        \includegraphics[width=\textwidth]{fig2-2.eps}
        \caption{Phase response}
      \end{subfigure}
      \caption{Plots of the frequency response}
    \end{figure}

    \begin{figure}[H]
      \centering
      \includegraphics[width=0.5\textwidth]{fig2-3.eps}
      \caption{Plot of filtered $x[n]$.}
    \end{figure}

  \item {\bf Please repeat parts (a)--(c) with $L = 3, f_c = 0.5 \text{ and } f_s = 10$.} \\[12pt]
    \begin{figure}[H]
      \centering
      \begin{subfigure}{0.49\textwidth}
        \includegraphics[width=\textwidth]{fig3-1.eps}
        \caption{Magnitude response}
      \end{subfigure}%
      \begin{subfigure}{0.49\textwidth}
        \includegraphics[width=\textwidth]{fig3-2.eps}
        \caption{Phase response}
      \end{subfigure}
      \caption{Plots of the frequency response}
    \end{figure}

    \begin{figure}[H]
      \centering
      \includegraphics[width=0.5\textwidth]{fig3-3.eps}
      \caption{Plot of filtered $x[n]$.}
    \end{figure}

  \item {\bf What is the effect of increasing $L$? What about increasing $f_c$?}\\[12pt]
    Increasing $L$ makes the cutoff sharper.
    Increasing $f_c$ shift the cutoff frequency to higher frequency.

\section{Problem 2}
\begin{enumerate}[label=(\alph*)]
  \addtocounter{enumi}
  \item {\bf Plot the magnitude response, phase response and impulse response of the filter.}    \\[12pt]
  \begin{center}
    \begin{figure}[H]
      \centering
      \begin{subfigure}{0.49\textwidth}
        \includegraphics[width=\textwidth]{fig4-1.eps}
        \caption{Magnitude response}
      \end{subfigure}%
      \begin{subfigure}{0.49\textwidth}
        \includegraphics[width=\textwidth]{fig4-2.eps}
        \caption{Phase response}
      \end{subfigure}
      \begin{subfigure}{0.49\textwidth}
        \includegraphics[width=\textwidth]{fig4-3.eps}
        \caption{Impulse response}
      \end{subfigure}
      \caption{Plots of the responses}
    \end{figure}
  \end{center}

\item {\bf Change the filter type as ``Chebyshev II" and repeat parts (a)--(b), what is the
  difference?}    \\[12pt]
  \begin{center}
    \begin{figure}[H]
      \centering
      \begin{subfigure}{0.49\textwidth}
        \includegraphics[width=\textwidth]{fig5-1.eps}
        \caption{Magnitude response}
      \end{subfigure}%
      \begin{subfigure}{0.49\textwidth}
        \includegraphics[width=\textwidth]{fig5-2.eps}
        \caption{Phase response}
      \end{subfigure}
      \begin{subfigure}{0.49\textwidth}
        \includegraphics[width=\textwidth]{fig5-3.eps}
        \caption{Impulse response}
      \end{subfigure}
      \caption{Plots of the responses}
    \end{figure}
  \end{center}

  Chebyshev II seems smoother than elliptic filter, and their
  impulse response is different.
\end{enumerate}

\section{Problem 3}
\begin{enumerate}[label=(\alph*)]
  \item {\bf Design a 16-order lowpass filter such that
    \[ y[n] \approx \cos \left( 2 \pi (n-1) T_s \right), \; n = 1, 2, \cdots ,M}, \]
    when $T_s = 0.002, f_1 = 100 \text{ and } M = 1000$. Write down the filter coefficients and plot the
  output signal in your report.}    \\[12pt]
  The coefficients are

  \begin{equation*}
  \begin{split}
    B = [& \num{5.8242e-10},   \num{9.3187e-09},   \num{6.9890e-08}, 
         \num{3.2616e-07},   \num{1.0600e-06}, \\
         & \num{2.5440e-06}, \num{4.6640e-06},   \num{6.6629e-06}, 
         \num{7.4957e-06}, \num{6.6629e-06}, \\
         &\num{4.6640e-06},   \num{2.5440e-06}, 
         \num{1.0600e-06},   \num{3.2616e-07},   \num{6.9890e-08},\\
         &\num{9.3187e-09},   \num{5.8242e-10} ] 
  \end{split}
  \end{equation*}

  \begin{equation*}
  \begin{split}
    A = [&\num{1.0000e+00},  \num{-9.5922e+00},   \num{4.3995e+01},  \num{-1.2779e+02},   \num{2.6265e+02},\\
         &\num{-4.0445e+02}, \num{4.8212e+02},  \num{-4.5335e+02},   \num{3.3956e+02},  \num{-2.0310e+02},\\
         &\num{9.6627e+01},  \num{-3.6160e+01}, \num{1.0429e+01},  \num{-2.2398e+00},   \num{3.3768e-01},\\
         &\num{-3.1918e-02},   \num{1.4244e-03} ]
  \end{split}
  \end{equation*}
  

  \begin{center}
    \begin{figure}[H]
        \centering
        \includegraphics[width=0.6\textwidth]{fig6.eps}
        \caption{Plot of signal after filter}
    \end{figure}
  \end{center}

  \item {\bf Design a 16-order bandpass filter such that
    \[ y[n] \approx \cos \left( 2 \pi f_s (n-1) T_s \right), \; n = 1, 2, \cdots ,M}, \]
    when $T_s = 0.002, f_1 = 100 \text{ and } M = 1000$. Write down the filter coefficients and plot the
  output signal in your report.}    \\[12pt]
  The coefficients are

  \begin{equation*}
  \begin{split}
    B = [& \num{1.1637e-01}, \num{-2.5839e-16}, \num{-1.1507e+00}, \num{-4.7545e-15}, \num{5.7994e+00}, \\
         &\num{-3.2248e-14}, \num{-1.9555e+01}, \num{2.0837e-13}, \num{4.9048e+01}, \num{-9.4593e-13}, \\
         &\num{-9.6595e+01}, \num{3.2545e-12}, \num{1.5406e+02}, \num{-9.5254e-13}, \num{-2.0258e+02},  \\
         &\num{7.4087e-12}, \num{2.2171e+02}, \num{-6.1386e-12}, \num{-2.0258e+02}, \num{7.4219e-12},  \\
         &\num{1.5406e+02}, \num{-2.1432e-12}, \num{-9.6595e+01}, \num{2.2491e-13}, \num{4.9048e+01},  \\
         &\num{1.3561e-13}, \num{-1.9555e+01}, \num{4.1343e-14}, \num{5.7994e+00}, \num{-3.3074e-15}, \\
         &\num{-1.1507e+00}, \num{-1.2920e-16}, \num{1.1637e-01} ]
  \end{split}
  \end{equation*}

  \begin{equation*}
  \begin{split}
    A = [ &\num{1.0000e+00}, \num{-3.8858e-15}, \num{-6.6622e+00}, \num{-1.3234e-13}, \num{2.4075e+01},  \\
          & \num{-3.6238e-13}, \num{-6.0089e+01}, \num{-1.5952e-11}, \num{1.1397e+02}, \num{-9.3806e-11}, \\
          & \num{-1.7227e+02}, \num{-2.9684e-10}, \num{2.1318e+02}, \num{-8.6462e-10}, \num{-2.1919e+02}, \\
          & \num{-1.0678e-09}, \num{1.8858e+02}, \num{-3.6786e-10}, \num{-1.3590e+02}, \num{3.3623e-11}, \\
          & \num{8.1627e+01}, \num{3.8085e-12}, \num{-4.0406e+01}, \num{-1.5429e-11}, \num{1.6158e+01}, \\
          & \num{-4.6336e-12}, \num{-5.0495e+00}, \num{-3.8525e-13}, \num{1.1647e+00}, \num{-1.1713e-14}, \\
  & \num{-1.7760e-01}, \num{-1.9429e-16}, \num{1.3542e-02} ]
  \end{split}
  \end{equation*}

  I bet you would never ever want to look at these ugly coefficients!
  \begin{center}
    \begin{figure}[H]
        \centering
        \includegraphics[width=0.6\textwidth]{fig7.eps}
        \caption{Plot of signal after filter}
    \end{figure}
  \end{center}
  
\end{enumerate}

\end{document}

