\documentclass[12pt, a4paper]{article}

\usepackage[hmargin=2.5cm, vmargin=2cm]{geometry}
\usepackage{amsthm, amssymb, mathtools, yhmath, graphicx}
\usepackage{fontspec, type1cm, titlesec, titling, fancyhdr, tabularx}
\usepackage{color}
\usepackage{unicode-math}
\usepackage{float}
\usepackage{hhline}
\usepackage{comment}
\usepackage[abbreviations]{siunitx}
\usepackage{csvsimple}
\usepackage{subcaption}
\usepackage{cleveref}

\usepackage[CheckSingle, CJKmath]{xeCJK}
\usepackage{CJKulem}
\usepackage{enumitem}
\usepackage{tikz}
\usepackage[siunitx]{circuitikz}
\usepackage{wrapfig}
%\setCJKmainfont[BoldFont=cwTex Q Hei]{cwTex Q Ming}
%\setCJKsansfont[BoldFont=cwTex Q Hei]{cwTex Q Ming}
%\setCJKmonofont[BoldFont=cwTex Q Hei]{cwTex Q Ming}
\setCJKmainfont[BoldFont=cwTeX Q Hei]{cwTeX Q Ming}

\def\normalsize{\fontsize{12}{18}\selectfont}
\def\large{\fontsize{14}{21}\selectfont}
\def\Large{\fontsize{16}{24}\selectfont}
\def\LARGE{\fontsize{18}{27}\selectfont}
\def\huge{\fontsize{20}{30}\selectfont}

\newtheorem{lemma}{Lemma}

%\titleformat{\section}{\bf\Large}{\arabic{section}}{24pt}{}
%\titleformat{\subsection}{\large}{\arabic{subsection}.}{12pt}{}
%\titlespacing*{\subsection}{0pt}{0pt}{1.5ex}

\parindent=24pt

\DeclarePairedDelimiter{\abs}{\lvert}{\rvert}
\DeclarePairedDelimiter{\norm}{\lVert}{\rVert}
\DeclarePairedDelimiter{\inpd}{\langle}{\rangle}
\DeclarePairedDelimiter{\ceil}{\lceil}{\rceil}
\DeclarePairedDelimiter{\floor}{\lfloor}{\rfloor}

\newcommand{\unit}[1]{\:(\text{#1})}
\newcommand{\df}[1]{\mathop{}\!\mathrm{d^#1}}
\newcommand{\img}{\mathrm{i}}
\newcommand{\dD}{\mathrm{d}}
\newcommand{\dI}{\,\mathrm{d}}

\title{ \bf {\Huge Signal and System}\\ Homework \#1}
\author{B02901178 江誠敏}

\begin{document}

\maketitle

\begin{enumerate}
  \item Suppose the output signal of a system is expressed as
    \[ y(t) = \cos \left( \omega_c t + \int_{-\infty}^{t} x(\tau) \dD \tau \right) \]
    where $x(t)$ denotes the input signal and $\omega_c$ is a real-valued constant.
    \begin{enumerate}
      \item Is this system linear? You have to justify your answer. \\[12pt] 
        No. First We proof a lemma.
        \begin{lemma} \label{l1}
          For a linear system $f$, $y = \mathbf{0}$ if $x = \mathbf{0}$,  where
          $\mathbf{0}$ is the zero function $\mathbf{0}(t) = 0 ,\; \forall t$.
          \begin{proof}
            If $x = \mathbf{0}$, by definition, $y(t) = f(x)(t) = f(0x)(t) = 0f(x)(t) = 0, \; \forall t$.
            So $y = \mathbf{0}$.
          \end{proof}
        \end{lemma}
        Now consider the system in the statement. Notice that if $x = \mathbf{0}$, 
        then $y(0) = \cos( \omega_c \cdot 0 + 0 ) = \cos(0) = 1 \neq 0$.
        Hence by \cref{l1}, the system is not linear.
      \item Is this system time-invariant? You have to justify your answer. \\[12pt]
        Case 1: $\omega_c \neq 0$ \\
        No. Consider $x$ to be the zero function $\mathbf{0}$, and let
        $y = f(x)$, where $f$ is the system in the statement. But for any $\tau$, 
        the function $x'$ define by $x'(t) = x(t + \tau)$ is still the zero function,
        so if the system is time-invariant, $y(t+\tau) = f(x')(t) = f(x)(t) = y(t)$ since
        $x' = x = \mathbf{0}$. Hence $y(\tau) = y(0), \; \forall \tau$. That is, $y$ is
        constant. But $y(t) = \cos(\omega_c t + 0) = \cos(\omega_c t)$ isn't a constant
        function if $\omega_c \neq 0$. Hence the system isn't time invariant. \\[10pt]
        Case 2: $\omega_c = 0$ \\
        In this case \[ y(t) = \cos \left( \int_{-\infty}^{t} x(\tau) \dD \tau \right) \],
        for any time-shift function of $x$, say $x',\; x'(t) = x(t+\delta)$, we have
        \begin{align*} 
          f(x')(t) &= \cos \left( \int_{-\infty}^{t} x'(\tau) \dD \tau \right) \\
                   &= \cos \left( \int_{-\infty}^{t} x(\tau+\delta) \dD \tau \right) \\
                   &\stackrel{\tau'= \tau + \delta}{=} \cos \left( \int_{-\infty}^{t+\delta} x(\tau') \dD \tau' \right) \\
                   &= f(x)(t+\tau) = y(t+\tau)
        \end{align*}

      \item Is this system causal? You have to justify your answer.
        The system is causal, since the only thing that is affected by the input is the integral,
        and the integral range satisfied $\tau \leq t$, so the system is casual.
    \end{enumerate}
  \item Determine the signals below as periodic or aperiodic. If periodic identify the 
        fundamental period. You have to justify your answer.
    \begin{enumerate}
      \item $f(t) = \sin(\frac{\pi}{2} t) + \cos(\frac{3\pi}{4} t)$ \\[12pt]
        Let $\alpha = \pi / 2, \beta = 3 \pi / 4$, if $T$ is a period. 
        Since $f(0) = f(0+T) $,
        \begin{align*}
          &\Rightarrow 1 = f(0) = f(T) = \sin(\alpha T) + \cos(\beta T) \\
          &\Rightarrow 1 - \cos(\beta T) = \sin(\alpha T) 
        \end{align*}
        Notice that if a function has period $T$, its derivative would also has period $T$.
        By some calculation, we obtain $f''(t) = -\alpha^2 \sin(\alpha t) - \beta^2 \cos(\beta t)$.
        So by $f''(0) = f''(T) $, 
        \begin{align*} & \Rightarrow -\beta^2 \cos(0 \cdot \beta) = -\beta^2 =
          -\alpha^2 \sin(\alpha T) - \beta^2 \cos(\beta T) \\
          & \Rightarrow \beta^2 ( 1 - \cos(\beta T)) = \alpha^2 \sin(\alpha T) \\
          & \Rightarrow \beta^2 \sin(\alpha T) = \alpha^2 \sin(\alpha T)
        \end{align*}
        Since $\alpha^2 \neq \beta^2$ we have $\sin(\alpha T) = 0$. That is $\alpha T \in \pi \mathbb{N}$.
        \[ T = 2, 4, 6, 8, 10, \cdots \]
        But then by $1 = \sin(\alpha T) + \cos(\beta T) = \cos(\beta T)$, $\beta T \in 2\pi \mathbb{N}$.
        \[ T = 8/3, 16/3, 8, 32/3 \cdots \]
        So the smallest possible $T$ is $8$. Now we only need to check that $8$ is indeed the answer.
        Since 
        \begin{align*}
          f(t+8) &= \sin( (t + 8) \pi/2) + \cos(3 (t + 8) \pi/4) \\ 
                 &= \sin(\pi t/2 + 4\pi) + \cos(3 \pi/4 + 6 \pi)  \\
                 &=  \sin \left(\frac{\pi}{2} t \left) + \cos \left( \frac{3\pi}{4} t \right) \\
                 &= f(t)
        \end{align*}
        So the fundamental period is $T = 8$.
      \item $f(t) = \cos(2 \pi t) + \cos(2 \sqrt{2} \pi t)$ \\[12pt]
        Assume that $T$ is its period. Notice that 
        \[ f(t) \leq \abs{ \cos(2 \pi t) } 
        + \abs{ \cos(2 \sqrt{2} \pi t) } \leq 1 + 1 = 2\]
        and $f(0) = 2$. So $f(T) = 2$ too. Hence the two cosines has to be one at $t = T$
        \[ \Rightarrow a = \frac{T}{2\pi} \in \mathbb{N} ,\quad b = \frac{T}{2\sqrt{2}\pi} \in \mathbb{N} 
        \quad \quad \Rightarrow \quad \frac{2\sqrt{2}\pi}{2\pi} = \frac{a}{b} \in \mathbb{N}\]
        but $ a / b = \sqrt{2} $, which leads to a contradiction since $\sqrt{2}$ is clearly
        irrational.
        

    \end{enumerate}
\end{enumerate}


\end{document}

