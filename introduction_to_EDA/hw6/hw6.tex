\documentclass[12pt, a4paper]{article}

\usepackage[hmargin=2.5cm, vmargin=2cm]{geometry}
\usepackage{amsthm, amssymb, mathtools, yhmath, graphicx}
\usepackage{fontspec, type1cm, titlesec, titling, fancyhdr, tabularx}
\usepackage{color, unicode-math, float, hhline}

\usepackage[CheckSingle, CJKmath]{xeCJK}
\usepackage{CJKulem}
\usepackage{enumitem}
\usepackage{tikz}
\usepackage{circuitikz}
%\setCJKmainfont[BoldFont=cwTex Q Hei]{cwTex Q Ming}
%\setCJKsansfont[BoldFont=cwTex Q Hei]{cwTex Q Ming}
%\setCJKmonofont[BoldFont=cwTex Q Hei]{cwTex Q Ming}
\setCJKmainfont[BoldFont=cwTeX Q Hei]{cwTeX Q Ming}

\def\normalsize{\fontsize{12}{18}\selectfont}
\def\large{\fontsize{14}{21}\selectfont}
\def\Large{\fontsize{16}{24}\selectfont}
\def\LARGE{\fontsize{18}{27}\selectfont}
\def\huge{\fontsize{20}{30}\selectfont}

%\titleformat{\section}{\bf\Large}{\arabic{section}}{24pt}{}
%\titleformat{\subsection}{\large}{\arabic{subsection}.}{12pt}{}
%\titlespacing*{\subsection}{0pt}{0pt}{1.5ex}

\parindent=24pt

\DeclarePairedDelimiter{\abs}{\lvert}{\rvert}
\DeclarePairedDelimiter{\norm}{\lVert}{\rVert}
\DeclarePairedDelimiter{\inpd}{\langle}{\rangle}
\DeclarePairedDelimiter{\ceil}{\lceil}{\rceil}
\DeclarePairedDelimiter{\floor}{\lfloor}{\rfloor}

\newcommand{\img}{\mathsf{i}}
\newcommand{\ex}{\mathsf{e}}
\newcommand{\dD}{\mathrm{d}}
\newcommand{\dI}{\,\mathrm{d}}

\setenumerate{itemsep=0pt,topsep=0pt}
\parindent=0pt
\parskip=\baselineskip

\title{EDA HW5}
\author{B02901178 江誠敏}
\begin{document}

\begin{document}
\maketitle 
\section{Problem 1.}
首先我們讓 $a = 0$ ,這樣在下一個 clock ,$y'_1 = aY_1$ 不論 $Y_1$ 的值
都會是 $0$。 \\
接著在讓 $a = 1$ ,如此在沒有 fault 下  $Y_1 = a y_1 = a \cdot 0 = 0$ , 
而有 stuck at 1 fault 的話 $Y_1 = a y_1 = 1 \cdot 1 = 1$ 。
因此在下一個 clock $y'_1 = Y_1$ 就會有差別了。\\
最後因為 $z = a y_1$ , 讓 $a = 1$ , $z = y_1$ 便可看出上一個的差別。

總結:
$a = (0, 1, 1)$ , 正常情況下 $z = (0, 0, 0)$ , stuck at 1 的情況下
$z = (0, 0, 1)$ 。

\section{Problem 2.}
Proceed as hint, let $f = cg + c'd$, $f_{g=0} \oplus f_{g=1} = c$.
So $c = 1$. Also $g = a+b = 0$ , $a = b = 0$, Hence all the posible 
pattern are $(a, b, c, d) = (0, 0, 1, 0) \text{ and } (0, 0, 1, 1)$.
For both case, if a stuck at 1 ever occurs, $f = 1$, otherwise $f = 0$.


\end{document}

